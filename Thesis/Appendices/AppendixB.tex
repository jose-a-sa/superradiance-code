% !TEX root = ../main.tex

\chapter{Additional Newman-Penrose definitions and computations} % Main chapter title
\label{AppendixNPFormalism}

In this appendix we will present important computations of NP formalism in Kerr background~\eref{eq2:KerrBL}, using the Kinnersley tetrad defined in~\eref{eq3:kinnerslytetrad}.
We will find useful in the one form conversion from the Kinnersley vectors, $(\bm{e}_a)^\flat = (e_a)_\mu \dd x^\mu$,
\begin{align}
    \begin{split}
        \bm{\mathfrak{l}}^\flat =& \frac{1}{\Delta} \Bigr(\Delta, -\rho^2, \,0, -a \Delta \sin^2\theta \Bigr) ~, \\
        \bm{\mathfrak{n}}^\flat =& \frac{1}{2 \rho^2} \Bigr(\Delta, \,\rho^2, \,0, -a \Delta \sin^2\theta \Bigr) ~, \\
        \bm{\mathfrak{m}}^\flat =& \frac{1}{ \sqrt{2} \bar{\rho} } \Bigr( i a \sin\theta, \,0, -\rho^2, - i (r^2+a^2) \sin\theta \Bigr) ~,
    \end{split}
\end{align}
where $\Delta=r^2-2 M r + a^2$, $~\bar{\rho} = r + i a \cos\theta$, $~\rho^2 = \bar{\rho} \bar{\rho}^*$.

\section{Spin coefficients}
\label{AppendixNPSpinCoef}

The spin connection is defined as the covariant derivative of the tetrad field projected onto the tetrad frame. 
For example, we write $\gamma_{412} = \bar{\mathfrak{m}}^\mu \mathfrak{l}_{\mu;\nu} \mathfrak{n}^\nu =  \bar{\mathfrak{m}}^\mu \mathfrak{n}^\nu \nabla_\nu \mathfrak{l}_{\mu} = \bar{\mathfrak{m}}^\mu \mathbbl{\Delta} \mathfrak{l}_\mu$.
It has $24$ components due to the antisymmetry of the first tetrad indices. These can be encapsulated using $12$ complex variables,
\begin{equation}
    \begin{alignedat}{4}
        \kappa  =& \gamma_{311} ~,\qquad \varrho =& \gamma_{314} ~,\qquad \varepsilon =& \tfrac{1}{2} (\gamma_{211} + \gamma_{341}) ~, \\
        \sigma  =& \gamma_{313} ~,\qquad \mu     =& \gamma_{243} ~,\qquad \gamma      =& \tfrac{1}{2} (\gamma_{212} + \gamma_{342}) ~, \\
        \lambda =& \gamma_{311} ~,\qquad \tau    =& \gamma_{312} ~,\qquad \alpha      =& \tfrac{1}{2} (\gamma_{214} + \gamma_{344}) ~, \\
        \nu     =& \gamma_{311} ~,\qquad \pi     =& \gamma_{241} ~,\qquad \beta       =& \tfrac{1}{2} (\gamma_{213} + \gamma_{343}) ~.
    \end{alignedat}
\end{equation}

The computation of these coefficients can be done without computation of the Christopher symbols associated with the covariant derivative.
This is cleverly avoided by observing that for any torsion-free connection, $(e_b)_{[\mu;\nu]} = (e_b)_{[\mu,\nu]}$.
Therefore we define
\begin{align}
    \lambda_{abc} = (e_a)^\mu (e_c)^\nu \left[ (e_b)_{\mu,\nu} - (e_b)_{\nu,\mu} \right]~.
\end{align}
The computation of the various spin coefficients can be easily performed noticing that $\lambda_{abc} = \gamma_{abc} - \gamma_{cba}$, which can be inverted to
\begin{align}
    \gamma_{abc} = \frac{1}{2} \left( \lambda_{abc} + \lambda_{cab} - \lambda_{bca} \right) ~.
\end{align}
All relevant non-vanishing $\lambda$-symbols can be computed by simple coordinate derivatives on the one-form basis,
\begin{equation}
    \begin{alignedat}{3}
        \lambda_{122} =& -\frac{1}{\rho^4} \left[ (r-M) \rho^2 - r \Delta \right] ~,\qquad && \lambda_{314} = - \frac{2 i a \cos\theta}{\rho^2} ~, \\
        \lambda_{132} =& \frac{i \sqrt{2} a r \sin\theta}{\rho^2 \bar{\rho}} ~,\qquad && \lambda_{324} = - \frac{i a \Delta \cos\theta}{\rho^4}~, \\
        \lambda_{213} =& -\frac{\sqrt{2} a^2 \cos\theta \sin\theta }{\rho^2 \bar{\rho}} ~,\qquad && \lambda_{334} = \frac{(i a + r \cos\theta ) \csc\theta }{\sqrt{2} \bar{\rho}^2} ~, \\
        \lambda_{243} =& -\frac{\Delta}{2 \rho^2 \bar{\rho}} ~,\qquad && \lambda_{341} = - \frac{1}{ \bar{\rho}} ~.
    \end{alignedat}
\end{equation}
All other necessary symbols may be found by the symmetry $\lambda_{abc}=-\lambda_{cba}$ or by complex conjugation ($3 \rightleftarrows 4$). For example, for computing the spin coefficient $\mu$, we need to use the relation $\lambda_{432}=-\lambda_{234}=-(\lambda_{243})^*$.
Assembling all symbols, we obtain
\begin{align}
    \begin{split}
        \kappa = \sigma &= \lambda = \nu = 0  ~, \\[0.15cm]
        \varrho = - \frac{1}{\bar{\rho}}  ~,\qquad \mu = - \frac{\Delta}{ 2 \rho^2 \bar{\rho}^*}  ~,&\qquad \tau = - \frac{i a \sin\theta}{\sqrt{2} \rho^2}  ~,\qquad \pi = \frac{i a \sin\theta}{\sqrt{2} (\bar{\rho}^*)^2}  ~, \\[0.15cm]
        \varepsilon = 0  ~,\qquad \gamma = \mu + \frac{r-M}{2 \rho^2}  &~,\qquad \alpha = \pi-\beta^* ~,\qquad \beta = \frac{\cot\theta}{2 \sqrt{2} \bar{\rho} }  ~.
    \end{split}
\end{align}

