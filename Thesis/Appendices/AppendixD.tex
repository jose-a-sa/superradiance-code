% !TEX root = ../Main.tex

\chapter{Eigenvalue small-$c$ expansion}
\label{AppendixEigenvalues}

Using the Leaver continued fraction equation for the eigenvalue, defined in \eref{eq3:evInversion0th}, is possible to expand the eigenvalue for $c\ll 1$,
\begin{align}
    \uu[s]{\mathscr{A}}{\ell m} = \sum_{p=0}^\infty f_p \,c^p ~.
\end{align}
Directed substitution into the continued fraction is done in \cite{Fackerell1977,Seidel1989}, where the coefficients are presented up to $\mathscr{O}(c^6)$.
Defining
\begin{align}
    h(\ell) = \frac{\left(\ell^2-s^2\right) \left[\ell^2-(k_{+}-k_{-})^2\right] \left[\ell^2-(k_{+}+k_{-})^2\right]}{2 \ell^3 \left(\ell^2-\frac{1}{4}\right)} = \frac{2 \left(\ell^2-m^2\right) \left(\ell^2-s^2\right)^2}{\ell^3 \left(4 \ell^2-1\right)}
\end{align}
we may list the series coefficients below,
\begin{subequations}
\begin{align}
    f_0 &= \ell (\ell+1) - s(s+1) ~, \\[0.15cm]
    f_1 &= -\frac{2 m s^2}{\ell (\ell+1)} ~,\\[0.15cm]
    f_2 &= h(\ell+1) - h(\ell) -1 \\[0.15cm]
    f_3 &= 2 m s^2 \bigg[ \frac{ h(\ell)}{(\ell-1) \ell^2 (\ell+1)}-\frac{h(\ell+1)}{\ell (\ell+1)^2
    (\ell+2)} \bigg] ~, \\[0.15cm]
    \begin{split}
        f_4 &= 4 m^2 s^4 \bigg[ \frac{h(\ell+1)}{\ell^2 (\ell+1)^4 (\ell+2)^2} 
        - \frac{h(\ell)}{(\ell-1)^2 \ell^4 (\ell+1)^2} \bigg] 
        +\frac{h(\ell+1)^2}{2 (\ell+1)} - \frac{h(\ell)^2}{2\ell} \\
        &~ + \frac{(\ell-1) h(\ell-1) h(\ell)}{2 \ell (2\ell-1)} + \frac{h(\ell) h(\ell+1)}{2 \ell (\ell+1)} 
        - \frac{(\ell+2) h(\ell+1) h(\ell+2)}{2(\ell+1) (2 \ell+3)} ~,
    \end{split}
\end{align}
\end{subequations}
\begin{subequations}[resume]
\begin{align}
    \begin{split}
        f_5 &= 8 m^3 s^6 \bigg[ \frac{h(\ell)}{(\ell-1)^3 \ell^6 (\ell+1)^3} 
        - \frac{h(\ell+1)}{\ell^3 (\ell+1)^6 (\ell+2)^3} \bigg] \\ 
        &~ + 2 m s^2 \bigg[ \frac{3 h(\ell)^2}{2 (\ell-1) \ell^3 (\ell+1)}
        - \frac{3 h(\ell+1)^2}{2 \ell (\ell+1)^3 (\ell+2)} 
        + \frac{(3 \ell+7) h(\ell+1) h(\ell+2)}{2 \ell (\ell+1)^3 (\ell+3) (2 \ell+3)} \\
        &\qquad - \frac{(3 \ell-4) h(\ell-1) h(\ell)}{ 2(\ell-2) \ell^3 (\ell+1) (2 \ell-1)} 
        - \frac{(7 \ell^2 + 7\ell + 4 ) h(\ell) h(\ell+1)}{2(\ell-1) \ell^3 (\ell+1)^3 (\ell+2)} \bigg]
    \end{split} ~,\\[0.15cm]
    \begin{split}
        f_6 &= \frac{16 m^4 s^8}{\ell^4 (\ell+1)^4} \bigg[ 
        \frac{h(\ell+1)}{(\ell+1)^4 (\ell+2)^4} 
        - \frac{h(\ell)}{(\ell-1)^4 \ell^4} \bigg] \\
        &~ + \frac{4 m^2 s^4}{\ell^2 (\ell+1)^2} \bigg[ 
        \frac{3 h(\ell+1)^2}{(\ell+1)^3 (\ell+2)^2}
        - \frac{3 h(\ell)^2}{(\ell-1)^2 \ell^3}
        - \frac{(3\ell^2+14\ell+17)h(\ell+1)h(\ell+2)}{(\ell+1)^3(\ell+2)(\ell+3)^3(2\ell+3)} \\
        & +\frac{(11\ell^4+22\ell^3+31\ell^2+20\ell+6) h(\ell)h(\ell+1)}{(\ell-1)^2 \ell^3 (\ell+1)^3 (\ell+2)^2}
        + \frac{(3\ell^2 -8\ell +6) h(\ell-1) h(\ell)}{(\ell-2)^2 (\ell-1) \ell^3 (2\ell-1)} \bigg] \\
        & + \frac{h(\ell+1)^3}{2(\ell+1)^2}
        - \frac{h(\ell)^3}{2 \ell^2} 
        -\frac{(\ell-1)^2 h(\ell-1)^2 h(\ell)}{4 \ell^2 (2 \ell-1)^2}
        +\frac{(\ell-1) (7 \ell-3) h(\ell-1) h(\ell)^2}{4 \ell^2 (2 \ell-1)^2} \\
        & +\frac{ (2 \ell^2+4 \ell+3 ) h(\ell)^2 h(\ell+1)}{4 \ell^2 (\ell+1)^2}
        -\frac{ (2 \ell^2+1 ) h(\ell) h(\ell+1)^2}{4 \ell^2 (\ell+1)^2} \\
        & -\frac{(\ell+2) (7 \ell+10) h(\ell+1)^2 h(\ell+2)}{4 (\ell+1)^2 (2 \ell+3)^2}
        +\frac{(\ell+2)^2 h(\ell+1) h(\ell+2)^2}{4 (\ell+1)^2 (2 \ell+3)^2} \\
        & +\frac{(\ell+3) h(\ell+1) h(\ell+2) h(\ell+3)}{12 (\ell+1) (2 \ell+3)^2}  
        +\frac{(\ell+2)(3 \ell^2+2 \ell-3 ) h(\ell) h(\ell+1) h(\ell+2)}{4 \ell (\ell+1)^2 (2 \ell+3)^2} \\
        & -\frac{(\ell-1)(3 \ell^2+4 \ell-2 ) h(\ell-1) h(\ell) h(\ell+1)}{4 \ell^2 (\ell+1) (2 \ell-1)^2} 
        -\frac{(\ell-2) h(\ell-2) h(\ell-1) h(\ell)}{12 \ell (2 \ell-1)^2} ~.
    \end{split}
\end{align}
\end{subequations}
