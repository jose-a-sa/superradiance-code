% !TEX root = ../Main.tex

\chapter{Tetrad techniques} % Main chapter title
\label{AppendixTetradFormalism}


\section{Noncoordinate representation}
\label{AppendixTetradMetric}

The standard way of expressing quantities in GR was to use a local coordinate basis. This corresponds to use
\begin{align}
    \frac{\partial}{\partial x^\mu} \qquad (x^\mu = t, r, \theta, \varphi)
\end{align}
as our vector basis. One-form basis can be defined the usual way. The tetrad formalism allows for an alternative choice of a \emph{noncoordinate} basis, by introducing a set of linear independent four-vectors \cite{Wald2010, Chandrasekhar1998}, 
\begin{align}
    \bm{e}_a = (e_a)^\mu \frac{\partial}{\partial x^\mu} \qquad (a = 1, 2, 3, 4) ~.
\end{align}
We will use Greek alphabet ($\alpha,\beta,\gamma,\dots$) for the coordinate components and the Latin alphabet ($a,b,c,\dots$) for the tetrad components.
The tetrad fields also defined directional derivatives, for example for any scalar field $f$
\begin{align}
    f_{, a} = \bm{e}_a ( f ) = (e_a)^\mu \frac{\partial f}{\partial x^\mu} =  (e_a)^\mu f_{,  \mu}  ~.
\end{align}

However, this formalism must not be mistaken with as change of coordinates, $y=\phi(x)$, such that $(e_a)^\mu = \partial x^\mu / \partial y^a$, since those coordinates may not exist.
A tetrad frame is a pointwise rotation of the coordinate frame, \emph{i.e.} the concept is related to passive transformations rather that active (diffeomorphisms).

Given any tensor field $F_{\mu\nu}$, we can obtain its tetrad components by projecting it onto the tetrad frame,
\begin{align}
    F_{ab} = (e_a)^\mu (e_b)^\nu F_{\mu\nu} ~.
\end{align}
We may invert this expression, by defining the inverse tetrad, $(e^a)_\mu$, such that
\begin{align}
    (e_a)^\mu (e^b)_\mu = (e_a)^\mu (e^b)^\nu g_{\mu\nu} = \delta_a{}^b ~,
\end{align}
hence invariant quantities remain unchanged,
\begin{align}
    \bm{A}^2 = A^\mu A_\mu = A^a (e_a)^\mu \, A_b (e^b)_\mu = A^a A_a~.
\end{align}

We can then substitute the manifold metric for the tetrad ``metric''
\begin{align}
    \eta_{ab} = \bm{e}_a \cdot \bm{e}_b = g_{\mu\nu} (e_a)^\mu (e_b)^\nu ~,
\end{align}
which can be used for raising/lowering tetrad indices,
\begin{align}
    A^a = \eta^{ab} A_b ~,
\end{align}
and to contract tetrad components, such as $\eta_{ab} A^a A^b = \bm{A}^2$. This implies that we may return to the original metric using 
\begin{align}
    g^{\mu\nu} = \eta^{ab} (e_a)^\mu (e_b)^\nu ~.
\end{align}
By analyzing the underlying symmetries of spacetime, one may choose a basis makes the components of $\eta_{ab}$ constant, which is particularly important for the NP formalism.
Going forward, we will assume that this is the case.

\section{Spin connection}
\label{AppendixSpinConnection}

The analogy with the coordinate basics breaks when applying the a directional derivative of a tetrad components, 
\begin{align}
    A_{a,b} = (e_b)^\nu \partial_\nu A_{a} = (e_b)^\nu \nabla_\nu \left[ (e_a)^\mu A_\mu \right] = (e_a)^\mu (e_b)^\nu A_{\mu ; \nu} + (e_c)^\mu (e_a)_{\mu;\nu} (e_b){}^\nu A^c ~,
    \label{eqA:PartialbAa}
\end{align}
due to extra terms resultant of the tetrad derivatives. Tetrad decompositions, $A^a$, are scalars and must not be mistaken as vector fields such as $(e_a)^\mu$ or $A^\mu$. These extra terms can be written using the spin connection
\begin{align}
    \gamma_{cab} = (e_c)^\mu (e_a)_{\mu;\nu} (e_b){}^\nu ~,
\end{align}  
which is antisymmetric in the first two indices,
\begin{align}
    \gamma_{cab} = - \gamma_{acb} ~,
\end{align}
due to the metric compatibility of the covariant derivative, $\nabla_\mu g_{\nu\rho} = 0$.
Nonetheless, this only holds if $\eta_{ab}$ is constant, otherwise we would have $\gamma_{abc} + \gamma_{bac} = \eta_{ab,c}$

The other term is called the \emph{intrinsic} derivative of $A_a$ in the direction of $e_b$, being defined as the projection of the tensor $A_{\mu;\nu}$ in the tetrad frame,
\begin{align}
    A_{a\rvert  b} = (e_a)^\mu (e_b){}^\nu A_{\mu;\nu}
    \label{eqA:intrinsicDerA}
\end{align}
If we have a higher rank tensor, $F_{\mu\nu}$, we can generalize the intrinsic derivative inverting~\eqref{eqA:PartialbAa} and generalizing for multiple indices with the use of the spin connection,
\begin{align}
    F_{ab\rvert c} = F_{ab,c} - \eta^{nm} ( F_{nb} \gamma_{mac} + F_{an} \gamma_{mbc}) ~.
\end{align}

Obviously, the spin connection replaces the Christoper symbols, $\Gamma^\rho_{\mu\nu}$, in the tetrad formalism, although they are fundamentally different.
We will avoid the computation of the Christoper symbols because every equation involving a covariant derivative will become an intrinsic derivative in NP formalism due to tetrad projections. 
This will become useful during calculations as we generally need $\tfrac{1}{2} d^2(d+1)$ computations to fully define the latter, while the spin connection has $\tfrac{1}{2} d^2(d-1)$ independent components.
For $d=4$, the use of the spin connection implies $16$ components less to work with.