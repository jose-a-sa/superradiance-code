% !TEX root = ../main.tex

\chapter{Tetrad techniques} % Main chapter title
\label{AppendixTetrad}

The standard way of expressing quantities in GR was to use a local coordinate basis. This corresponds to use
\begin{align}
   \frac{\partial}{\partial x^\mu} \qquad ( x^\mu = t,r,\theta,\varphi )
\end{align}
as our vector basis. One-form basis can be defined the usual way. The tetrad formalism allows for an alternative choice of a \emph{noncoordinate} basis, by introducing a set of linear independent four-vectors, 
\begin{align}
    e_a = (e_a)^\mu \frac{\partial}{\partial x^\mu} \qquad ( a = 1,2,3,4 ) ~.
 \end{align}
We will use Greek alphabet ($\alpha,\beta,\gamma,\dots$) for the coordinate components and the Latin alphabet ($a,b,c,\dots$) for the tetrad components.

Given any tensor field, we project it to the tetrad frame to obtain its tetrad components,
\begin{align}
    A_{a} = (e_a)^\mu A_{\mu} ~.
\end{align}
We may invert this expression, by defining the inverse tetrad, $(e^a)_\mu$,
\begin{align}
    e_a \cdot e^b = (e_a)^\mu (e^b)_\mu = (e_a)^\mu (e^b)^\nu g_{\mu\nu} = \delta_a{}^b ~,
\end{align}
such that invariant quantities remain unchanged, \emph{i.e}
\begin{align}
    A^2 = A^\mu A_\mu = A^a (e_a)^\mu \, A_b (e^b)_\mu = A^a A_a~.
\end{align}
We can then substitute the manifold metric for
\begin{align}
    \eta_{ab} = e_a \cdot e_b = g_{\mu\nu} (e_a)^\mu (e_b)^\nu ~,
\end{align}
which can be used for raising/lowering tetrad indices and to contract tetrad components.
By analyzing the underlying symmetries of spacetime, one may choose a basis makes the components of $\eta_{ab}$ constant.

The analogy with the coordinate basics breaks when applying the derivative, 
\begin{align}
    A_{a,b} = (e_b)^\mu \nabla_\mu A_{a} = (e_a)^\mu (e_b)^\nu A_{\mu ; \nu} + (e_c)^\mu (e_a)_{\mu;\nu} (e_b){}^\nu A^c ~,
\end{align}
due to extra terms resultant of the tetrad derivatives. These terms can be written using the spin connection
\begin{align}
    \gamma_{cab} = (e_c)^\mu (e_a)_{\mu;\nu} (e_b){}^\nu ~,
\end{align}  
which is antisymmetric in the first two indices, $\gamma_{(ca)b}=0$, due to the metric compatibility of the covariant derivative, $\nabla_\mu g_{\nu\rho} = 0$. 