% !TEX root = ../main.tex

\chapter{Spin-weighted spherical harmonics}
\label{AppendixSWHs}

Spin-weight spherical harmonics \cite{Goldberg1967,Scanio1977,TorresdelCastillo2003} are a generalization of the standard spherical harmonics
found in many well know physical problems such as the hydrogen atom.
They define a set of eigenfunctions which solves the equation
\begin{equation}
	\label{eqA:diffSWSH}
	\frac{1}{\sin\theta} \frac{\dd}{\dd\theta} \left( \sin\theta 
	\frac{\dd \,\uu[s]{Y}{\ell m}}{\dd \theta} \right)
    + \left[ s - \frac{(m + s \cos\theta)^2}{\sin^2\theta} \right] \uu[s]{Y}{\ell m} = - \lambda \;\uu[s]{Y}{\ell m}~,
\end{equation}
with eigenvalues $\lambda= \ell(\ell +1) - s(s+1)$.

These harmonics are complex functions defined on the $S^2$. If take a point in a sphere $(\theta,\varphi)$, we can define a right-handed basis at each point, $\bm{e}_\theta = \partial_\theta$ and $\bm{e}_\varphi = 1/\sin\theta \,\partial_\varphi$, where $\bm{e}_\theta \cdot \bm{e}_\theta = \bm{e}_\varphi \cdot \bm{e}_\varphi = 1$ and $\bm{e}_\theta \cdot \bm{e}_\varphi = 0$.
A given function $f$ defined on $S^2$ is said to have spin-weight $s$ if under the rotation of an angle $\alpha$ of the tangent vectors to the sphere,
\begin{align}
	\label{eqC:spinTransformation}
	\bm{e}_\theta \to \cos\alpha \,\bm{e}_\theta - \sin\alpha \,\bm{e}_\varphi ~, \qquad \bm{e}_\theta \to \sin\alpha \,\bm{e}_\theta + \cos\alpha \,\bm{e}_\varphi ~,
\end{align}
implies that the function transforms as
\begin{align}
	f(\theta, \varphi) \to e^{i s \alpha} f(\theta, \varphi) ~.
\end{align}
In the case of spherical symmetry, $a=0$, we may write the Kinnersly angular vector as $\bm{\mathfrak{m}} = (\bm{e}_\theta + i \,\bm{e}_\varphi)/(\sqrt{2} r^2)$. Under the same transformation, we have $\bm{\mathfrak{m}}\to e^{i\alpha} \bm{\mathfrak{m}}$. From definition \eref{eq3:maxwellNPphi}, since we contract the Maxwell tensor with $\bm{\bar{\mathfrak{m}}}$ once to obtain $\phi_2$, we know that $\phi_2 \to e^{-i \alpha} \phi_2$, thus is has spin-weight $-1$. On the other hand, for gravitational waves the NP scalars $\psi_0$ and $\psi_4$ are double contractions $\bm{\mathfrak{m}}$ and $\bm{\bar{\mathfrak{m}}}$ on the Weyl tensor, respectively. Therefore they are $s=2$ and $s=-2$ quantities, respectively.

All spin-weight spherical harmonics can be obtained using raising and lowering operators on the scalar spherical harmonics. In particular we have that $\uu[0]{Y}{\ell m} = Y_{\ell m}$.
These operators are defined as
\begin{align}
	\begin{split}
		\eth f &= - (\sin{\theta})^s \left\{ \partial_\theta + \frac{i}{\sin{\theta}} \, \partial_\varphi \right\} \left[ (\sin{\theta})^{-s} f \right] 
		= - \left( \partial_\theta + \frac{i}{\sin{\theta}} \, \partial_\varphi - s \cot\theta \right) f ~, \\
		\bar{\eth} f &= - (\sin{\theta})^{-s} \left\{ \partial_\theta - \frac{i}{\sin{\theta}} \, \partial_\varphi \right\} \left[ (\sin{\theta})^{s} f \right]
		= - \left( \partial_\theta - \frac{i}{\sin{\theta}} \, \partial_\varphi + s \cot\theta \right) f ~.
	\end{split}
\end{align}
Is clear from the definition of the operators, that for a function $f$ is a function with spin-weight $s$, then $\eth f$ has spin-weight $s+1$ while $\bar\eth f$ has spin-weight $s-1$, due to an extra $e^{\pm i \alpha}$ factor under the transformation \eref{eqC:spinTransformation}.

Expanding $\eth \bar{\eth}$ we can found the property that for any function $f$ with definite spin-weight, we have
\begin{align}
	\frac{1}{2} (\bar\eth \eth - \eth \bar{\eth} ) f = s f .
\end{align}
This last equation can also be shown using the properties
\begin{align}
	\begin{split}
		\eth ~\uu[s]{Y}{\ell m} = +\sqrt{\ell(\ell+1)-s(s+1)} ~\,\uu[s+1]{Y}{\ell m} ~, \\
		\bar\eth ~\uu[s]{Y}{\ell m} = -\sqrt{\ell(\ell+1)-s(s+1)} ~\,\uu[s-1]{Y}{\ell m} ~.
	\end{split}
\end{align}
We can apply multiple raising and lowering operators to obtain any spherical harmonic, given that $\ell\ge \max\{ |m|, |s| \}$,
\begin{align}
	\begin{split}
	\uu[s]{Y}{\ell m}(\theta,\varphi) &= (-1)^m \sqrt{\frac{2\ell+1}{4 \pi} (\ell+m)! (\ell-m)! (\ell+s)! (\ell-s)!} \\
	&\qquad\qquad \times \sum_{k=0}^{\ell-s} \frac{(-1)^m \left(\sin\tfrac{\theta}{2} \right)^{m+s+2k} \left(\cos\tfrac{\theta}{2} \right)^{2\ell-m-s-2k}}{k!(\ell-m-k)!(\ell-s-k)!(m+s+k)!} e^{i m \varphi}
	\end{split}
\end{align}
For this work, will be useful to list the lowest dipole ($s=-1$, $\ell=1$) spherical harmonics
\begin{align}
	\begin{split}
	\uu[-1]{Y}{1,\pm 1}(\theta,\varphi) &= -\sqrt{\frac{3}{8\pi}} \sin\theta ~,\\
	\uu[-1]{Y}{10}(\theta,\varphi) &= -\sqrt{\frac{3}{16\pi}} (\cos\theta \pm 1) e^{\pm i \varphi} ~,
	\end{split}
\end{align}
while the $s=1$ harmonics can be obtained using properties
\begin{align}
	\begin{split}
		\uu[-s]{Y}{\ell m}(\theta,\varphi)^* &= (-1)^{-s+m} \,\uu[s]{Y}{\ell, -m}(\theta,\varphi) ~, \\
		\uu[-s]{Y}{\ell m}(\pi-\theta,\varphi+\pi)^* &= (-1)^{\ell} \,\uu[s]{Y}{\ell m}(\theta,\varphi) ~.
	\end{split}	
\end{align}
Another possible way of writing the spin-weight spherical harmonics is by using the hypergeometric function,
\begin{align}
	\begin{split}
	\uu[s]{Y}{\ell m}(\theta,\varphi) &= (-1)^m \sqrt{\frac{2\ell+1}{4 \pi} \frac{(\ell+m)! (\ell-m)!}{(\ell+s)! (\ell-s)!}} \left(\sin\tfrac{\theta}{2} \right)^{m+s} \left(\cos\tfrac{\theta}{2} \right)^{2\ell-m-s} \\
	&\qquad\qquad \times \uu[2]{F}{1}\left( m-\ell, s-\ell, m+s+1; - \tan^2\tfrac{\theta}{2} \right) e^{i m \varphi} ~.
	\end{split}
\end{align}

The product of two spin-weighted spherical harmonics with the same argument can be written as a linear combination of other harmonics, admitting a Clebsh-Gordon decomposition,
\begin{align}
	\label{eqC:clebschGordanDecomp}
	\uu[s']{Y}{j' m'} ~\uu[s]{Y}{j m} = \sum_{S,J,M} C_{SJM} ~\uu[S]{Y}{JM} ~,
\end{align}
where
\begin{align}
	\begin{split}
		\label{eqC:clebschGordanCoef}
		C_{SJM} &= (-1)^{j+j'-J} \sqrt{\frac{(2j+1)(2j'+1)}{4\pi(2J+1)}} \\
		&\qquad\qquad \times \langle j',m' ; j, m | J,M \rangle\langle j',s' ; j, s | J, S \rangle \,\delta_{M,m+m'} \,\delta_{S,s+s'} ~,
	\end{split}
\end{align}
with the restriction that the triangle inequality must hold, $|j - j'|\le J \le j+j'$. 

Since these harmonics are generalizations of the standard $s=0$ spherical harmonics, we expect that for each spin-weight $s$ they for an orthogonal and complete set of functions
\begin{align}
	\begin{split}
		\int \dd \Omega ~\,\uu[s]{Y}{\ell' m'}(\theta,\varphi)^* ~\uu[s]{Y}{\ell m}(\theta,\varphi) &= \delta_{\ell\ell'} \,\delta_{mm'} ~,\\
		\sum_{\ell=|s|}^\infty \sum_{m=-\ell}^\ell \uu[s]{Y}{\ell m}(\theta_0,\varphi_0)^* ~\uu[s]{Y}{\ell m}(\theta,\varphi) &= \delta(\cos\theta - \cos\theta_0) \delta(\varphi - \varphi_0) ~,
	\end{split}
\end{align}
so that any spin-weighted $s$ function $f(\theta,\varphi)$ can be written as
\begin{align}
	\begin{split}
		f(\theta,\varphi) = \sum_{\ell=|s|}^\infty \sum_{m=-\ell}^\ell c_{\ell m} ~\uu[s]{Y}{\ell m}(\theta,\varphi) ~,
	\end{split}
\end{align}
so each mode coefficient $c_{\ell m}$ is uniquely defined.