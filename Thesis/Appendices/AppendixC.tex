% !TEX root = ../main.tex

\chapter{Spin-weighted spherical harmonics}
\label{AppendixSWHs}

SWSHs play an important role BH physics and was first introduced by Teukolsky when considering non-scalar wave perturbations on a Kerr background, obtaining a separable master equation in four dimensions. After the usual change of coordinates, the polar differential equation goes as 
\begin{equation}
	\frac{\dd}{\dd x} \left( (1-x^2) \, \frac{\dd S}{\dd x} \right) + (c x)^2 - 2 c s x  -\frac{(m + s x)^2}{1 - x^2} + s  = - \lambda
\label{eqA:diffSWSH}
\end{equation}
with $x=\cos\theta$, where $\lambda$ is the eigenvalue for a given SWSH solution. Periodic boundary conditions on the azimuthal wave function constrains $m$ to the integers.   

\section{Connection with spheroical harmonics}

By setting $s=0$ (scalar) and $c=0$ (spherical), then it's clear that~\eqref{eqA:diffSWSH} appears as a generalization of the spherical harmonics equation. In this last case, the solution are given by the associated Legendre polynomials, $P^m_\ell (x)$, for which the eigenvalue is $\ell(\ell+1)$, restricted to the condition of $|m| \le \ell$. The closed form for spherical harmonics, after normalization, is
\begin{equation}
	\uu[0]{Y^m}{\ell}(x)= (-1)^{m} \sqrt{\frac{(2\ell +1)}{4\pi} \frac{(\ell -m)!}{(\ell +m)!}} \, P_\ell^m (x)
\label{eqA:SH}
\end{equation}
where $P_{\ell}^{m}$ are the associated Legendre polynomials can be obtained using Rodrigues' formula.

\section{Spin raising/lowering differential operators}

\section{Generalized addition of angular momentum formula}

\section{Some useful harmonics}