% !TEX root = ../main.tex

\chapter{Superradiant scattering of plane waves} % Main chapter title
\label{Chapter5}


The prediction of EM and gravitational radiation amplification was a surprising prediction of Einstein's theory of gravitation in Kerr geometry.
A method of direct or indirect observation of this process would provide a probe for rotating BHs and thus it would constitute an important test of GR in regions of extreme gravity.

Having shown that superradiance occurs for small frequencies we owe to find astrophysical sources that emit EM waves.
Binary systems of rotating neutron stars and BH may exhibit the necessary conditions for superradiant scattering.
These objects, also known as pulsars, possess a strong magnetic field with magnetic dipole moment typically not aligned with the rotation axis.
Obviously the magnetic field configuration of a neutron star can be very complicated but its main properties are best described by the \emph{oblique-rotator} model, which considers only the leading order in the multipolar expansion, \emph{i.e.} a magnetic moment dipole
\begin{align}
    \mathbf{m}_P = \frac{m_P}{2} \left[ e^{-i \omega_S t} \sin\alpha_S ( \mathbf{\hat{x}} \pm i \mathbf{\hat{y}}) + \,\text{c.c.} \right] + m_P \cos\alpha_S \,\mathbf{\hat{z}} ~,
\end{align}
where $\omega_S$ is the frequency of rotation.
The upper (lower) sign corresponds to a neutron star co-rotating (counter-rotating) with the BH.
The moment $\mathbf{m}_P$ makes an angle of $\alpha_S$ with respect with the rotation axis, resulting in the precession of the pulsars' magnetic axis, which produces a periodic focused beam of EM radiation.
This periodicity is so precise that makes pulsars ideal for measuring time differences in GR tests.

(IN NEED OF A FIGURE WITH BH AND PULSAR)







\section{Harmonic decomposition}

\section{Phase differences}



\cleardoublepage