% !TEX root = ../main.tex

\chapter{Numerical methods} % Main chapter title
\label{Chapter4}

In this chapter will will develop the necessary method to compute the necessary coefficients to compute the gain/loss factor, using Mathematica\texttrademark.
We will go beyond the spherical approximation and calculate the SWSHs eigenvalues for any BH angular momentum.
With the eigenvalue defined for a particular mode, we will compute the asymptotic radial coefficients, which in turn are used to compute the amplification factor in three diferent ways.

\section{Eigenvalues}

The need for obtaining the eigenvalues $\uu[s]{\mathscr{E}}{\ell m}$ rests on the dependency to solve the radial equation numerically with no spherical approximation.
Additionally, the relative normalization constant $\mathscr{B}$, which depends explicitly on the eigenvalue, will be rather important in one the methods used to calculate the gain/loss factor $\uu[s]{Z}{\ell m}$ for each mode $(\omega,\ell,m)$.
There is no reason to differentiate the eigenvalue for given BH angular momentum and a particular frequency, since the parameter dependence of the eigenvalue is $c=a\omega$.
Considering the focus in superradiant modes, we only will need eigenvalues of coupling which is in the range $0<c<3$. 
Even for extremal BHs, the typical frequency value for a superradiant mode, $\bar{\omega}\sim 1/2$, so this margin is sufficient even for observing the effects in non-superradiant modes.
Due to symmetries XXX, instead of computing for negative values of $c$, we will consider all integer azimuthal numbers $|m|\le \ell$.

\subsection{Leaver method}

The first method implemented was Leaver's.
Consists is using the three recursion relation obtained for SWSHs and correspondent continued fraction~\eref{eq3:evInversion0th} and it's inversions.
Since the problem is now numerical, we have to stop the continued fraction at some particular $p=N$.
By substitution of the parameters $s$ and $m$ and $c$, we are left with an equation with $N$ roots for $\uu[s]{\mathscr{A}}{\ell m}$.
A root-finding algorithm is a method that allows to approximate roots of some equation $f(x)=0$, by suggestion of a connected region were $f$ has different signs at the boundary.
The method ``FindRoot'' in Mathematica\texttrademark~ allows to distinguish the roots of equation by finding the closest to a particular input value.
Firstly, we use the the expansion coefficients for $c\ll 1$~(\aref{AppendixEigenvalues}) to suggest a value of the eigenvalue $\uu[s]{\mathscr{E}}{\ell m}$ that is close to $\ell(\ell+1)$.
We improved on this method by starting the curve at $c=0$, and then obtaining the eigenvalue numerically for small increments in $c$ and then using the last eigenvalue solution as the initial guess for the next increment.
This is particularly useful to generate and save a complete table of eigenvalues for given range and then use interpolation methods to guess eigenvalues for intermediate $c$ values.

For both methods the obtained curves are well behaved for $\ell=1$, but for bigger $\ell$ we start to observe some discontinuities, especially when we increase the range of $c$.

(A PLOT WITH DISCONTINUITIES)

For a fixed $s$ and $m$, we have an infinite number of curves labeled by $\ell$ and in some cases the root finding algorithm selecting roots from adjacent curves, either from the branch $\ell-1$ or $\ell+1$.
These solutions cannot ever intersect, otherwise the eigenvalue would be degenerate and the SWSHs would not be a orthogonal basis of functions.
The issue rests on the lack of accuracy when identifying of the $\ell$-th root.
We lose accuracy when trying to obtaining roots on levels further down in the continued fraction.
We solve the problem by considering the inversion~\eref{eq3:evInversionRth}, choosing $r=\ell+\max\{|m|,|s|\}$, as the main information in the value taken by the $\ell$-th root is in the $\beta_r$, with the fractions providing higher order contributions in $c$.

Once the eigenvalue root is know, one can find any number of the series expansions coefficients $a_p$, for a particular eigenfunction~\eref{eq3:SWSHseriesLeaver}, by using the three-coefficient recursion relation~\eref{eq3:ap3CoefRecursion}.

\subsection{Spectral method}

Due to initial problems with the Leaver method, we decided to use the spectral method.
Because the spheroidal~\eqref{eq3:teukolskyAngular} can be seen as a perturbed version of the spherical case, $c=0$.
We may rewrite the equation using three operators depending on their order in $c$, 
\begin{align}
	\label{eq4:angularEqH012}
	( \mathscr{H}^{(0)} +  \mathscr{H}^{(1)} + \mathscr{H}^{(2)} ) \uu[s]{S}{\ell m} = - \uu[s]{\mathscr{E}}{\ell m}  \;\uu[s]{S}{\ell m} 
\end{align}
The zeroth order operator $\mathscr{H}^{(0)}$, defines the eigenvalue problem for the spin-weighted spherical harmonics, which will provide the complete non-perturbed basis, $\mathscr{H}^{(0)} \,\uu[s]{Y}{\ell m} = -\ell(\ell+1) \,\uu[s]{Y}{\ell m}$.
The other two operators are quickly identified from the angular equation as $\mathscr{H}^{(1)} = - 2 s c \cos\theta$ and $\mathscr{H}^{(2)} = c^2 \cos^2\theta$.
Simple perturbation theory states that
\begin{align}
	\uu[s]{\mathscr{E}}{\ell m} &= \ell(\ell+1) - \int \dd\Omega \,(\uu[s]{Y}{\ell m})^* \,\mathscr{H}^{(1)} \,\uu[s]{Y}{\ell m} + \mathscr{O}(c^2) ~, \\
	\uu[s]{S}{\ell m} &= \uu[s]{Y}{\ell m} - \sum_{j\ne\ell} \frac{\int \dd\Omega \,(\uu[s]{Y}{j m})^* \,\mathscr{H}^{(1)} \,\uu[s]{Y}{\ell m}}{j(j+1)-\ell(\ell+1)} \, \uu[s]{Y}{j m} +\mathscr{O}(c^2) ~.
\end{align}
We may include $\mathscr{H}^{(2)}$ by using a higher order expansion, which can be found in any Quantum Mechanics textbook.
The integral $\int \dd\Omega \,(\uu[s]{Y}{j m})^* \,\mathscr{H}^{(1)} \,\uu[s]{Y}{\ell m}$ may be computed using Clebsch-Gordon coefficients decomposition generalized for spin-weighted functions. The integral must be proportional Leaver series coefficient $f_1$ defined in~\aref{AppendixEigenvalues}.
These operator can be written in terms of general matrix elements in the basis of spin-weighted spherical harmonics,
\begin{align}
	\begin{split}
	h^{(1)}_{j\ell} &= \int \dd\Omega \,\cos\theta \,(\uu[s]{Y}{j m})^* \,\uu[s]{Y}{\ell m} = \sqrt{\frac{2 \ell+1}{2 j+1}} \langle \ell,m; 1,0 | j,m \rangle \langle \ell, -s; 1,0 | j, -s \rangle ~, \\
	h^{(2)}_{j\ell} &= \int \dd\Omega \,\cos^2\theta \,(\uu[s]{Y}{j m})^* \,\uu[s]{Y}{\ell m} = \frac{\delta_{j\ell}}{3} + \frac{2}{3}\sqrt{\frac{2 \ell+1}{2 j+1}} \langle \ell,m; 2,0 | j,m \rangle \langle \ell, -s; 2,0 | j, -s \rangle ~.
	\end{split}
\end{align}

Perturbation theory shows that the SWSHs can be expanded using of spherical harmonics.
This should not be a surprising fact as any angular function $f(\theta,\varphi)$ with a particular spin-weight $s$, can be represented using a decomposition using spin-weighted spherical harmonics.
Having this idea in mind, we write 
\begin{align}
	e^{i m \varphi} \, \uu[s]{S}{\ell m}(c; \theta) = \sum_{j} b_{j}^{(\ell)}(c) \,\uu[s]{Y}{j m}(\theta,\varphi) \qquad \Big( \ell, j \ge \max\{|s|,|m|\} \Big) ~.
\end{align}
Replacing the expansion in~\eqref{eq4:angularEqH012}, we can take advantage of the orthogonality of the harmonics, $\int \dd\Omega \,(\uu[s]{Y}{j m})^* \,\uu[s]{Y}{\ell m} = \delta_{\ell j}$, by multiplying the the equation by $(\uu[s]{Y}{\ell m})^*$ and integrating the solid angle.
The angular equation is replaced by an eigenvalue matrix equation $\sum_{j} a_{ij} \, b_{j}^{(\ell)} = - \uu[s]{\mathscr{E}}{\ell m} \,b_{i}^{(\ell)}$, such that
\begin{equation}
	\label{eq4:spectralMatrix}
	a_{ij} =
	\begin{cases} 
		~c^2 \,h^{(2)}_{ii} - 2 s c \,h^{(1)}_{ii} - i(i+1) & ~ i=j \\[-0.5ex]
		~c^2 \,h^{(2)}_{ij} - 2 s c \,h^{(1)}_{ij} & ~ |i-j|=1 \\[-0.5ex]
		~c^2 \,h^{(2)}_{ij} & ~ |i-j|=2 \\[-0.5ex]
		~0 & ~\text{otherwise}
	\end{cases}  \qquad \Big( i, j \ge \max\{|s|,|m|\} \Big) ~,
\end{equation}
where the the eigenvalues of this matrix are $-\uu[s]{\mathscr{E}}{\ell m}$ and the correspondent eigenvector is given by $b_j^{(\ell)}$.

Like the Leaver method, we will have to truncate the matrix at some finite size.
From \eqref{eq4:spectralMatrix}, we know that the zeroth order contribution to the $\ell$-th eigenvalue will be the element $a_{\ell\ell}$.
We opted to implement a $N\times N$ submatrix such that $i,j\ge\max\{|s|,|m|\}$ and truncating the matrix at $N > \ell - \max\{|s|,|m|\} + 1$, in order to include $a_{\ell\ell}$ in the approximation.
In reality, we must implement a variable $N\equiv N(c)$, so that it increases the size of the taken submatrix in order to include extra corrections for larger values of $c$. The size of submatrix also increases linearly with $\ell$.
The best way to approximate $\uu[s]{\mathscr{E}}{\ell m}$ would be to take a square submatrix so that $a_{\ell\ell}$ is centered and on the diagonal but the first implementation will suffice since we will not take large $\ell$ values.
Optimized numerical methods allow for fast computation of the eigenvalues and eigenvectors of a band-diagonal matrix.
The ``Eigensystem'' method used in Mathematica\texttrademark~ returns a array of positively sorted eigenvalues and their correspondent normalized eigenvectors, guaranteeing~\eqref{eq3:SWSHorthogonality}. As for the eigenvalues we need to select the negative of the ($N-\ell-1+\max\{|s|,|m|\}$)-th element. In order to ensure that the SWSHs have the same phase convections of their spherical counterparts, we must ensure that the correspondent eigenvector has $b_\ell^{(\ell)}>0$, by mapping the obtained vector components as $b_j^{(\ell)}\mapsto \mathrm{sgn}(b_\ell^{(\ell)}) \,b_j^{(\ell)}$.

(EV PLOTS)

\section{Radial equation}

Finding non-approximate form to the amplification factor $\uu[\pm 1]{Z}{\ell m}$ requires the numerical solving of the radial~\eqref{eq3:radialTeukolskyAdimensional}, which is already in an adimensional form.
We computed the angular eigenvalues beforehand, which depend on the mode $(\ell,m)$ as well as the coupling $c=a\omega$.
Additionally the equation depends on the BH parameters $(M,J)$ and $\omega$ explicitly, but it is possible to normalize all variables so that we only need to specify $(\mathscr{J}, \ell, m, \bar{\omega})$, where $\mathscr{J}=a/M=J/M^2$.
We choose to work with barred frequencies because $\bar{\Omega}_H = \mathscr{J}/2$, which makes it easier to numerically select superradiant modes.

We need to obtain numerical interpolations for $\uu[\pm 1]{R}{\ell m}$, by integrating the solution outwards from the horizon, at $x=0$, up to a sufficiently large $x_\infty \gg |\bar{\omega}|^{-1}$. 
The solutions for $s=\pm 1$ contain the all the EM field information, but they have different asymptotic behaviors.
For $\phi_0$, \eqref{eq3:asymptoticR} tells us that the ingoing coefficient tends to overshadow the outgoing coefficient, while the opposite occurs for $\phi_2$.
This way seems natural to try to solve the both equations, where we can obtain $Y_\mathrm{in}$ and $Z_\mathrm{out}$ separately (\tref{tb3:approximatedRsolutionsYZ}).

Knowing the irregularities of the solution at the horizon, we propose the ansatz
\begin{align}
	\label{eq4:numericalRansatz}
	\uu[\pm 1]{R}{\ell m} = (r_+)^{\mp 1} \, x^{ \mp 1 - i \varpi / \tau} f_{\pm}(x) ~,
\end{align}
where $f(x)$ is a new function obeys a regular second-order differential equation.
Thus we need to set two initial conditions at the horizon, $f_{\pm}(0)$ and $f_{\pm}{\!}'(0)$.
We expect to $|f_\pm(x)|$ to become approximately constant to large $x$, because this
form is written in way that also matches the behavior of the radial function at infinity, $\uu[\pm 1]{R}{\ell m}\sim r^{\mp 1}$.

(PLOT SHOWING THAT F IS CONSTANT AT INF)

Comparing \eqref{eq4:numericalRansatz} with the asymptotic form at large $x$ as well as near the horizon, we obtain
\begin{subequations}
	\begin{align}
			\label{eq4:YinFplusInf}
			r_{+} \tau \,\frac{Y_\mathrm{in}}{Y_\mathrm{hole}} &= \frac{f_{+}(x_\infty)}{f_{+}(0)} \exp\left[ - i \bar{\omega} x_\infty - i \bar{\omega} (2-\tau)\log(x_\infty)   -i \frac{\varpi}{\tau} \log(x_\infty) \right] ~, \\[0.15cm]
			\label{eq4:ZoutFminusInf}
			\frac{1}{r_{+} \tau} \,\frac{Z_\mathrm{out}}{Z_\mathrm{hole}} &= \frac{f_{-}(x_\infty)}{f_{-}(0)} \exp\left[ + i \bar{\omega} x_\infty + i \bar{\omega} (2-\tau)\log(x_\infty) -i \frac{\varpi}{\tau} \log(x_\infty) \right] ~.
	\end{align}
\end{subequations}
If both solutions are normalized such that $f_{\pm}(0) = 1$, then we have to deal with the relative normalization of $Z_\mathrm{hole}/Y_\mathrm{hole}$. We can obtain such ratio in terms of known parameters by considering \eqref{eq3:separationB} at $x\simeq0$,
\begin{align}
	(r_{+} \tau)^2 \,\frac{Z_\mathrm{hole}}{Y_\mathrm{hole}} = -\frac{\mathscr{B} \tau^2 }{2 \varpi (i \tau + 2 \varpi)} ~.
\end{align}
Therefore to compute the amplification factor we use
\begin{align}
	\label{eq4:ZFPlusMinus}
	\uu[\pm 1]{Z}{\ell m} = \frac{\mathscr{B}^2 \tau^4 }{4 \varpi^2 (\tau^2 + 4 \varpi^2)} \left| \frac{f_{-}(x_\infty)}{f_{+}(x_\infty)} \right|^2 - 1 ~.
\end{align}

Another way of dealing with the relative normalization would be to select different initial conditions at the horizon $x=0$.
We could cancel the relative normalization of $Z_\mathrm{hole}/Y_\mathrm{hole}$ if we would set any normalization which results in $f_{-}(0)/f_{+}(0) = -\mathscr{B} \tau^2 /[2 \varpi (i \tau + 2 \varpi)]$, eliminating the dependence of $\mathscr{B}$, $\tau$, $\varpi$ in \eqref{eq4:ZFPlusMinus}.

The differential equation obtained from substituting \eref{eq4:numericalRansatz} into \eqref{eq3:radialTeukolskyAdimensional} is identically zero for $x=0$.
Therefore no matter what initial conditions set for $f_{\pm}{\!}'$ the system would not evolve due to stiffness, which makes the step size of the integrator effectively zero.
The usual solution for stiff differential equation is to start the solver a small distance from the horizon $\epsilon>0$.
We adjust the initial conditions by substituting the series expansion of $f_{\pm}(x) = \sum_{n=0}^{N_H} a_n x^n$ in the radial equation, discarding a terms higher than $\mathscr{O}(x^{N_H})$ and obtaining the coefficients $a_n \propto a_0$, $1 \le n \le N_H$. 
Therefore we may set the initial conditions as
\begin{align}
	f_{\pm}(\epsilon) = f_{\pm}(0) \sum_{n=0}^{N_H} \left(\frac{a_n}{a_0}\right) \epsilon^n ~,\qquad f_{\pm}{\!}'(\epsilon) = f_{\pm}(0) \sum_{n=1}^{N_H} \left(\frac{a_n}{a_0}\right) \epsilon^n ~,
\end{align}
$f_{\pm}(0)=1$ are the original horizon conditions consider.
We found $\epsilon=10^{-12}$, $N_H=6$ and $x_\infty = 200 \times 2\pi/|\bar{\omega}|$ working perfectly for the ``NDSolve'' integrator.
Effectively we will have $|f_{\pm}{\!}'(\epsilon)|\simeq \epsilon$, but this contribution is sufficient to remove stiffness from the system and has important contributions in case of extremal BHs ($\mathscr{J}\to 1$).

(PROBABLY SOME PLOTS OF Z, DIFFERENT Ms, ANALYSE PLOT)

This previous method requires us to call the integrator twice which is not very effective numerically.
Exploring the conservation of the wronskian (conserved current) of \eqref{eq3:D2PlusVeff}, we can obtain
\begin{align}
	\frac{\dd E_\mathrm{out}}{\dd t} - \frac{\dd E_\mathrm{in}}{\dd t} = \frac{\dd E_\mathrm{hole}}{\dd t} ~,
\end{align}
which simply states total energy conservation.
Therefore we can rewrite \eqref{eq3:Zdef} only using \emph{hole-in} ratio, thus we are able to get the amplification factor with just with the $s=+1$ solution,
\begin{align}
	\label{eq4:ZFPlusPlus}
	\uu[\pm 1]{Z}{\ell m} = - \frac{\bar{\omega} \tau^2}{\varpi} \left|\frac{f_{+}(0)}{f_{+}(x_\infty)}\right|^2 ~.
\end{align}
If we use the \emph{out-hole} ratio in the amplification factor we only need to solve for $s=-1$,
\begin{align}
	\label{eq4:ZFMinusMinus}
	\uu[\pm 1]{Z}{\ell m} = - \left( 1 + \frac{\mathscr{B}^2 \tau^2}{4 \bar{\omega} \varpi(\tau^2 + 4 \varpi^2)} \left|\frac{f_{-}(x_\infty)}{f_{-}(0)}\right|^2 \right)^{-1} ~.
\end{align}
Thus we have three ways of computing $\uu[\pm 1]{Z}{\ell m}$, which only two of them are independent.
For example, we can relate the RHS of \eqref{eq4:ZFPlusMinus} from both \eref{eq4:ZFPlusPlus} and \eref{eq4:ZFMinusMinus}.

It is expected that if the amplification factors based on only one solution are approximately equal, then the same would be true when considering the third factor, which uses both solutions.
But from a better look at FIGURE. XXX we can see that this fact is not true, especially in higher values of $\ell$.

(PLOTS SHOWING THREE WAYS OF COMPUTING Z, $\ell=1$ and $\ell=2$)

Somehow it appears that we are not compute the ratio of $Y_\mathrm{in}$ and $Z_\mathrm{out}$ with enough accuracy, probably because the large values that $f_{\pm}(x_\infty)$ take do not hold the necessary precision to perform the division.
We owe to correct this problem by considering higher order in the asymptotic expansion of \eqref{eq3:asymptoticR}.
Separately, we will substitute both two asymptotic series in \eqref{eq3:radialTeukolskyAdimensional}, one for the ingoing part and another for the outgoing.
Together they have the form
\begin{align}
	\label{eq4:radialIOseries}
	\uu[-1]{R}{\ell m}(r) &= e^{-i \bar{\omega} x}  \, x^{ - 1 - i (2-\tau) \bar{\omega} } \sum_{n=0}^{N_\infty} i_n \, x^{-n} + \, e^{-i \bar{\omega} x} x^{ 1 + i (2-\tau) \bar{\omega} } \sum_{n=0}^{N_\infty} o_n \, x^{-n} ~,
\end{align}
where we identify $i_0 = Z_\mathrm{in}/r_{+}$ and $o_0 = Z_\mathrm{out} \,r_{+}$.
Although we have chosen the $s=-1$ solution, the same procedure can be done for $s=+1$, because when using a higher order expansion, both ingoing and outgoing coefficients are present.
Firstly, we directly substitute the series into \eqref{eq3:radialTeukolskyAdimensional}, neglecting terms above $\mathscr{O}(x^{N_\infty})$ and grouping the exponentials terms, in order to obtain $i_n\propto i_0$ and $o_n\propto o_0$ ($1 \le n \le N_\infty$), exactly like the horizon series used above to define boundary conditions at the horizon.
Secondly, substitution of the numerical ansatz \eref{eq4:numericalRansatz} in the LHS of the previous equation, together with it's derivative, we have a system two linear equations, which in the limit of large-$x$ limit allows to determine
\begin{align}
	\frac{1}{r_+} \,Z_\mathrm{in} = i_0\Big( f_{-}(x_\infty),  f_{-}{\!}'(x_\infty) \Big) ~,\qquad r_{+} \,Z_\mathrm{out} = o_0\Big( f_{-}(x_\infty),  f_{-}{\!}'(x_\infty) \Big) ~.
\end{align}
Lastly, we may use previous expression~\eref{eq3:amplificationBAoutAin} to compute $\uu[\pm1]{Z}{\ell m}$.
FIGURE. XXX shows for a smaller $x_\infty = 80 \times 2 \pi / |\bar{\omega}|$ and $N_\infty = 10$, with the same $\epsilon = 10^{-12}$, that this new method solves the precision problems from the first method when using both $\phi_0$ and $\phi_2$. 





\cleardoublepage