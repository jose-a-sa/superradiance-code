% !TEX root = ../main.tex

\chapter{Teukolsky master equation} % Main chapter title
\label{Chapter3}

\section{Newman-Penrose formalism}

Study of gravitational and electromagnetic perturbations in a BH background were performed long before Kerr found his solution, for other spacetimes such as Schwarzschild's. Despite it's simplicity, the procedure involved was already algebraically tedious. In the Kerr case, the metric was far more complicated, making the problem almost untreatable.

Fortunately, the NP formalism~\cite{Newman1962} provides an alternative method of studying perturbations.
Results as a natural introduction of spinor techniques into GR, after the choice of a null complex tetrad basis,
\begin{align}
    e_a = (e_a)^\mu \frac{\partial}{\partial x^\mu} \qquad (a = 1, 2, 3, 4) ~,
\end{align}
where all quantities will be projected, \emph{i.e.} for the Weyl tensor we define
\begin{align}
    C_{abcd} =  (e_a)^\alpha  (e_b)^\beta  (e_c)^\gamma  (e_d)^\delta  C_{\alpha\beta\gamma\delta} ~.
\end{align}
Penrose believed that the light-cone was the essential element of the spacetime, thus it was of importance to find null directions. The basis consisted in two real vectors, $l$ and $n$, and two complex conjugate vectors $m$ and $\bar{m}$. Besides satisfying
\begin{align}
    l^2 = n^2 = m^2 = \bar{m}^2 = 0 ~,
\end{align}
orthogonality conditions of NP formalism require
\begin{align}
    l \cdot m = l \cdot \bar{m} = n \cdot m = n \cdot \bar{m} = 0 ~.
\end{align}
Still we are left with the ambiguity raised by multiplication of scalar functions to each vector, therefore its customary to impose normalization conditions to the basis,
\begin{align}
    l \cdot n = 1 ~, \qquad m \cdot \bar{m} = -1 ~.
\end{align}
This formalism is a special case of tetrad calculus, where we can identify the new basis as $(l,n,m,\bar{m})$. The ``metric'' for manipulating tetrad indices, $\eta_{ab}$, is defined by all restrictions provided above,
\begin{align}
    g^{\mu\nu} = \eta^{ab} (e_a)^\mu (e_b)^\nu = l^{\mu} n^{\nu} + n^{\mu} l^{\nu} - m^{\mu} \bar{m}^{\nu} - \bar{m}^{\mu} m^{\nu} ~.
\end{align}
Additionally this vectors define new directional derivatives.
We will depart shortly from standard notation~\cite{Teukolsky1972,Teukolsky1973,Teukolsky1974}, by redefining these derivatives as
\begin{align}
    \mathbbl{D}=\nabla_l ~,\qquad \mathbbl{\Delta}=\nabla_n ~,\qquad \bbdelta=\nabla_m ~,\qquad \bar{\bbdelta}=\nabla_{\bar{m}} ~.
    \label{eq3:tetradCovDer}
\end{align}

More details and definitions on the tetrad formalism can be found in~\aref{AppendixTetradFormalism}.

\subsection{Kinnersley tetrad}

The Riemann tensor may have up to twenty non-vanishing components.
We know that ten of these are present in the symmetric Ricci tensor, that is intrinsically connected to matter and energy.
The other components are pure gravitational degrees of freedom and are encoded in the Weyl tensor.
It becomes the most useful object when the Ricci tensor vanishes, such as vacuum solutions and source-free gravitational waves.
In order to remove the Ricci tensor degrees of freedom, the tensor must be constructed trace-free,
\begin{align}
    \eta^{ad} C_{abcd} = C_{1bc2} + C_{1bc2} - C_{3bc4} - C_{4bc3} = 0 ~. 
\end{align}
Together with the other symmetries inherited from the Riemann tensor, for instance the first Bianchi identity, $C_{a[bcd]}=0$, it is possible to vanish some components and rewrite others such that only ten degrees of freedom remain.
As a result, in NP formalism the Weyl tensor can be represented by five complex scalars, usually chosen as
\begin{align}
    \begin{split}
        \psi_0 &= - C_{1313} = - C_{\alpha\beta\gamma\delta}\, l^\alpha m^\beta l^\gamma m^\delta ~,\qquad
        ~\psi_1 = - C_{1213} = - C_{\alpha\beta\gamma\delta}\, l^\alpha n^\beta l^\gamma m^\delta ~,\\
        \psi_2 &= - C_{1342} = - C_{\alpha\beta\gamma\delta}\, l^\alpha m^\beta \bar{m}^\gamma n^\delta ~,\qquad
        \psi_3 = - C_{1242} = - C_{\alpha\beta\gamma\delta}\, l^\alpha n^\beta \bar{m}^\gamma n^\delta ~,\\
        \psi_4 &= - C_{2424} = - C_{\alpha\beta\gamma\delta}\, n^\alpha \bar{m}^\beta n^\gamma \bar{m}^\delta ~.
    \end{split}
\end{align}
The complex conjugates can be obtained by doing the replacement $3 \rightleftarrows 4$, by exchanging $m$ with $\bar{m}$ and vice-versa. 
Weyl tensor has a unique decomposition in therms of a linear combination of NP scalars and tensorial product of two-forms such as $l_{[\mu} n_{\nu]}$ and $m_{[\rho} \bar{m}_{\sigma]}$. 
It is clear that the values that these five complex scalars take is completely dependent on the choice of tetrad frame. 

BH solutions are ``type D'' spacetimes according to Petrov's classification, which was a major restriction necessary to the discovery of Kerr's metric.
For these spacetimes it is possible to find two different doubly-degenerate principal directions of the Weyl tensor, which we choose to be the real vectors of the tetrad, $l$ and $n$~\cite{Chandrasekhar1998}.
These yield
\begin{align}
    C_{\mu\alpha\beta[\nu} l_{\rho]} l^\alpha l^\beta = 0 ~, \qquad C_{\mu\alpha\beta[\nu} n_{\rho]} n^\alpha n^\beta = 0  ~.
\end{align}
In NP formalism terms, this implies, respectively,
\begin{align}
    \psi_0=\psi_1=0 ~,\qquad \psi_3=\psi_4=0 ~.
\end{align}

Finding the principal directions may not be trivial, but we can apply successive local transformations of the six-parameter Lorentz group in order to rotate the tetrad vectors. 
This procedure allows for the simplification of the Weyl tensor by vanishing NP scalars, ``locking'' the orientation of the tetrad frame.
Weyl scalar $\psi_2$ becomes invariant under boosts in the principal directions.
These keep the light-cone structure intact by maintaining the direction of $l$ and $n$ unchanged (up to multiplication of scalar functions), being useful to change between ingoing and outgoing frames~\cite{Teukolsky1974}. 
Kinnersly solved type D vaccum field equations~\cite{Kinnersley1969}, finding a suitable tetrad 
\begin{align}
    \begin{split}
        l =& \left(\frac{r^2+a^2}{\Delta}, \, 1, \,0, \,\frac{a}{\Delta} \right) ~, \\
        n =& \frac{1}{2 \rho^2} \Bigr( r^2+a^2, \,-\Delta, \,0 , \,a \Bigr) ~, \\
        m =& \frac{1}{ \sqrt{2} \bar{\rho}^2 } \Bigr( i a \sin\theta, \,0, \,1, \, i \csc\theta \Bigr) ~,
    \end{split}
    \label{eq3:kinnerslytetrad}
\end{align}
where $\bar{\rho} = r + i a \cos\theta$ and $\rho^2 \equiv | \bar{\rho} |^2 = \bar{\rho} \bar{\rho}^*$.

The NP formalism provides a full set of first-order coupled diferential equations, relating the NP scalar (Weyl and Maxwell tensors) and the spin-coeficients resultant of the Kinnersley tetrad. These equations result from second Bianchi identity, $C_{\mu\nu[\rho \sigma ; \lambda ]} = 0$, and~\eqref{eq2:maxwellEM}.
To study GWs, instead of perturbing the background metric, the NP formalism provides a natural way of performing perturbations by modification of the tetrad, $l=l^B+l^P$, $n=n^B+n^P$, etc., and also the NP scalars, $\psi_a = \psi_a{}^B + \psi_a{}^P$, maintaining only first-order terms.
The formalism reveals decoupled equations for $\psi_0{}^P$ and $\psi_4{}^P$, which implies that these dynamic variables are the only independent degrees of freedom of the GWs.

\subsection{Maxwell equations}

We focus with more detail on EM perturbations with a fixed background because is a simpler procedure and then we will tie with the same master equation that also describes GW perturbations. 

In NP formalism, all Maxwell equations, $F_{[\mu\nu ; \rho]}=0$ and~\eqref{eq2:maxwellEM}, reduce to
\begin{align}
    F_{[ab \rvert c]} = 0 ~,\qquad \eta^{bc} F_{ab \rvert c} = 0 ~.
    \label{eq3:maxwellFabEqs}
\end{align}
The Mawell tensor $F_{\mu\nu}$ has a total of six components which encodes the vector quantities of the electric and the magnetic fields. We may reduce the equation using three complex NP scalars,
\begin{align}
    \begin{split}
        \phi_0 &= F_{13} = F_{\alpha\beta} l^\alpha m^\beta ~,\qquad
        \phi_1 = \tfrac{1}{2} (F_{12} + F_{43}) = \tfrac{1}{2} F_{\alpha\beta} (l^\alpha n^\beta + \bar{m}^\alpha m^\beta) ~,\\
        \phi_2 &= F_{42} = F_{\alpha\beta} \bar{m}^\alpha n^\beta ~.
    \end{split}
    \label{eq3:maxwellNPphi}
\end{align}
Considering all posible combinations of NP indices in~\eref{eq3:maxwellFabEqs}, we gather eight equations, double the amount of necessary relations. 
This occurs because the conjugates $\phi_0^*$, $\phi_1^*$, $\phi_2^*$ are coupled in these equations. Eliminating every term of the form $F_{23\rvert a}$ or $F_{14\rvert b}$, 
\begin{subequations}
    \begin{align}
        \phi_{2\rvert 1} &= \phi_{1\rvert 4} ~, \label{eq3:phi21phi14}\\
        \phi_{1\rvert 2} &= \phi_{2\rvert 3} ~, \label{eq3:phi12phi23}\\
        \phi_{1\rvert 1} &= \phi_{0\rvert 4} ~, \label{eq3:phi11phi04}\\
        \phi_{0\rvert 2} &= \phi_{1\rvert 3} ~. \label{eq3:phi02phi13}
    \end{align}
\end{subequations}
We may expand explicitly the left-hand side of~\eqref{eq3:phi21phi14},
\begin{align}
    \begin{split}
        \phi_{2\rvert 1} &= \phi_{2, 1} - \eta^{ab}( \gamma_{a41} F_{b2} + \gamma_{a21} F_{4b} ) \\
        &= \phi_{2, 1} - (\gamma_{241} F_{12} + \gamma_{121} F_{42}) + ( \gamma_{341} F_{42} + \gamma_{421} F_{43} ) \\
        &= \phi_{2, 1} + 2 F_{42} \left(\frac{\gamma_{341} + \gamma_{211}}{2}\right) + 2 \gamma_{421} \left(\frac{F_{12} + F_{43}}{2}\right) \\
        &= \mathbbl{D} \phi_2 + 2 \varepsilon \phi_2 - 2 \pi \phi_1 ~,
    \end{split}
    \label{eq3:phi21SpinCoef}
\end{align}
where we used the antisymmetry of the spin connection, $\gamma_{abc}=-\gamma_{bac}$. The right-hand side yields
\begin{align}
    \begin{split}
        \phi_{1\rvert 4} &= \phi_{1, 4} - \tfrac{1}{2} \eta^{ab}( \gamma_{a14} F_{b2} + \gamma_{a24} F_{1b} + \gamma_{a44} F_{b3} + \gamma_{a34} F_{4b} ) \\
        &= \phi_{1, 4} - \tfrac{1}{2} ( \gamma_{144} F_{23} + \gamma_{134} F_{42} + \gamma_{214} F_{12} + \gamma_{234} F_{41} ) \\ 
        &\qquad\qquad + \tfrac{1}{2} ( \gamma_{314} F_{42} + \gamma_{414} F_{42} + \gamma_{324} F_{14} + \gamma_{424} F_{13} ) \\
        &= \phi_{2, 1} - \gamma_{244} F_{13} + \gamma_{314} F_{42} \\
        &= \bar{\bbdelta} \phi_1 - \lambda \phi_0 + \tau \phi_2  ~.
    \end{split}
    \label{eq3:phi21SpinCoef}
\end{align}
The spin coefficients $\varepsilon$, $\pi$, $\lambda$, $\tau$, along with other NP definitions are found in~\aref{AppendixNPSpinCoef}. If we repeat the same expansion for the other Maxwell equations, we gather the set
\begin{subequations}
    \begin{align}
        \label{eq3:phi21phi14SpinCoef}
        \mathbbl{D} \phi_2 - \bar{\bbdelta} \phi_1 &= -\lambda \phi_0 + 2 \pi \phi_1 + (\varrho - 2 \varepsilon) \phi_2 ~,\\
        \label{eq3:phi12phi23SpinCoef}
        \mathbbl{\Delta} \phi_1 - \bbdelta \phi_2 &= \nu \phi_0 - 2 \mu \phi_1 + (2\beta-\tau) \phi_2 ~,\\
        \label{eq3:phi11phi04SpinCoef}
        \mathbbl{D} \phi_1 - \bar{\bbdelta} \phi_0 &= (\pi - 2 \alpha) \phi_0 + 2 \varrho \phi_1 - \kappa \phi_2 ~,\\
        \label{eq3:phi02phi13SpinCoef}
        \mathbbl{\Delta} \phi_0 - \bbdelta \phi_1 &= (2 \gamma - \mu) \phi_0 - 2 \tau \phi_1 + \sigma \phi_2 ~.
    \end{align}
\end{subequations}
The Kinnersley tetrad guarantees that $\kappa = \sigma = \lambda = \nu = 0$, decoupling all equations above. After substitution of all spin coefficients,
\begin{subequations}
    \begin{align}
        \label{eq3:phi21phi14Expand}
        \left( \mathbbl{D} + \frac{1}{\bar{\rho}^*} \right) \phi_2 &= 
        \left( \bar{\bbdelta} + \frac{2 i a \sin\theta}{\sqrt{2} (\bar{\rho}^*)^2} \right) \phi_2 ~,\\
        \label{eq3:phi12phi23Expand}
        \left( \mathbbl{\Delta} - \frac{\Delta}{\rho^2 \bar{\rho}^*} \right) \phi_1 &= 
        \left[ \bbdelta + \frac{1}{\sqrt{2} \bar{\rho}} \left( \cot\theta - \frac{i a \sin\theta}{\bar{\rho}^*} \right) \right] \phi_2 ~,\\
        \label{eq3:phi11phi04Expand}
        \left( \mathbbl{D} + \frac{2}{\bar{\rho}^*} \right) \phi_1 &=
        \left[ \bar{\bbdelta} + \frac{1}{\sqrt{2} \bar{\rho}^*}\left( \cot\theta - \frac{i a \sin\theta}{\bar{\rho}^*} \right) \right] \phi_0 ~, \\
        \label{eq3:phi02phi13Expand}
        \left[ \mathbbl{\Delta} + \frac{\Delta}{2 \rho^2} \left( \frac{1}{\bar{\rho}^*} - \frac{2(r-M)}{\Delta} \right) \right] \phi_0 &=
        \left(\bbdelta + \frac{2 i a \sin\theta}{\sqrt{2} \bar{\rho} \bar{\rho}^*}\right) \phi_1  ~.
    \end{align}
    \label{eq3:phiAllExpand}
\end{subequations}

An important consequence of the symmetries of Kerr spacetime allows for a wave decomposition of the form $\phi_0, \phi_1, \phi_2 \sim e^{- i \omega t + i m \varphi}$.
Therefore, the four differential operators group into radial $(\mathbbl{D}, \mathbbl{\Delta})$ and angular $(\bbdelta, \bar{\bbdelta})$. The procedure for separation of the Maxwell equations can be further simplified by introducing new operators
\begin{equation}
    \begin{alignedat}{3}
        \mathscr{D}_n &= \partial_r - \frac{i \mathscr{K}}{\Delta} + 2n \frac{r-M}{\Delta} ~,\qquad && \mathscr{D}_n^\dagger = \partial_r + \frac{i \mathscr{K}}{\Delta} + 2n \frac{r-M}{\Delta} ~,\\
        \mathscr{L}_n &= \partial_\theta - \mathscr{Q} + n \cot\theta ~,\qquad && \mathscr{L}_n^\dagger = \partial_\theta + \mathscr{Q} + n \cot\theta ~,
    \end{alignedat}
\end{equation}
where we define the functions $\mathscr{K}=(r^2+a^2)\omega - m a$, $\mathscr{Q} = a \omega \sin\theta - m \csc\theta$.
In this definition, $n$ is a non-negative integer.
These operators are related to the tetrad by
\begin{align}
    \mathbbl{D} = \mathscr{D}_0 ~,\qquad \mathbbl{\Delta} = - \frac{\Delta}{2 \rho^2 }\mathscr{D}^\dagger_0 ~,\qquad \bbdelta = \frac{1}{\sqrt{2} \bar{\rho}} \mathscr{L}^\dagger_0 ~,\qquad \bar\bbdelta = \frac{1}{\sqrt{2} \bar{\rho}^*} \mathscr{L}_0 ~,
\end{align}
as a result of the substitutions $\partial_t\rightarrow-i \omega$, $\partial_\varphi\rightarrow i m$.
We may use the fact that $\mathscr{D}_n$ and $\mathscr{L}_n$ act mostly as radial and angular derivatives, respectively, to deduce the properties
\begin{subequations}
    \begin{align}
        \label{eq3:propDeltaD}
        \mathscr{D}_n \Delta &= \Delta \mathscr{D}_{n+1} ~, \\[0.15cm]
        \label{eq3:propSinL}
        \mathscr{L}_n \sin\theta &= \sin\theta\, \mathscr{L}_{n+1} ~, \\[0.15cm]
        \label{eq3:propBarRhoD}
        \left(\mathscr{D}_n + \frac{q}{\bar{\rho}^*} \right) \frac{1}{(\bar{\rho}^*)^p} &= 
        \frac{1}{(\bar{\rho}^*)^p} \left(\mathscr{D}_n + \frac{q-p}{\bar{\rho}^*} \right) ~, \\[0.15cm]
        \label{eq3:propBarRhoL}
        \left(\mathscr{L}_n + \frac{i q a \sin\theta}{\bar{\rho}^*} \right) \frac{1}{(\bar{\rho}^*)^p} &= 
        \frac{1}{(\bar{\rho}^*)^p} \left(\mathscr{L}_n + \frac{i (q-p) a \sin\theta}{\bar{\rho}^*} \right) ~, \\[0.15cm]
        \label{eq3:propCommutLD}
        \left(\mathscr{D}_n + \frac{q}{\bar{\rho}^*} \right) 
        \left(\mathscr{L}_n + \frac{i q a \sin\theta}{\bar{\rho}^*} \right) &= 
        \left(\mathscr{L}_n + \frac{i q a \sin\theta}{\bar{\rho}^*} \right)
        \left(\mathscr{D}_n + \frac{q}{\bar{\rho}^*} \right) ~,
    \end{align}
\end{subequations}
for any integers $p,q,n$, holding also for either $\mathscr{D}^\dagger_n$ or $\mathscr{L}^\dagger_n$.

In order to achieve the separable form, we still need to perform a replacement of the Maxwell NP scalars by new dynamical variables
\begin{align}
    \label{eq3:phiBarRhoToPhi}
    \Phi_0 = \phi_0 ~,\qquad \Phi_1 = \sqrt{2} \bar{\rho}^* \phi_1 ~,\qquad \Phi_2 = 2 (\bar{\rho}^*)^2 \phi_2  ~,
\end{align}
and using properties~\eref{eq3:propBarRhoD} and~\eref{eq3:propBarRhoL}, we go from Eqs.~\eref{eq3:phiAllExpand} to
\begin{subequations}
    \begin{align}
        \label{eq3:D0Phi2L0Phi1}
        \left( \mathscr{D}_0 - \frac{1}{\bar{\rho}^*} \right) \Phi_2 &=
        \left( \mathscr{L}_0 + \frac{i a \sin\theta}{\bar{\rho}^*} \right) \Phi_1 ~, \\
        \label{eq3:Dd0Phi1Ld1Phi2}
        \Delta \left( \mathscr{D}^\dagger_0 + \frac{1}{\bar{\rho}^*} \right) \Phi_1 &= 
        -\left( \mathscr{L}^\dagger_1 - \frac{i a \sin\theta}{\bar{\rho}^*} \right) \Phi_2 ~, \\
        \label{eq3:D0Phi1L1Phi0}
        \left( \mathscr{D}_0 + \frac{1}{\bar{\rho}^*} \right) \Phi_1 &= 
        \left( \mathscr{L}_1 - \frac{i a \sin\theta}{\bar{\rho}^*} \right) \Phi_0 ~, \\
        \label{eq3:Dd1Phi0Ld0Phi1}
        \Delta \left( \mathscr{D}^\dagger_1 - \frac{1}{\bar{\rho}^*} \right) \Phi_0 &= 
        -\left( \mathscr{L}^\dagger_0 + \frac{i a \sin\theta}{\bar{\rho}^*} \right) \Phi_1 ~.
    \end{align}
    \label{eq3:AllDPhiLPhi}
\end{subequations}
Now we may use commutatively property~\eref{eq3:propCommutLD} together with~\eref{eq3:propDeltaD} to separate the equations for $\Phi_0$ and $\Phi_2$.
In order to obtain the first equation, we must first apply the operator $(\mathscr{L}^\dagger_0 + i a \sin\theta/\bar{\rho}^*)$ to~\eqref{eq3:D0Phi1L1Phi0} and then use the commutativity relation to substitute~\eqref{eq3:Dd1Phi0Ld0Phi1}.
Similarly, applying $(\mathscr{L}_0 + i a \sin\theta/\bar{\rho}^*)$ to~\eqref{eq3:Dd0Phi1Ld1Phi2} we obtain the final equation.
Together yield
\begin{align}
    \label{eq3:DDLLPhi0}
    \left[ \Delta \mathscr{D}_1 \mathscr{D}^\dagger_1 + \mathscr{L}^\dagger_0 \mathscr{L}_1 + 2 i \omega (r+i a \cos\theta) \right] \Phi_0 = 0 ~, \\
    \label{eq3:DDLLPhi2}
    \left[ \Delta \mathscr{D}^\dagger_0 \mathscr{D}_0 + \mathscr{L}_0 \mathscr{L}^\dagger_1 - 2 i \omega (r+i a \cos\theta) \right] \Phi_2 = 0 ~.
\end{align}
Still, there is another way of combining equations, \emph{i.e} Eq.~\eref{eq3:Dd0Phi1Ld1Phi2} with~\eref{eq3:Dd1Phi0Ld0Phi1} and the remaining two form the set
\begin{align}
    \label{eq3:LLPhi0DDPhi2}
    \mathscr{L}_0 \mathscr{L}_1 \Phi_0 &= \mathscr{D}_0 \mathscr{D}_0 \Phi_2 ~,\\
    \label{eq3:LLPhi2DDPhi0}
    \mathscr{L}^\dagger_0 \mathscr{L}^\dagger_1 \Phi_2 &= \Delta \mathscr{D}^\dagger_0 \mathscr{D}^\dagger_0 \Delta \Phi_0 ~.
\end{align}
Thus, we went from four first-order differential equations relating three NP scalars to four second-order differential equations, two of each decoupled, eliminating the need for the scalar $\Phi_1$.
The last two equations imply that each one of the complex NP scalars contains all the information necessary to describe a EM wave (two polarizations).
One may think that we only need one of each group of equations to solve all perturbations, as of today no closed form solution has been found.
Thus the problem has to be tackled using approximations or numerical methods, recurring to all last four equations.

Due to the nature of the operators $\mathscr{D}_n$ and $\mathscr{L}_n$, we may separate the equations for $\Phi_0 \sim R_{+1}(r) S_{+1}(\theta)$ and $\Phi_2 \sim R_{-1}(r) S_{-1}(\theta)$ into two pairs of equations, 
\begin{subequations}
    \begin{align}
        \label{eq3:separationDDRp}
        \left(\Delta \mathscr{D}_0 \mathscr{D}^\dagger_0 + 2 i \omega r \right) \Delta R_{+1} 
        &= \lambdabar \Delta R_{+1} ~, \\
        \label{eq3:separationLLSp}
        \left( \mathscr{L}^\dagger_0 \mathscr{L}_1 - 2 a \omega \cos\theta \right) S_{+1}
        &= - \lambdabar S_{+1}  ~, 
    \end{align}
    \label{eq3:separationRSp}
\end{subequations}
and
\begin{subequations}
    \begin{align}
        \label{eq3:separationDDRm}
        \left( \Delta \mathscr{D}^\dagger_0 \mathscr{D}_0 - 2 i \omega r \right) R_{-1}
        &= \lambdabar R_{-1} ~, \\
        \label{eq3:separationLLSm}
        \left( \mathscr{L}_0 \mathscr{L}^\dagger_1 + 2 a \omega \cos\theta \right) S_{-1}
        &= - \lambdabar S_{-1}  ~,
    \end{align}
    \label{eq3:separationRSm}
\end{subequations}
where $\lambdabar$ is a separation constant.
We use the property~\eref{eq3:propDeltaD} in to obtain~\eqref{eq3:separationDDRp}.
The constant $\lambdabar$ must be real, as the angular differential operators $\mathscr{L}_n$ are also real.
Notice that we not distinguish the separation constants of both equations. Performing the transformation $\theta \rightarrow \pi-\theta$, the angular operators transforms as $\mathscr{L}^\dagger_0 \mathscr{L}_1 \rightarrow \mathscr{L}_0 \mathscr{L}^\dagger_1$. Then if $S_{+1}(\theta)$ is a solution for~\eqref{eq3:separationLLSp} for a given separation constant $\lambdabar$, this implies that $\tilde{S}_{-1}(\theta)=S_{+1}(\pi-\theta)$ is a solution for~\eqref{eq3:separationLLSm} for the same constant. 
In other words, the separation constant must me the same for both equations. 
Also, solutions $R_{-1}$ and $\Delta R_{+1}$ obey the same complex conjugate equations due to $\mathscr{D}^\dagger_n=(\mathscr{D}_n)^*$.

The second-order equations relating $\Phi_0$ and $\Phi_2$ can be separated in the same fashion. Naturally, the separation constant will differ from the eigenvalue Eqs.~\eref{eq3:separationRSp} and~\eref{eq3:separationRSm}.
Using the same substitutions made previously, we divide each equation by the corresponding ansatz to obtain
\begin{align}
    \label{eq3:separationB}
    \frac{\mathscr{L}_0 \mathscr{L}_1 S_{+1}}{S_{-1}} = \frac{\Delta \mathscr{D}_0 \mathscr{D}_0 R_{-1}}{\Delta R_{+1}} &= \mathscr{B} ~, \\
    \label{eq3:separationBdagger}
    \frac{\mathscr{L}^\dagger_0 \mathscr{L}^\dagger_1 S_{-1}}{S_{+1}} = \frac{\Delta \mathscr{D}^\dagger_0 \mathscr{D}^\dagger_0 \Delta R_{+1}}{R_{-1}} &= \mathscr{B} ~.
\end{align}
The separation constant $\mathscr{B}$ is real and equal for both equations. This claim rests on the same arguments of the eigenvalue $\lambdabar$. We also make the angular functions $S_{-1}$, $S_{+1}$ equally normalized. We may observe the latter by assuming two different separation constants $B_1$, $B_2$. Then, we have
\begin{align}
    \begin{split}
        (\mathscr{B}_1)^2 \int_0^\pi \dd\theta \sin\theta \, (S_{-1})^2 &=
        \int_0^\pi \dd\theta \sin\theta \, ( \mathscr{L}_0 \mathscr{L}_1  S_{+1} )^2 \\
        &= \int_0^\pi \dd\theta \sin\theta \, ( \mathscr{L}^\dagger_0 \mathscr{L}^\dagger_1 \mathscr{L}_0 \mathscr{L}_1  S_{+1} ) S_{+1} \\
        &=  \mathscr{B}_1 \mathscr{B}_2 \int_0^\pi \dd\theta \sin\theta \, (S_{+1})^2 ~,
    \end{split}
\end{align}
where we used integration by parts twice.
Thus $(\mathscr{B}_1)^2 = \mathscr{B}_1 \mathscr{B}_2 = \mathscr{B}^2$.
We can compute the coefficient by computing the operation
\begin{align}
    \begin{split}
        \mathscr{L}^\dagger_0 \mathscr{L}^\dagger_1 \mathscr{L}_0 \mathscr{L}_1 &=
        \mathscr{L}^\dagger_0 ( \mathscr{L}_0 \mathscr{L}^\dagger_1 - 4 a\omega \cos\theta )\mathscr{L}_1 \\
        &= \mathscr{L}_0 \mathscr{L}^\dagger_1 ( - \lambdabar + 2 a\omega \cos\theta) - 4 a\omega \cos\theta \,\mathscr{L}^\dagger_0 \mathscr{L}_1 + 4 a\omega \sin\theta \,\mathscr{L}_1 \\
        &= -\lambdabar \, \mathscr{L}_0 \mathscr{L}^\dagger_1 + 2 a\omega \left[ \cos\theta \, \mathscr{L}_0 \mathscr{L}^\dagger_1 -\sin\theta(\mathscr{L}_1+\mathscr{L}^\dagger_1) \right] \\ &\mkern200mu - 4 a\omega \cos\theta \,\mathscr{L}^\dagger_0 \mathscr{L}_1 + 4 a\omega \sin\theta \,\mathscr{L}_1 \\
        &= (-\lambdabar + 2 a\omega \cos\theta ) \mathscr{L}_0 \mathscr{L}^\dagger_1 + 4 a\omega \mathscr{Q} \sin\theta \\
        &= (-\lambdabar + 2 a\omega \cos\theta ) (-\lambdabar - 2 a\omega \cos\theta ) + 4 a\omega ( -a\omega \sin^2\theta + m ) \\
        &= \lambdabar^2 - 4 a^2 \omega^2 + 4 a \omega m = \mathscr{B}^2 ~,
    \end{split}
\end{align}
applied on the angular function $S_{+1}$.
The commutation relations between the angular operators can by found directly or by noticing that $[e_a,e_b]=(\gamma^c{}_{ba}-\gamma^c{}_{ab}) e_c$.
Then using~\eqref{eq3:separationLLSp} it is possible to eliminate the second-order angular operators.
To obtain $\mathscr{B}^2$, the same procedure could be done for $S_{-1}$ or the the radial functions.

\subsection{Mode decomposition}

Will be more profitable to study Eqs.~\eref{eq3:DDLLPhi0} and~\eref{eq3:DDLLPhi2} as a special case of the Teukolsky master equation which describes all the linearized perturbations arround the Kerr geometry.
The generality of this equation is the primary reason for the focus on the EM case.
The treatment for GWs differs in the perturbation formalism only in algebraic complexity, resulting in the same master equation.
With the Teukolsky master equation we can proceed considering general perturbations, but there are several numerical and analytical details that make EM waves and GWs differ later on.

The general equation reads
\begin{align}
    \begin{split}
        & \frac{1}{\Delta^s} \frac{\partial}{\partial r} \left( \Delta^{s+1} \frac{\partial \Upsilon_s}{\partial r} \right) 
        + \frac{1}{\sin\theta} \frac{\partial}{\partial\theta} \left( \sin\theta \frac{\partial \Upsilon_s}{\partial \theta} \right) 
        - \left[ \frac{(r^2+a^2)^2}{\Delta} - a^2 \sin^2\theta \right]\frac{\partial^2 \Upsilon_s}{\partial t^2} \\[0.15cm]
        - & \frac{4 M a r}{\Delta}\frac{\partial^2 \Upsilon_s}{\partial t \partial \varphi} 
        - \left( \frac{a^2}{\Delta} -\frac{1}{\sin^2\theta} \right)\frac{\partial^2 \Upsilon_s}{\partial \varphi^2} 
        + 2s\left[ \frac{M(r^2-a^2)}{\Delta} - r - i a \cos\theta \right] \frac{\partial \Upsilon_s}{\partial t} \\[0.15cm]
        + & 2s\left[ \frac{a(r-M)}{\Delta}+\frac{i \cos\theta}{\sin^2\theta}\right] \frac{\partial \Upsilon_s}{\partial \varphi}
        - (s^2 \cot^2\theta - s) \Upsilon_s = 0 ~,
    \end{split}
    \label{eq3:teukolsky}
\end{align}
where $s$ is the field \emph{spin weight} and each field quantity $\Upsilon_s$ is related to the NP scalars as shown in the~\tref{tb3:solutionsTeukolskyEq}.
Depending on the spin weight, the equation may describe massless scalar ($s=0$) or Dirac fields ($s=\pm \tfrac{1}{2}$), as well as electromagnetic ($s=\pm 1$) or gravitational waves ($s=\pm 2$). Substituting the spin-weight for the EM waves we obtain Eqs.~\eref{eq3:DDLLPhi0} and~\eref{eq3:DDLLPhi2}.
\begin{table}[h]
    \centering
    \renewcommand{\arraystretch}{1.25}
    \begin{tabular}{ | c | l | }
        \hline
        $s$ & $\qquad ~ \Upsilon_s$ \\
        \hline\hline
        $+1$ & $\Phi_0 = \phi_0$ \\
        \hline
        $-1$ & $\Phi_2 = 2 (\bar{\rho}^*)^2 \phi_2$ \\
        \hline
        $+2$ & $\Psi_0 = \psi_0$  \\
        \hline
        $-2$ & $\Psi_4 = (\bar{\rho}^*)^4 \psi_4$ \\
        \hline
    \end{tabular}
    \caption{Performance at peak F-measure}
    \label{tb3:solutionsTeukolskyEq}
\end{table}

Obviously, Teukolsky equation is explicitly independent of $t$ and $\varphi$, thus $\Upsilon_s$ accepts a decomposition in $e^{-i \omega t + i m \varphi}$, which we already assumed in the EM case to separate the equations.
Stationary and axisymmetry of the spacetime geometry guarantees this form.
The azimuthal wave number $m$ must be an integer, due to periodic boundary conditions on the BL coordinate $\varphi$.
We may separate all perturbations in a completely general mode decomposition
\begin{align}
    \Upsilon_s = \int\dd\omega \, e^{-i \omega t} \left( \,\sum_{\ell,m} e^{i m \varphi} \, {}_{s}S_{\ell m}(\theta) {}_{s}R_{\ell m}(r) \right) ~.
\end{align}
The integer $\ell$ plays a role in labelling all possible solutions for the eigenvalue problem of both radial and angular equations,
\begin{align}
    \label{eq3:teukolskyRadial}
    \frac{1}{\Delta^s} \frac{\dd}{\dd r} \left( \Delta^{s+1} \frac{\dd\, {}_{s}R_{\ell m}}{\dd r} \right)
    + \left[ \frac{\mathscr{K}^2 - 2 i s (r-M)\mathscr{K}}{\Delta} + 4 i s \omega r -{}_{s}\mathscr{F}_{\ell m}  \right] {}_{s}R_{\ell m} = 0 ~, \\[0.15cm]
    \label{eq3:teukolskyAngular}
    \frac{1}{\sin\theta} \frac{\dd}{\dd\theta} \left( \sin\theta \frac{\dd\, {}_{s}S_{\ell m}}{\dd \theta} \right)
    + \left[ a^2 \omega^2 \cos^2\theta - 2 s a \omega \cos\theta - \frac{(m + s \cos\theta)^2}{\sin^2\theta} + s + {}_{s}\mathscr{A}_{\ell m} \right] {}_{s}S_{\ell m}  = 0~.
\end{align}
The radial and angular eigenvalues are related to the separation constant on Eqs.~\eref{eq3:separationRSp} and~\eref{eq3:separationRSm} through
\begin{align}
    {}_{s}\mathscr{F}_{\ell m} = {}_{s}\mathscr{A}_{\ell m} - 2 m a \omega + a^2 \omega^2 \stackrel[(s=\pm 1)]{}{=} \lambdabar - s(s+1) ~.
    \label{eq3:relationEVs}
\end{align}
Due to the form of the angular equation, the eigenvalues ${}_{s}\mathscr{C}_{\ell m}$, ${}_{s}\mathscr{A}_{\ell m}$ as well as the function ${}_{s}S_{\ell m}(\theta)$ depends also on the coupling $a \omega$.
Clearly, the same does not hold for the radial function ${}_{s}R_{\ell m}(r)$. 

\section{Spin-Weighted Spheroidal Harmonics}

To shed some light into the explicit form of $\lambdabar$, we will need to dive into the eigenvalue problem for the angular equation. We may transform the~\eqref{eq3:teukolskyAngular} into a more familiar form using the change of coordinate $z=\cos\theta$ and renaming the BH-wave coupling $c=a \omega$, obtaining
\begin{align}
    \frac{\dd}{\dd z} \left[ (1-z^2) \frac{\dd\, {}_{s}S_{\ell m}}{\dd z} \right] + \left[ (c z)^2 - 2 c s z  -\frac{(m + s z)^2}{1 - z^2} + s  + {}_{s}\mathscr{A}_{\ell m} \right] {}_{s}S_{\ell m} = 0 ~.
    \label{eq3:swshEq}
\end{align}
We may also use freely ${}_{s}S_{\ell m}(\cos\theta) \equiv {}_{s}S_{\ell m}(\theta)$.
We will consider $c$ real as we are analyzing superradiance of EM waves in the vacuum, although we could generalize the spin-weighted spheroidal harmonic (SWSH) equation to imaginary $c$ values to describe waves in a particular medium.

Spherically symmetric problems allow for a decomposition using spherical harmonics $Y_{\ell m}(\theta,\varphi)$ of angular dependent functions with finite boundary conditions.
These have innumerable applications in physics such as the hydrogen atom or the description of anisotropies in the cosmic microwave background.
By setting $s=0$ and $c=0$ (spherical), then it is clear that solution for~\eqref{eq3:swshEq} are given by the associated Legendre polynomials, $P^m_\ell(z)$, and the therefore it is a generalization of the spherical harmonics, \emph{i.e.}
\begin{align}
    e^{i m \varphi} \,{}_{0}S_{\ell m}(\theta) \stackrel[(c=0)]{}{=} Y_{\ell m}(\theta,\varphi) ~,\qquad {}_{0}\mathscr{A}_{\ell m} \stackrel[(c=0)]{}{=} \ell (\ell + 1) ~.
\end{align}
The value of $\ell$ are non-negative integers, with the restriction of $\ell \ge |m|$. 
Since this generalization must hold for the spherical case, we require that spin-weighted spheroidal harmonics (SWSHs) are normalized for any spin $s$ and coupling $c$,
\begin{align}
    \int_{0}^\pi \dd\theta \sin\theta \, | \,{}_{s}S_{\ell m}(\theta) |^2 =
    \int_{-1}^{1} \dd z  \, | \,{}_{s}S_{\ell m}(z) |^2 = \frac{1}{2\pi} ~.
\end{align}

Still on subject of spherical harmonics ($c=0$), perturbations of any type in Schwarzschild spacetime are described using spin-weighted \emph{spherical} harmonics, ${}_{s}Y_{\ell m}(\theta,\varphi)$.
Due to the shared symmetries with spherical harmonics it is possible to find a closed form for $s\ne0$ harmonics, which is a well studied problem.
We can raise and lower spin-weight with the use of the operators commonly denoted as $\bar{\eth}$ and $\eth$, respectively, applied on $Y_{\ell m}$,
\begin{align}
    \begin{split}
        (\bar{\eth})^s Y_{\ell m} &= \sqrt{\frac{(\ell+s)!}{(\ell-s)!}} \,{}_{s}Y_{\ell m}  ~, \\
        (-1)^s  (\eth)^s Y_{\ell m} &= \sqrt{\frac{(\ell+s)!}{(\ell-s)!}} \,{}_{-s}Y_{\ell m} ~,
    \end{split}
\end{align}
limited by $|s|\le\ell$.
Eqs.~\eref{eq3:separationB} and~\eref{eq3:separationBdagger} are closely related to former, as applications of operators $\mathscr{L}^\dagger_s$ and $\mathscr{L}_s$, generalize $\bar{\eth}$ and $\eth$, respectively. 

Thus, for the non-spherical symmetry we could in principle obtain all SWSHs if we knew the closed form for ${}_{0}S_{\ell m}$ and its eigenvalues. The problem lies in the fact that no such decomposition of elementary function was found. This require us to follow numerical and approximate methods to find the values of ${}_{s}S_{\ell m}$.

The major advances on the study of the eigenvalues of the SWSHs was performed by Leaver in 1985. Working out the asymptotical and critical behavior of the equation, we observe that equation the equation diverges at the poles, $z = \pm 1$, where the it takes the form $(1 \mp z) \,{}_{s}S_{\ell m}{}'(z) \sim \mp \tfrac{1}{4} (m \pm s)^2 \, {}_{s}S_{\ell m}(z)$. In order to guarantee that SWSH is everywhere analytical, Leaver proposed the series expansion at $x=-1$,
\begin{align}
    {}_{s}S_{\ell m}(z) = e^{c z} (1+z)^{k_-} (1-z)^{k_+} \sum_{p=0}^\infty a_p (1+z)^p ~,
\end{align}
where $k_{\pm} = \tfrac{1}{2}|m \pm s|$.
The exponential in the ansatz accounts for the large $z$ behavior of the equation.
Substituting in the angular equation, we obtain a three-term recurrence relation between the expansion coefficients $a_p$ and the boundary condition at $x=-1$,
\begin{align}
    \label{eq3:ap3CoefRecursion}
    \alpha_p a_{p+1} + \beta_p a_p + \gamma_p a_{p-1} = 0 ~,\qquad
    \alpha_0 a_1 + \beta_0 a_0 = 0 ~,
\end{align}
where
\begin{align}
    \begin{split}
        \alpha_p &= -2 (1 + p) (1 + 2 k_{+} + p) ~,\\
        \beta_p  &= (k_{-} + k_{+} + p - s) (1 + k_{-} + k_{+} + p + s) \\
        &\qquad\qquad\qquad - 2 c (1 + 2 k_{-} + 2 p + s) - c^2 - {}_{s}\mathscr{A}_{\ell m} ~,\\
        \gamma_p &= 2 c (k_{-} + k_{+} + p + s) ~.
    \end{split}
\end{align}
We then find an equation for the eigenvalue ${}_{s}\mathscr{A}_{\ell m}$ with explicit dependence on $m$, $s$ and $c$, by combining the previous relations into a continued fraction,
\begin{align}
    \beta_0 = \frac{\alpha_0 \gamma_1}{\beta_1 -} \frac{\alpha_1 \gamma_2}{\beta_2 -} \frac{\alpha_2 \gamma_3}{\beta_3 -} \cdots \left( \equiv  \frac{\alpha_0 \gamma_1}{\beta_1 - \frac{\alpha_1 \gamma_2}{\beta_2 - \frac{\alpha_2 \gamma_3}{\beta_3 - \dots}}} \right)  ~.
    \label{eq3:evInvertion0th}
\end{align}
We can also consider the $r$-th inversion of~\eqref{eq3:evInvertion0th},
\begin{align}
    \beta_r - \frac{\alpha_{r-1} \gamma_r}{\beta_{r-1} -} \frac{\alpha_{r-2} \gamma_{r-1}}{\beta_{r-2} -} \cdots \frac{\alpha_{1} \gamma_2}{\beta_{1} -} \frac{\alpha_{0} \gamma_{1}}{\beta_{0}} = \frac{\alpha_{r} \gamma_{r+1}}{\beta_{r+1} -} \frac{\alpha_{r+1} \gamma_{r+2}}{\beta_{r+2} -} \cdots ~.
    \label{eq3:evInvertionRth}
\end{align}
These equations involve an infinite fraction that depends explicitly on $c$, $m$, $s$, which leads to suspicion that this leads to a infinite spectra.
This is a close parallel to the spherical case, where we have a infinite number of harmonics, although no proof has been found.
If we notice that $\gamma_p\propto c$, then the zero order expansion of the eigenvalue expansion on for $c\ll 1$ leads to the equation $\beta_r=0$.
In the spherical geometry, this corresponds to truncating the series at $r$, since $\gamma_p=0$ for any $p$. Since the the eigenvalue root depends on integer $r$, 
\begin{align}
    {}_{s}\mathscr{A}_{\ell m} = \ell(\ell+1) - s(s+1) + \mathscr{O}(a\omega)~,
    \label{eq3:evSWSH0th}
\end{align}
implying the discretization of the spectra, where we identified $\ell=r+k_{+}+k_{-}$. 
Since $r\ge 0$, then we must have $\ell\ge\mathrm{max}\{|m|,|s|\}$, \emph{i.e.} the leading contribution for the monopole expansion is the dipole for EM waves and the quadrupole for GWs.
Changing $r$ for $\ell$ corresponds simply to a relabeling of the eigenfunctions in order to match the values of the spectra when $c=0$ and also when $s=0$.

(MORE ON PERTURBATIONS $a\omega\sim 1$)

Will be useful expanding ${}_{s}\mathscr{A}_{\ell m}$ in high order terms in order to obtain the series coefficients for the eigenvalue (\aref{AppendixEigenvalues}).
Up to sixth order, the only the zero-order term depends on the sign of the spin weight as it occurs in~\eqref{eq3:relationEVs}.
In the general angular equation, inversion of spin corresponds to inversion of poles, \emph{i.e.} stays invariant under the transformation $(s,z)\to(-s,-z)$.
Under this transformation, the eigenvalue must obey $s+ {}_{s}\mathscr{A}_{\ell m} = -s + {}_{-s}\mathscr{A}_{\ell m}$, thus it is beneficial to define 
\begin{align}
    {}_{s}\mathscr{E}_{\ell m} = {}_{s}\mathscr{A}_{\ell m} - s(s+1) ~,
    \label{eq3:sElm}
\end{align}
to also exploit the symmetry of spin inversion in numerical computations.

\section{Analytic radial expansions}



\subsection{Near horizon approximation}
\subsection{Assymptotic expansion}

\section{Amplification factor $Z_{slm}$}


\cleardoublepage