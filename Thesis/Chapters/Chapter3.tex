% !TEX root = ../main.tex

\chapter{Teukolsky's master equation} % Main chapter title
\label{Chapter3}

%----------------------------------------------------------------------------------------

\section{Newman-Penrose formalism}

The study of gravitational and electromagnetic perturbations in a BH background was performed long before Kerr found his solution, for other spacetimes such as Schwarzschild's.
Despite its simplicity, the procedure involved was already algebraically tedious.
In the Kerr case, the metric was far more complicated, making the problem almost untreatable.

Fortunately, the NP formalism \cite{Newman1962} provides an alternative method of studying perturbations.
This formalism results from a natural introduction of spinor techniques in GR, after the choice of a null complex tetrad basis,
\begin{align}
    \bm{e}_a = (e_a)^\mu \frac{\partial}{\partial x^\mu} \qquad (a = 1, 2, 3, 4) ~,
\end{align}
where all quantities will be projected, \emph{i.e.} for the Weyl tensor we define
\begin{align}
    C_{abcd} =  (e_a)^\alpha \,(e_b)^\beta \,(e_c)^\gamma \,(e_d)^\delta \,C_{\alpha\beta\gamma\delta} ~.
\end{align}
Penrose believed that the light-cone was the essential element of the spacetime, thus it was of importance to find null directions.
The basis consisted in two real vectors, $\bm{\mathfrak{l}}$ and $\bm{\mathfrak{n}}$, and two complex conjugate vectors $\bm{\mathfrak{m}}$ and $\bar{\bm{\mathfrak{m}}}$.
Besides satisfying
\begin{align}
    \bm{\mathfrak{l}}^2 = \bm{\mathfrak{n}}^2 = \bm{\mathfrak{m}}^2 = \bar{\bm{\mathfrak{m}}}^2 = 0 ~,
\end{align}
orthogonality conditions of NP formalism require
\begin{align}
   \bm{\mathfrak{l}} \cdot \bm{\mathfrak{m}} = \bm{\mathfrak{l}} \cdot \bar{\bm{\mathfrak{m}}} = \bm{\mathfrak{n}} \cdot \bm{\mathfrak{m}} = \bm{\mathfrak{n}} \cdot \bar{\bm{\mathfrak{m}}} = 0 ~.
\end{align}
Still we are left with the ambiguity raised by multiplication of scalar functions to each vector, therefore it is customary to impose normalization conditions to the basis,
\begin{align}
    \bm{\mathfrak{l}} \cdot \bm{\mathfrak{n}} = 1 ~, \qquad \bm{\mathfrak{m}} \cdot \bar{\bm{\mathfrak{m}}} = -1 ~.
\end{align}
This formalism is a special case of tetrad calculus, where we can identify the new basis as $(\bm{\mathfrak{l}},\bm{\mathfrak{n}},\bm{\mathfrak{m}},\bar{\bm{\mathfrak{m}}})$. The ``metric'' for manipulating tetrad indices, $\eta_{ab}$, is defined by all restrictions provided above,
\begin{align}
    g^{\mu\nu} = \eta^{ab} (e_a)^\mu (e_b)^\nu = \mathfrak{l}^{\mu} \mathfrak{n}^{\nu} + \mathfrak{n}^{\mu} \mathfrak{l}^{\nu} - \mathfrak{m}^{\mu} \bar{\mathfrak{m}}^{\nu} - \bar{\mathfrak{m}}^{\mu} \mathfrak{m}^{\nu} ~.
\end{align}
Additionally these vectors define new directional derivatives.
We will depart shortly from standard notation, by redefining these derivatives as
\begin{align}
    \mathbbl{D}=\nabla_{\bm{\mathfrak{l}}} ~,\qquad \mathbbl{\Delta}=\nabla_{\bm{\mathfrak{n}}} ~,\qquad \bbdelta=\nabla_{\bm{\mathfrak{m}}} ~,\qquad \bar{\bbdelta}=\nabla_{\bar{\bm{\mathfrak{m}}}} ~.
    \label{eq3:tetradCovDer}
\end{align}

More details and definitions on the tetrad formalism can be found in~\aref{AppendixTetradFormalism}.

%----------------------------------------------------------------------------------------

\subsection{Kinnersley tetrad}

The Riemann tensor may have up to twenty non-vanishing components.
We know that ten of these are present in the symmetric Ricci tensor, that is intrinsically connected to matter and energy.
The other components are pure gravitational degrees of freedom and are encoded in the Weyl tensor.
It becomes the most useful object when the Ricci tensor vanishes, such as for vacuum solutions and source-free gravitational waves.
In order to remove the Ricci tensor degrees of freedom, the tensor must be constructed as trace-free,
\begin{align}
    \eta^{ad} C_{abcd} = C_{1bc2} + C_{1bc2} - C_{3bc4} - C_{4bc3} = 0 ~. 
\end{align}
Together with the other symmetries inherited from the Riemann tensor, for instance the first Bianchi identity, $C_{a[bcd]}=0$, it is possible to show that some of these NP components vanish while others remain related, leaving us with ten degrees of freedom.
As a result, in the NP formalism the Weyl tensor can be represented by five complex scalars, usually chosen as
\begin{align}
    \begin{split}
        \psi_0 &= - C_{1313} = - C_{\alpha\beta\gamma\delta}\, \mathfrak{l}^\alpha \mathfrak{m}^\beta \mathfrak{l}^\gamma \mathfrak{m}^\delta ~,\qquad
        ~\psi_1 = - C_{1213} = - C_{\alpha\beta\gamma\delta}\, \mathfrak{l}^\alpha \mathfrak{n}^\beta \mathfrak{l}^\gamma \mathfrak{m}^\delta ~,\\
        \psi_2 &= - C_{1342} = - C_{\alpha\beta\gamma\delta}\, \mathfrak{l}^\alpha \mathfrak{m}^\beta \bar{\mathfrak{m}}^\gamma \mathfrak{n}^\delta ~,\qquad
        \psi_3 = - C_{1242} = - C_{\alpha\beta\gamma\delta}\, \mathfrak{l}^\alpha \mathfrak{n}^\beta \bar{\mathfrak{m}}^\gamma \mathfrak{n}^\delta ~,\\
        \psi_4 &= - C_{2424} = - C_{\alpha\beta\gamma\delta}\, \mathfrak{n}^\alpha \bar{\mathfrak{m}}^\beta \mathfrak{n}^\gamma \bar{\mathfrak{m}}^\delta ~.
    \end{split}
\end{align}
The complex conjugates can be obtained by doing the replacement $3 \rightleftarrows 4$, by exchanging $\bm{\mathfrak{m}}$ with $\bar{\bm{\mathfrak{m}}}$ and vice-versa. 
The Weyl tensor has a unique decomposition in terms of a linear combination of NP scalars and tensorial product of two-forms.
This decomposition has the general form,
\begin{align}
    \begin{split}
    \frac{1}{4} \, C_{\mu\nu\rho\sigma} = & - \psi_0 \,V_{\mu\nu} V_{\rho\sigma} - \psi_1 \, (V_{\mu\nu} W_{\rho\sigma} + W_{\mu\nu} V_{\rho\sigma}) \\
    & - \psi_2 \, (U_{\mu\nu}V_{\rho\sigma} + V_{\mu\nu}U_{\rho\sigma} + W_{\mu\nu}W_{\rho\sigma}) \\
    & - \psi_3 \, (U_{\mu\nu}W_{\rho\sigma} + W_{\mu\nu}U_{\rho\sigma})
    - \psi_4 \, U_{\mu\nu}U_{\rho\sigma} + \text{ c.c.}
    \end{split}
\end{align}
where $U_{\mu\nu} = \mathfrak{l}_{[\mu} \mathfrak{m}_{\nu]}$, $V_{\mu\nu} = \bar{\mathfrak{m}}_{[\mu} \mathfrak{n}_{\nu]}$, $W_{\mu\nu} = \mathfrak{l}_{[\mu} \mathfrak{n}_{\nu]} - \mathfrak{m}_{[\mu} \bar{\mathfrak{m}}_{\nu]}$.
It is clear that the values that these five complex scalars take is completely dependent on the choice of tetrad frame. 

BH solutions are ``type D'' spacetimes according to Petrov's classification, which was a major restriction necessary to the discovery of Kerr's metric.
For these spacetimes it is possible to find two different doubly-degenerate principal directions of the Weyl tensor, which we choose to be the real vectors of the tetrad, $\bm{\mathfrak{l}}$ and $\bm{\mathfrak{n}}$ \cite{Chandrasekhar1998}.
These yield
\begin{align}
    C_{\mu\alpha\beta[\nu} \mathfrak{l}_{\rho]} \mathfrak{l}^\alpha \mathfrak{l}^\beta = 0 ~, \qquad C_{\mu\alpha\beta[\nu} \mathfrak{n}_{\rho]} \mathfrak{n}^\alpha \mathfrak{n}^\beta = 0  ~.
\end{align}
In NP formalism terms, this implies, respectively,
\begin{align}
    \psi_0=\psi_1=0 ~,\qquad \psi_3=\psi_4=0 ~.
\end{align}

Finding the principal directions may not be trivial, but we can apply successive local transformations of the six-parameter Lorentz group in order to rotate the tetrad vectors. 
This procedure allows for the simplification of the Weyl tensor by vanishing NP scalars, ``locking'' the orientation of the tetrad frame.
The Weyl scalar $\psi_2$ becomes invariant under boosts in the principal directions, usually refereed as ``type III'' rotations \cite{Chandrasekhar1998}.
These keep the light-cone structure intact by maintaining the direction of $\bm{\mathfrak{l}}$ and $\bm{\mathfrak{n}}$ unchanged (up to multiplication of scalar functions), being useful to change between ingoing and outgoing frames \cite{Teukolsky1974}. 
Kinnersly solved the type D vacuum field equations \cite{Kinnersley1969}, finding a suitable tetrad 
\begin{align}
    \label{eq3:kinnerslytetrad}
    \begin{split}
        \bm{\mathfrak{l}} =& \left(\frac{r^2+a^2}{\Delta}, \, 1, \,0, \,\frac{a}{\Delta} \right) ~, \\
        \bm{\mathfrak{n}} =& \frac{1}{2 \rho^2} \Bigr( r^2+a^2, \,-\Delta, \,0 , \,a \Bigr) ~, \\
        \bm{\mathfrak{m}} =& \frac{1}{ \sqrt{2} \bar{\rho}^2 } \Bigr( i a \sin\theta, \,0, \,1, \, i \csc\theta \Bigr) ~,
    \end{split}
\end{align}
where $\bar{\rho} = r + i a \cos\theta$ and $\rho^2 \equiv | \bar{\rho} |^2 = \bar{\rho} \bar{\rho}^*$.

The NP formalism provides a full set of first-order coupled differential equations, relating the NP scalars components of the Weyl and Maxwell tensors. 
These equations result from the second Bianchi identity, $C_{\mu\nu[\rho \sigma ; \lambda ]} = 0$, and the Maxwell equations.
In order to write these equations explicitly we need to define the \emph{spin coefficients} using the connection $\gamma_{abc} = (e_{a})^\mu (e_{b})_{\mu;\nu} (e_{c})^\nu$, which replaces the Christopher symbols in this formalism.

To study GWs, instead of perturbing the background metric, the NP formalism provides a natural way of performing perturbations by modification of the tetrad, $\bm{\mathfrak{l}}=\bm{\mathfrak{l}}^B+\bm{\mathfrak{l}}^P$, $\bm{\mathfrak{n}}=\bm{\mathfrak{n}}^B+\bm{\mathfrak{n}}^P$, etc., and also the NP scalars, $\psi_a = \psi_a{}^B + \psi_a{}^P$, maintaining only first-order terms \cite{Teukolsky1973a}.
The formalism reveals decoupled equations for $\psi_0{}^P$ and $\psi_4{}^P$, which implies that these dynamic variables are the only independent degrees of freedom of the GWs.

%----------------------------------------------------------------------------------------

\subsection{Maxwell equations}

We focus with more detail on EM perturbations with a fixed background because they involve a simpler procedure and then we will tie with the same master equation that also describes GW perturbations. 
 
In the NP formalism, all Maxwell equations, $F_{[\mu\nu; \rho]}=0$ and \eqref{eq2:maxwellEM}, reduce to
\begin{align}
    \label{eq3:maxwellFabEqs}
    F_{[ab \rvert c]} = 0 ~,\qquad \eta^{bc} F_{ab \rvert c} = 0 ~
\end{align}
(see \aref{AppendixTetradFormalism}).
The Maxwell tensor $F_{\mu\nu}$ has a total of six components which encodes the vector quantities of the electric and the magnetic fields. We may reduce the equation using three complex NP scalars,
\begin{align}
    \label{eq3:maxwellNPphi}
    \begin{split}
        \phi_0 &= F_{13} = F_{\alpha\beta} \,\mathfrak{l}^\alpha \mathfrak{m}^\beta ~,\qquad
        \phi_1 = \tfrac{1}{2} (F_{12} + F_{43}) = \tfrac{1}{2} F_{\alpha\beta} \,(\mathfrak{l}^\alpha \mathfrak{n}^\beta + \bar{\mathfrak{m}}^\alpha \mathfrak{m}^\beta) ~,\\
        \phi_2 &= F_{42} = F_{\alpha\beta} \,\bar{\mathfrak{m}}^\alpha \mathfrak{n}^\beta ~.
    \end{split}
\end{align}
%$\tfrac{1}{2} F_{\mu\nu} = - \phi_0 \,V_{\mu\nu} - \phi_1 \,W_{\mu\nu} - \phi_2 \,U_{\mu\nu}+ \text{ c.c.}$
Considering all possible combinations of NP indices in \eref{eq3:maxwellFabEqs}, we gather eight equations, double the amount of necessary relations. 
This occurs because the conjugates $\phi_0^*$, $\phi_1^*$, $\phi_2^*$ are coupled in these equations. Eliminating every term of the form $F_{23\rvert a}$ or $F_{14\rvert b}$, 
\begin{subequations}
    \begin{align}
        \phi_{2\rvert 1} &= \phi_{1\rvert 4} ~, \label{eq3:phi21phi14}\\
        \phi_{1\rvert 2} &= \phi_{2\rvert 3} ~, \label{eq3:phi12phi23}\\
        \phi_{1\rvert 1} &= \phi_{0\rvert 4} ~, \label{eq3:phi11phi04}\\
        \phi_{0\rvert 2} &= \phi_{1\rvert 3} ~. \label{eq3:phi02phi13}
    \end{align}
\end{subequations}
We may expand explicitly the left-hand side of \eqref{eq3:phi21phi14},
\begin{align}
    \label{eq3:phi21SpinCoef}
    \begin{split}
        \phi_{2\rvert 1} &= \phi_{2, 1} - \eta^{ab}( \gamma_{a41} F_{b2} + \gamma_{a21} F_{4b} ) \\
        &= \phi_{2, 1} - (\gamma_{241} F_{12} + \gamma_{121} F_{42}) + ( \gamma_{341} F_{42} + \gamma_{421} F_{43} ) \\
        &= \phi_{2, 1} + 2 F_{42} \left(\frac{\gamma_{341} + \gamma_{211}}{2}\right) + 2 \gamma_{421} \left(\frac{F_{12} + F_{43}}{2}\right) \\
        &= \mathbbl{D} \phi_2 + 2 \varepsilon \phi_2 - 2 \pi \phi_1 ~,
    \end{split}
\end{align}
where we used the antisymmetry of the spin connection, $\gamma_{abc}=-\gamma_{bac}$. The right-hand side yields
\begin{align}
    \label{eq3:phi14SpinCoef}
    \begin{split}
        \phi_{1\rvert 4} &= \phi_{1, 4} - \tfrac{1}{2} \eta^{ab}( \gamma_{a14} F_{b2} + \gamma_{a24} F_{1b} + \gamma_{a44} F_{b3} + \gamma_{a34} F_{4b} ) \\
        &= \phi_{1, 4} - \tfrac{1}{2} ( \gamma_{144} F_{23} + \gamma_{134} F_{42} + \gamma_{214} F_{12} + \gamma_{234} F_{41} ) \\ 
        &\qquad\qquad + \tfrac{1}{2} ( \gamma_{314} F_{42} + \gamma_{414} F_{42} + \gamma_{324} F_{14} + \gamma_{424} F_{13} ) \\
        &= \phi_{2, 1} - \gamma_{244} F_{13} + \gamma_{314} F_{42} \\
        &= \bar{\bbdelta} \phi_1 - \lambda \phi_0 + \tau \phi_2  ~.
    \end{split}
\end{align}
The spin coefficients $\varepsilon$, $\pi$, $\lambda$, $\tau$, along with other NP definitions are found in~\aref{AppendixNPSpinCoef}. If we repeat the same expansion for the other Maxwell equations, we gather the set
\begin{subequations}
    \begin{align}
        \label{eq3:phi21phi14SpinCoef}
        \mathbbl{D} \phi_2 - \bar{\bbdelta} \phi_1 &= -\lambda \phi_0 + 2 \pi \phi_1 + (\varrho - 2 \varepsilon) \phi_2 ~,\\
        \label{eq3:phi12phi23SpinCoef}
        \mathbbl{\Delta} \phi_1 - \bbdelta \phi_2 &= \nu \phi_0 - 2 \mu \phi_1 + (2\beta-\tau) \phi_2 ~,\\
        \label{eq3:phi11phi04SpinCoef}
        \mathbbl{D} \phi_1 - \bar{\bbdelta} \phi_0 &= (\pi - 2 \alpha) \phi_0 + 2 \varrho \phi_1 - \kappa \phi_2 ~,\\
        \label{eq3:phi02phi13SpinCoef}
        \mathbbl{\Delta} \phi_0 - \bbdelta \phi_1 &= (2 \gamma - \mu) \phi_0 - 2 \tau \phi_1 + \sigma \phi_2 ~.
    \end{align}
\end{subequations}
The Kinnersley tetrad guarantees that $\kappa = \sigma = \lambda = \nu = 0$, decoupling all equations above. After substitution of all spin coefficients,
\begin{subequations}
    \begin{align}
        \label{eq3:phi21phi14Expand}
        \left( \mathbbl{D} + \frac{1}{\bar{\rho}^*} \right) \phi_2 &= 
        \left( \bar{\bbdelta} + \frac{2 i a \sin\theta}{\sqrt{2} (\bar{\rho}^*)^2} \right) \phi_2 ~,\\
        \label{eq3:phi12phi23Expand}
        \left( \mathbbl{\Delta} - \frac{\Delta}{\rho^2 \bar{\rho}^*} \right) \phi_1 &= 
        \left[ \bbdelta + \frac{1}{\sqrt{2} \bar{\rho}} \left( \cot\theta - \frac{i a \sin\theta}{\bar{\rho}^*} \right) \right] \phi_2 ~,\\
        \label{eq3:phi11phi04Expand}
        \left( \mathbbl{D} + \frac{2}{\bar{\rho}^*} \right) \phi_1 &=
        \left[ \bar{\bbdelta} + \frac{1}{\sqrt{2} \bar{\rho}^*}\left( \cot\theta - \frac{i a \sin\theta}{\bar{\rho}^*} \right) \right] \phi_0 ~, \\
        \label{eq3:phi02phi13Expand}
        \left[ \mathbbl{\Delta} + \frac{\Delta}{2 \rho^2} \left( \frac{1}{\bar{\rho}^*} - \frac{2(r-M)}{\Delta} \right) \right] \phi_0 &=
        \left(\bbdelta + \frac{2 i a \sin\theta}{\sqrt{2} \bar{\rho} \bar{\rho}^*}\right) \phi_1  ~.
    \end{align}
    \label{eq3:phiAllExpand}
\end{subequations}

An important consequence of the symmetries of the Kerr spacetime allows for a wave decomposition of the form $\phi_0, \phi_1, \phi_2 \sim e^{- i \omega t + i m \varphi}$.
Therefore, the four differential operators group into radial $(\mathbbl{D}, \mathbbl{\Delta})$ and angular $(\bbdelta, \bar{\bbdelta})$. The procedure for separation of the Maxwell equations can be further simplified by introducing new operators
\begin{equation}
    \begin{alignedat}{3}
        \mathscr{D}_n &= \partial_r - \frac{i K}{\Delta} + 2n \frac{r-M}{\Delta} ~,\qquad && \mathscr{D}_n^\dagger = \partial_r + \frac{i K}{\Delta} + 2n \frac{r-M}{\Delta} ~,\\
        \mathscr{L}_n &= \partial_\theta - Q + n \cot\theta ~,\qquad && \mathscr{L}_n^\dagger = \partial_\theta + Q + n \cot\theta ~,
    \end{alignedat}
\end{equation}
where we define the functions $K=(r^2+a^2)\omega - m a$, $Q = a \omega \sin\theta - m \csc\theta$.
In this definition, $n$ is any integer.
These operators are related to the tetrad by
\begin{align}
    \label{eq3:ChadrasekharOperators}
    \mathbbl{D} = \mathscr{D}_0 ~,\qquad \mathbbl{\Delta} = - \frac{\Delta}{2 \rho^2 }\mathscr{D}^\dagger_0 ~,\qquad \bbdelta = \frac{1}{\sqrt{2} \bar{\rho}} \mathscr{L}^\dagger_0 ~,\qquad \bar\bbdelta = \frac{1}{\sqrt{2} \bar{\rho}^*} \mathscr{L}_0 ~,
\end{align}
as a result of the substitutions $\partial_t\rightarrow-i \omega$, $\partial_\varphi\rightarrow i m$.
We may use the fact that $\mathscr{D}_n$ and $\mathscr{L}_n$ act mostly as radial and angular derivatives, respectively, to deduce the properties
\begin{subequations}
    \begin{align}
        \label{eq3:propDeltaD}
        \mathscr{D}_n \Delta &= \Delta \mathscr{D}_{n+1} ~, \\[0.15cm]
        \label{eq3:propSinL}
        \mathscr{L}_n \sin\theta &= \sin\theta\, \mathscr{L}_{n+1} ~, \\[0.15cm]
        \label{eq3:propBarRhoD}
        \left(\mathscr{D}_n + \frac{q}{\bar{\rho}^*} \right) \frac{1}{(\bar{\rho}^*)^p} &= 
        \frac{1}{(\bar{\rho}^*)^p} \left(\mathscr{D}_n + \frac{q-p}{\bar{\rho}^*} \right) ~, \\[0.15cm]
        \label{eq3:propBarRhoL}
        \left(\mathscr{L}_n + \frac{i q a \sin\theta}{\bar{\rho}^*} \right) \frac{1}{(\bar{\rho}^*)^p} &= 
        \frac{1}{(\bar{\rho}^*)^p} \left(\mathscr{L}_n + \frac{i (q-p) a \sin\theta}{\bar{\rho}^*} \right) ~, \\[0.15cm]
        \label{eq3:propCommutLD}
        \left(\mathscr{D}_n + \frac{q}{\bar{\rho}^*} \right) 
        \left(\mathscr{L}_n + \frac{i q a \sin\theta}{\bar{\rho}^*} \right) &= 
        \left(\mathscr{L}_n + \frac{i q a \sin\theta}{\bar{\rho}^*} \right)
        \left(\mathscr{D}_n + \frac{q}{\bar{\rho}^*} \right) ~,
    \end{align}
\end{subequations}
for any integers $p,q,n$, holding also for either $\mathscr{D}^\dagger_n$ or $\mathscr{L}^\dagger_n$.

In order to achieve the separable form, we still need to perform a replacement of the Maxwell NP scalars by new dynamical variables
\begin{align}
    \label{eq3:phiBarRhoToPhi}
    \Phi_0 = \phi_0 ~,\qquad \Phi_1 = \sqrt{2} \bar{\rho}^* \phi_1 ~,\qquad \Phi_2 = 2 (\bar{\rho}^*)^2 \phi_2  ~,
\end{align}
and using properties \eref{eq3:propBarRhoD} and \eref{eq3:propBarRhoL}, we go from Eqs. \eref{eq3:phiAllExpand} to
\begin{subequations}
    \label{eq3:AllDPhiLPhi}
    \begin{align}
        \label{eq3:D0Phi2L0Phi1}
        \left( \mathscr{D}_0 - \frac{1}{\bar{\rho}^*} \right) \Phi_2 &=
        \left( \mathscr{L}_0 + \frac{i a \sin\theta}{\bar{\rho}^*} \right) \Phi_1 ~, \\
        \label{eq3:Dd0Phi1Ld1Phi2}
        \Delta \left( \mathscr{D}^\dagger_0 + \frac{1}{\bar{\rho}^*} \right) \Phi_1 &= 
        -\left( \mathscr{L}^\dagger_1 - \frac{i a \sin\theta}{\bar{\rho}^*} \right) \Phi_2 ~, \\
        \label{eq3:D0Phi1L1Phi0}
        \left( \mathscr{D}_0 + \frac{1}{\bar{\rho}^*} \right) \Phi_1 &= 
        \left( \mathscr{L}_1 - \frac{i a \sin\theta}{\bar{\rho}^*} \right) \Phi_0 ~, \\
        \label{eq3:Dd1Phi0Ld0Phi1}
        \Delta \left( \mathscr{D}^\dagger_1 - \frac{1}{\bar{\rho}^*} \right) \Phi_0 &= 
        -\left( \mathscr{L}^\dagger_0 + \frac{i a \sin\theta}{\bar{\rho}^*} \right) \Phi_1 ~.
    \end{align}
\end{subequations}
Now we may use the commutation relation \eref{eq3:propCommutLD} together with \eref{eq3:propDeltaD} to separate the equations for $\Phi_0$ and $\Phi_2$.
In order to obtain the first equation, we must first apply the operator $(\mathscr{L}^\dagger_0 + i a \sin\theta/\bar{\rho}^*)$ to \eqref{eq3:D0Phi1L1Phi0} and then use the commutation relation to substitute \eqref{eq3:Dd1Phi0Ld0Phi1}.
Similarly, applying $(\mathscr{L}_0 + i a \sin\theta/\bar{\rho}^*)$ to \eqref{eq3:Dd0Phi1Ld1Phi2} we obtain the final equation.
This yield
\begin{align}
    \label{eq3:DDLLPhi0}
    \left[ \Delta \mathscr{D}_1 \mathscr{D}^\dagger_1 + \mathscr{L}^\dagger_0 \mathscr{L}_1 + 2 i \omega (r+i a \cos\theta) \right] \Phi_0 = 0 ~, \\
    \label{eq3:DDLLPhi2}
    \left[ \Delta \mathscr{D}^\dagger_0 \mathscr{D}_0 + \mathscr{L}_0 \mathscr{L}^\dagger_1 - 2 i \omega (r+i a \cos\theta) \right] \Phi_2 = 0 ~.
\end{align}
Still, there is another way of combining equations, \emph{i.e} Eq. \eref{eq3:Dd0Phi1Ld1Phi2} with \eref{eq3:Dd1Phi0Ld0Phi1} and the remaining two form the set
\begin{align}
    \label{eq3:LLPhi0DDPhi2}
    \mathscr{L}_0 \mathscr{L}_1 \Phi_0 &= \mathscr{D}_0 \mathscr{D}_0 \Phi_2 ~,\\
    \label{eq3:LLPhi2DDPhi0}
    \mathscr{L}^\dagger_0 \mathscr{L}^\dagger_1 \Phi_2 &= \Delta \mathscr{D}^\dagger_0 \mathscr{D}^\dagger_0 \Delta \Phi_0 ~.
\end{align}
Thus, we went from four first-order differential equations relating three NP scalars to four second-order differential equations, two of each decoupled, eliminating the need for the scalar $\Phi_1$.
The last two equations imply that each of the complex NP scalars contains all the information necessary to describe an EM wave (two polarizations).
One may think that we only need one of each group of equations to solve all perturbations, but no closed form solution has yet been found.
Thus the problem has to be tackled using approximations or numerical methods, recurring to all last four equations (\ref{eq3:DDLLPhi0}\textendash\ref{eq3:LLPhi2DDPhi0}), as we will see below.

Due to the nature of the operators $\mathscr{D}_n$ and $\mathscr{L}_n$, we may separate the equations for $\Phi_0 \sim R_{+1}(r) S_{+1}(\theta)$ and $\Phi_2 \sim R_{-1}(r) S_{-1}(\theta)$ into two pairs of equations, 
\begin{subequations}
    \label{eq3:separationRSp}
    \begin{align}
        \label{eq3:separationDDRp}
        \left(\Delta \mathscr{D}_0 \mathscr{D}^\dagger_0 + 2 i \omega r \right) \Delta R_{+1} 
        &= \lambdabar \Delta R_{+1} ~, \\
        \label{eq3:separationLLSp}
        \left( \mathscr{L}^\dagger_0 \mathscr{L}_1 - 2 a \omega \cos\theta \right) S_{+1}
        &= - \lambdabar S_{+1}  ~, 
    \end{align}
\end{subequations}
and
\begin{subequations}
    \label{eq3:separationRSm}
    \begin{align}
        \label{eq3:separationDDRm}
        \left( \Delta \mathscr{D}^\dagger_0 \mathscr{D}_0 - 2 i \omega r \right) R_{-1}
        &= \lambdabar R_{-1} ~, \\
        \label{eq3:separationLLSm}
        \left( \mathscr{L}_0 \mathscr{L}^\dagger_1 + 2 a \omega \cos\theta \right) S_{-1}
        &= - \lambdabar S_{-1}  ~,
    \end{align}
\end{subequations}
where $\lambdabar$ is a separation constant.
We use the property \eref{eq3:propDeltaD} in to obtain \eqref{eq3:separationDDRp}.
The constant $\lambdabar$ must be real, as the angular differential operators $\mathscr{L}_n$ are also real.
Notice that we do not distinguish the separation constants of both equations. Performing the transformation $\theta \rightarrow \pi-\theta$, the angular operators transforms as $\mathscr{L}^\dagger_0 \mathscr{L}_1 \rightarrow \mathscr{L}_0 \mathscr{L}^\dagger_1$. Then if $S_{+1}(\theta)$ is a solution for \eqref{eq3:separationLLSp} for a given separation constant $\lambdabar$, this implies that $\tilde{S}_{-1}(\theta)=S_{+1}(\pi-\theta)$ is a solution for \eqref{eq3:separationLLSm} for the same constant. 
In other words, the separation constant must be the same for both equations. 
Also, solutions $R_{-1}$ and $\Delta R_{+1}$ obey the same complex conjugate equations due to $\mathscr{D}^\dagger_n=(\mathscr{D}_n)^*$.

The second-order equations relating $\Phi_0$ and $\Phi_2$ can be separated in the same fashion. Naturally, the separation constant will differ from the eigenvalue Eqs. \eref{eq3:separationRSp} and \eref{eq3:separationRSm}.
Using the same substitutions made previously, we divide each equation by the corresponding ansatz to obtain
\begin{align}
    \label{eq3:separationB}
    \frac{\mathscr{L}_0 \mathscr{L}_1 S_{+1}}{S_{-1}} = \frac{\Delta \mathscr{D}_0 \mathscr{D}_0 R_{-1}}{\Delta R_{+1}} &= \mathscr{B} ~, \\
    \label{eq3:separationBdagger}
    \frac{\mathscr{L}^\dagger_0 \mathscr{L}^\dagger_1 S_{-1}}{S_{+1}} = \frac{\Delta \mathscr{D}^\dagger_0 \mathscr{D}^\dagger_0 \Delta R_{+1}}{R_{-1}} &= \mathscr{B} ~.
\end{align}
The separation constant $\mathscr{B}$ is real and equal for both equations. This claim rests on the same arguments as for the eigenvalue $\lambdabar$. We also make the angular functions $S_{-1}$, $S_{+1}$ equally normalized. We may observe the latter by assuming two different separation constants $\mathscr{B}_1$, $\mathscr{B}_2$. Then, we have
\begin{align}
    \begin{split}
        (\mathscr{B}_1)^2 \int_0^\pi \dd\theta \sin\theta \, (S_{-1})^2 &=
        \int_0^\pi \dd\theta \sin\theta \, ( \mathscr{L}_0 \mathscr{L}_1  S_{+1} )( \mathscr{L}_0 \mathscr{L}_1  S_{+1} ) \\
        &= \int_0^\pi \dd\theta \sin\theta \, ( \mathscr{L}^\dagger_0 \mathscr{L}^\dagger_1 \mathscr{L}_0 \mathscr{L}_1  S_{+1} ) S_{+1} \\
        &=  \mathscr{B}_1 \mathscr{B}_2 \int_0^\pi \dd\theta \sin\theta \, (S_{+1})^2 ~,
    \end{split}
\end{align}
where we used integration by parts twice.
Thus $(\mathscr{B}_1)^2 = \mathscr{B}_1 \mathscr{B}_2 = \mathscr{B}^2$.
We can compute the coefficient by computing the operation
\begin{align}
    \begin{split}
        \mathscr{L}^\dagger_0 \mathscr{L}^\dagger_1 \mathscr{L}_0 \mathscr{L}_1 &=
        \mathscr{L}^\dagger_0 ( \mathscr{L}_0 \mathscr{L}^\dagger_1 - 4 a\omega \cos\theta )\mathscr{L}_1 \\
        &= \mathscr{L}_0 \mathscr{L}^\dagger_1 ( - \lambdabar + 2 a\omega \cos\theta) - 4 a\omega \cos\theta \,\mathscr{L}^\dagger_0 \mathscr{L}_1 + 4 a\omega \sin\theta \,\mathscr{L}_1 \\
        &= -\lambdabar \, \mathscr{L}_0 \mathscr{L}^\dagger_1 + 2 a\omega \left[ \cos\theta \, \mathscr{L}_0 \mathscr{L}^\dagger_1 -\sin\theta(\mathscr{L}_1+\mathscr{L}^\dagger_1) \right] \\ &\mkern200mu - 4 a\omega \cos\theta \,\mathscr{L}^\dagger_0 \mathscr{L}_1 + 4 a\omega \sin\theta \,\mathscr{L}_1 \\
        &= (-\lambdabar + 2 a\omega \cos\theta ) \mathscr{L}_0 \mathscr{L}^\dagger_1 + 4 a\omega Q \sin\theta \\
        &= (-\lambdabar + 2 a\omega \cos\theta ) (-\lambdabar - 2 a\omega \cos\theta ) + 4 a\omega ( -a\omega \sin^2\theta + m ) \\
        &= \lambdabar^2 - 4 a^2 \omega^2 + 4 a \omega m = \mathscr{B}^2 ~,
    \end{split}
\end{align}
applied on the angular function $S_{+1}$.
The commutation relations between the angular operators can be found directly or by noticing that $[\bm{e}_a,\bm{e}_b]= \eta^{cd} (\gamma_{cba}-\gamma_{cab}) \bm{e}_d$.
Then using \eqref{eq3:separationLLSp} it is possible to eliminate the second-order angular operators.
To obtain $\mathscr{B}^2$, the same procedure could be done for $S_{-1}$ or the the radial functions.

It will be more profitable to study Eqs. \eref{eq3:DDLLPhi0} and \eref{eq3:DDLLPhi2} as a special case of the Teukolsky master equation \cite{Teukolsky1972} which describes all the linearized perturbations around the Kerr BH.
The generality of this equation is the primary reason for the focus on the EM case.
The treatment for GWs differs in the perturbation formalism only in algebraic complexity, resulting in the same master equation.
With the Teukolsky master equation we can proceed considering general perturbations, but there are several numerical and analytical details that make EM waves and GWs differ later on \cite{TeukolskyPress1973b}.

The general equation reads
\begin{align}
    \begin{split}
        & \frac{1}{\Delta^s} \frac{\partial}{\partial r} \left( \Delta^{s+1} \frac{\partial \Upsilon_s}{\partial r} \right) 
        + \frac{1}{\sin\theta} \frac{\partial}{\partial\theta} \left( \sin\theta \frac{\partial \Upsilon_s}{\partial \theta} \right) 
        - \left[ \frac{(r^2+a^2)^2}{\Delta} - a^2 \sin^2\theta \right]\frac{\partial^2 \Upsilon_s}{\partial t^2} \\[0.15cm]
        & - \frac{4 M a r}{\Delta}\frac{\partial^2 \Upsilon_s}{\partial t \partial \varphi} 
        - \left( \frac{a^2}{\Delta} -\frac{1}{\sin^2\theta} \right)\frac{\partial^2 \Upsilon_s}{\partial \varphi^2} 
        + 2s\left[ \frac{M(r^2-a^2)}{\Delta} - r - i a \cos\theta \right] \frac{\partial \Upsilon_s}{\partial t} \\[0.15cm]
        &+ 2s\left[ \frac{a(r-M)}{\Delta}+\frac{i \cos\theta}{\sin^2\theta}\right] \frac{\partial \Upsilon_s}{\partial \varphi}
        - (s^2 \cot^2\theta - s) \Upsilon_s = 0 ~,
    \end{split}
    \label{eq3:teukolsky}
\end{align}
where $s$ is the field \emph{spin weight} and each field quantity $\Upsilon_s$ is related to the NP scalars as shown in the~\tref{tb3:solutionsTeukolskyEq}.
Depending on the spin weight, the equation may describe massless scalar ($s=0$) or Dirac fields ($s=\pm \tfrac{1}{2}$), as well as electromagnetic ($s=\pm 1$) or gravitational waves ($s=\pm 2$). Substituting the spin-weight for the EM waves we obtain Eqs. \eref{eq3:DDLLPhi0} and \eref{eq3:DDLLPhi2}.
\begin{table}[h]
    \centering
    \tabulinesep=1.5mm
    \begin{tabu}{@{\hskip 0.25cm}c@{\hskip 0.75cm}l@{\hskip 0.25cm}}
        \hline
        $s$ & $\qquad ~ \Upsilon_s$ \\
        \hline\hline
        $+1$ & $\Phi_0 = \phi_0$ \\
        \hline
        $-1$ & $\Phi_2 = 2 (\bar{\rho}^*)^2 \phi_2$ \\
        \hline
        $+2$ & $\Psi_0 = \psi_0$  \\
        \hline
        $-2$ & $\Psi_4 = 4 (\bar{\rho}^*)^4 \psi_4$ \\
        \hline
    \end{tabu}
    \caption{Newman-Penrose fields that obey the Teukolsky master equation for different spin-weights \cite{Chandrasekhar1998}}
    \label{tb3:solutionsTeukolskyEq}
\end{table}

Obviously, Teukolsky's equation is explicitly independent of $t$ and $\varphi$, thus $\Upsilon_s$ accepts a decomposition in $e^{-i \omega t + i m \varphi}$, which we already assumed in the EM case to separate the equations.
Stationarity and axisymmetry of the spacetime geometry guarantees this form.
The azimuthal wave number $m$ must be an integer, due to periodic boundary conditions on the BL coordinate $\varphi$.
We may separate all perturbations in a completely general mode decomposition
\begin{align}
    \label{eq3:multipoleExpansion}
    \Upsilon_s = \int\dd\omega \, e^{-i \omega t} \left( \,\sum_{\ell,m} e^{i m \varphi} \, {}_{s}S_{\ell m}(\theta) {}_{s}R_{\ell m}(r) \right) ~.
\end{align}
The integer $\ell$ plays a role in labelling all possible solutions for the eigenvalue problem of both radial and angular equations,
\begin{align}
    \label{eq3:teukolskyRadial}
    \frac{1}{\Delta^s} \frac{\dd}{\dd r} \left( \Delta^{s+1} \frac{\dd\, {}_{s}R_{\ell m}}{\dd r} \right)
    + \left[ \frac{K^2 - 2 i s (r-M)K}{\Delta} + 4 i s \omega r -{}_{s}\mathscr{F}_{\ell m}  \right] {}_{s}R_{\ell m} = 0 ~, \\[0.15cm]
    \label{eq3:teukolskyAngular}
    \frac{1}{\sin\theta} \frac{\dd}{\dd\theta} \left( \sin\theta \frac{\dd\, {}_{s}S_{\ell m}}{\dd \theta} \right)
    + \left[ a^2 \omega^2 \cos^2\theta - 2 s a \omega \cos\theta - \frac{(m + s \cos\theta)^2}{\sin^2\theta} + s + {}_{s}\mathscr{A}_{\ell m} \right] {}_{s}S_{\ell m}  = 0~.
\end{align}
The radial and angular eigenvalues are related to the separation constant on Eqs. \eref{eq3:separationRSp} and \eref{eq3:separationRSm} through
\begin{align}
    {}_{s}\mathscr{F}_{\ell m} = {}_{s}\mathscr{A}_{\ell m} - 2 m a \omega + a^2 \omega^2 \stackrel[(s=\pm 1)]{}{=} \lambdabar - s(s+1) ~.
    \label{eq3:relationEVs}
\end{align}
Due to the form of the angular equation, the eigenvalues ${}_{s}\mathscr{F}_{\ell m}$, \,${}_{s}\mathscr{A}_{\ell m}$ as well as the function ${}_{s}S_{\ell m}(\theta)$ depends also on the coupling $a \omega$.
Clearly, the same does not hold for the radial function ${}_{s}R_{\ell m}(r)$. 

%----------------------------------------------------------------------------------------

\section{Spin-Weighted Spheroidal Harmonics}

To shed some light into the explicit form of ${}_{s}\mathscr{A}_{\ell m}$, we will need to dive into the eigenvalue problem for the angular equation. We may transform \eqref{eq3:teukolskyAngular} into a more familiar form using the change of coordinate $z=\cos\theta$ and renaming the dimensionless parameter $c=a \omega$, obtaining
\begin{align}
    \frac{\dd}{\dd z} \left[ (1-z^2) \frac{\dd\, {}_{s}S_{\ell m}}{\dd z} \right] + \left[ (c z)^2 - 2 c s z  -\frac{(m + s z)^2}{1 - z^2} + s  + {}_{s}\mathscr{A}_{\ell m} \right] {}_{s}S_{\ell m} = 0 ~.
    \label{eq3:swshEq}
\end{align}
We may also use freely ${}_{s}S_{\ell m}(\cos\theta) \equiv {}_{s}S_{\ell m}(\theta)$.
We will consider $c$ real as we are analyzing superradiance of EM waves in vacuum, although we could generalize the spin-weighted spheroidal harmonic (SWSH) equation to imaginary $c$ values to describe waves in a particular medium.

Spherically symmetric problems allow for a decomposition using spherical harmonics $Y_{\ell m}(\theta,\varphi)$ of angular dependent functions with finite boundary conditions.
These have innumerable applications in physics such as the hydrogen atom or the description of anisotropies in the cosmic microwave background.
By setting $s=0$ and $c=0$ (spherical), then it is clear that the solutions for \eqref{eq3:swshEq} are given by the associated Legendre polynomials, $P^m_\ell(z)$. Therefore, ${}_{s}S_{\ell m}$ is a generalization of a spherical harmonic \cite{Berti2005}, with
\begin{align}
    {}_{0}S_{\ell m}(\theta,\varphi) \stackrel[(c=0)]{}{=} Y_{\ell m}(\theta,\varphi) ~,\qquad {}_{0}\mathscr{A}_{\ell m} \stackrel[(c=0)]{}{=} \ell (\ell + 1) ~,
\end{align}
where $\uu[s]{S}{\ell m}(\theta,\varphi) \equiv e^{i m \varphi} \uu[s]{S}{\ell m}(\theta)$.
The values of $\ell$ are non-negative integers, with the restriction of $\ell \ge |m|$. 
We require that spin-weighted spheroidal harmonics (SWSHs) are similarly normalized to unity and also a complete set of orthogonal functions, for any spin-weight and coupling $c$,
\begin{align}
    \label{eq3:SWSHorthogonality}
    \int_{0}^\pi \dd\theta \sin\theta \,{}_{s}S_{\ell m}(\theta) \,{}_{s}S_{\ell' m'}(\theta)  =
    \int_{-1}^{1} \dd z \,{}_{s}S_{\ell m}(z) {}_{s}S_{\ell' m}(z) = \frac{1}{2\pi} \delta_{\ell \ell'} ~.
\end{align}

Perturbations of any type in Schwarzschild spacetime are written using spin-weighted \emph{spherical} harmonics, ${}_{s}Y_{\ell m}(\theta,\varphi)$, which are still spherical harmonics ($c=0$).
Due to the shared symmetries with spherical harmonics it is possible to find a closed form for $s\ne0$ harmonics.
We can raise and lower spin-weight with the use of the operators commonly denoted as $\bar{\eth}$ and $\eth$ (see \aref{AppendixSWHs}), respectively, applied on $Y_{\ell m}$,
\begin{align}
    \begin{split}
        (\bar{\eth})^s Y_{\ell m} &= \sqrt{\frac{(\ell+s)!}{(\ell-s)!}} \,{}_{s}Y_{\ell m}  ~, \\
        (-1)^s  (\eth)^s Y_{\ell m} &= \sqrt{\frac{(\ell+s)!}{(\ell-s)!}} \,{}_{-s}Y_{\ell m} ~,
    \end{split}
\end{align}
limited by $|s|\le\ell$.
Eqs. \eref{eq3:separationB} and \eref{eq3:separationBdagger} are closely related to former, as applications of the operators $\mathscr{L}^\dagger_s$ and $\mathscr{L}_s$ generalize $\bar{\eth}$ and $\eth$, respectively.

Thus, for the non-spherical symmetry we could in principle obtain all SWSHs if we knew the closed form for ${}_{0}S_{\ell m}$ and its eigenvalues. The problem lies in the fact that no such decomposition of elementary function was found. This require us to follow numerical and approximate methods to find the values of ${}_{s}S_{\ell m}$.

The major advances on the study of the eigenvalues of the SWSHs was performed by Leaver in 1985 . Working out the asymptotical and critical behavior of the equation, we observe the equation diverges at the poles, $z = \pm 1$, where the it takes the form $(1 \mp z) \,{}_{s}S_{\ell m}{}'(z) \sim \mp \tfrac{1}{4} (m \pm s)^2 \, {}_{s}S_{\ell m}(z)$. In order to guarantee that a SWSH is everywhere analytical, Leaver proposed the series expansion at $z=-1$,
\begin{align}
    \label{eq3:SWSHseriesLeaver}
    {}_{s}S_{\ell m}(z) = e^{c z} (1+z)^{k_-} (1-z)^{k_+} \sum_{p=0}^\infty a_p (1+z)^p ~,
\end{align}
where $k_{\pm} = \tfrac{1}{2}|m \pm s|$.
The exponential in the ansatz accounts for the large $z$ behavior of the equation.
Substituting in the angular equation, we obtain a three-term recurrence relation between the expansion coefficients $a_p$ and the boundary condition at $z=-1$,
\begin{align}
    \label{eq3:ap3CoefRecursion}
    \alpha_p a_{p+1} + \beta_p a_p + \gamma_p a_{p-1} = 0 ~,\qquad
    \alpha_0 a_1 + \beta_0 a_0 = 0 ~,
\end{align}
where
\begin{align}
    \label{eq3:LeaverCoefRecursion}
    \begin{split}
        \alpha_p &= -2 (1 + p) (1 + 2 k_{+} + p) ~,\\
        \beta_p  &= (k_{-} + k_{+} + p - s) (1 + k_{-} + k_{+} + p + s) \\
        &\qquad\qquad\qquad - 2 c (1 + 2 k_{-} + 2 p + s) - c^2 - {}_{s}\mathscr{A}_{\ell m} ~,\\
        \gamma_p &= 2 c (k_{-} + k_{+} + p + s) ~.
    \end{split}
\end{align}
We then find an equation for the eigenvalue ${}_{s}\mathscr{A}_{\ell m}$ with explicit dependence on $m$, $s$ and $c$, by combining the previous relations into a continued fraction,
\begin{align}
    \label{eq3:evInversion0th}
    \beta_0 = \frac{\alpha_0 \gamma_1}{\beta_1 -} \frac{\alpha_1 \gamma_2}{\beta_2 -} \frac{\alpha_2 \gamma_3}{\beta_3 -} \cdots \left( \equiv  \frac{\alpha_0 \gamma_1}{\beta_1 - \frac{\alpha_1 \gamma_2}{\beta_2 - \frac{\alpha_2 \gamma_3}{\beta_3 - \dots}}} \right)  ~.
\end{align}
We can also consider the $r$-th inversion of \eqref{eq3:evInversion0th},
\begin{align}
    \label{eq3:evInversionRth}
    \beta_r - \frac{\alpha_{r-1} \gamma_r}{\beta_{r-1} -} \frac{\alpha_{r-2} \gamma_{r-1}}{\beta_{r-2} -} \cdots \frac{\alpha_{1} \gamma_2}{\beta_{1} -} \frac{\alpha_{0} \gamma_{1}}{\beta_{0}} = \frac{\alpha_{r} \gamma_{r+1}}{\beta_{r+1} -} \frac{\alpha_{r+1} \gamma_{r+2}}{\beta_{r+2} -} \cdots ~.
\end{align}
These equations involve an infinite fraction that depends explicitly on $c$, $m$, $s$, which leads to suspicion that this leads to an infinite spectrum.
This is in a close parallel to the spherical case, where we have a infinite number of harmonics, although no proof has been found.
If we notice that $\gamma_p\propto c$, then the zero order expansion of the eigenvalue expansion in $c\ll 1$ leads to $\beta_r=0$.
In the spherical geometry, this corresponds to truncating the series at $r$, as $\gamma_p=0$ for any $p$. Thus, the eigenvalue root depends explicitly on the integer $r$, entailing the discretization of the spectra
\begin{align}
    \label{eq3:evSWSH0th}
    {}_{s}\mathscr{A}_{\ell m} = \ell(\ell+1) - s(s+1) + \mathscr{O}(a\omega)~,
\end{align}
where we identified $\ell=r+k_{+}+k_{-}$.
Since $r\ge 0$, then we must have $\ell\ge\max\{|m|,|s|\}$, \emph{i.e.} the leading contribution for the monopole expansion is the dipole for EM waves and the quadrupole for GWs. 
Changing $r$ for $\ell$ corresponds simply to a relabeling of the eigenfunctions in order to match the values of the spectra when $c=0$ and also when $s=0$.

It will be useful to expand ${}_{s}\mathscr{A}_{\ell m}$ in high order terms in order to obtain the series coefficients for the eigenvalue (see \aref{AppendixEigenvalues}).
Up to sixth order, only the zero-order term depends on the sign of the spin weight, in agreement with \eqref{eq3:relationEVs}.
In the general angular equation, inversion of spin corresponds to inversion of poles, \emph{i.e.} stays invariant under the transformation $(s,z)\to(-s,-z)$.
Under this transformation, the eigenvalue must obey $s+ {}_{s}\mathscr{A}_{\ell m} = -s + {}_{-s}\mathscr{A}_{\ell m}$, thus it is beneficial to define 
\begin{align}
    \label{eq3:sElm}
    {}_{s}\mathscr{A}_{\ell m} = {}_{s}\mathscr{E}_{\ell m} - s(s+1) ~,
\end{align}
to also exploit the symmetry of spin inversion in numerical computations.
This way, we can simply write that $\lambdabar={}_{\pm1}\mathscr{E}_{\ell m}$.

%----------------------------------------------------------------------------------------

\section{Analytic radial approximations}

Like the angular equation, in general it is not possible to solve the radial \eqref{eq3:teukolskyRadial} by known analytical methods.
The only apparent line of attack would be to numerical solve the equation, but it will be important to find the asymptotic form of ${}_{s}R_{\ell m}$ at infinity as well as its near-horizon behavior. 
For both methods it will prove beneficial to change the variables into dimentionaless quantities,
\begin{align}
    x = \frac{r - r_{+}}{r_{+}} ~,\qquad \tau = \frac{r_{+} - r_{-}}{r_{+}} ~,\qquad \varpi = (2-\tau)( \bar\omega - m \bar{\Omega}_H) ~,
\end{align}
where every barred frequency is normalized relative to the BH horizon, $\bar\omega\equiv\omega r_{+}$. Due to \eref{eq2:spinLimit}, we have that $0\le\tau\le 1$. Using this coordinate, $x\to0$ represents the BH horizon. The radial equation now reads
\begin{align}
    \label{eq3:radialTeukolskyAdimensional}
    \begin{split}
        & \frac{1}{[x(x+\tau)]^s} \frac{\dd}{\dd x}\left( [x(x+\tau)]^{s+1} \frac{\dd \,{}_{s}R_{\ell m}}{\dd x} \right) + \\
        &\qquad + \left[ \frac{\mathscr{K}^2 - i s (2 x+\tau) \mathscr{K}}{x(x+\tau)} + 4 i s (1 + x)\bar{\omega} - {}_{s}\mathscr{F}_{\ell m} \right] {}_{s}R_{\ell m} = 0 ~,
    \end{split}
\end{align}
where we normalize $\mathscr{K} = K/r_{+} = \varpi + x(x+2)\bar{\omega}$. In this method, it will be sufficient to use a spherical approximation for the harmonics eigenvalues, ${}_{s}\mathscr{F}_{\ell m} = \ell(\ell+1) - s(s+1) + \mathscr{O}(a\omega)$, for small enough frequencies.

The near-horizon approximation corresponds to considering only small distances compared to the perturbations characteristic wavelength, $r-r_{+} \ll \omega^{-1}$ ($\bar{\omega} x \ll 1$).
Because superradiant scattering occurs when $\omega\lesssim\Omega_H$, we also neglect terms $\mathscr{O}(\varpi x)$.
In this limit, $\mathscr{K}\simeq\varpi$.
The resultant equation remains singular at the horizons $x=0$ and $x=-\tau$. Thus, making the substitution ${}_{s}R_{\ell m}(x)\simeq x^\alpha (x+\tau)^\beta F(x)$, the function $F(x)$ is analytical if $\alpha+\beta=-s$ and also if $\alpha=-\tfrac{1}{2} s \pm \left(\tfrac{1}{2} s + i \varpi/\tau \right)$.
Boundary conditions at the horizon requires that a physical observer measures a negative radial group velocity of the signal.
In order words, we require the wave to travel into the black hole and never outwards.
Since $x^{\pm i \varpi/\tau} \simeq e^{\pm i \kappa r_{*}}$, where $\kappa=\omega - m \Omega_H$, then the ingoing solution requires $\alpha=-s-i\varpi/\tau$. The near-horizon solution gives
\begin{align}
    \label{eq3:RnearSolution}
    {}_{s}R_{\ell m}(x) \simeq A \, x^{-s-i\varpi/\tau} (x+\tau)^{i\varpi/\tau} \,{}_{2}F_1\left( -\ell, \,\ell+1, \,1-s-\frac{2i\varpi}{\tau}; \,-\frac{x}{\tau}\right) ~,
\end{align}
where $F(x)={}_{2}F_1(a,b,c; x)$ is the hypergeometric function.

Asymptotically, we consider $x\gg\tau$, where $\mathscr{K}\sim x^2 \bar{\omega}$. The resultant equation, 
% \begin{align}
%     x^2 \frac{\dd^2 \,{}_{s}R_{\ell m}}{\dd x^2} + 2 (s+1) x \frac{\dd \,{}_{s}R_{\ell m}}{\dd x}  + ( \bar{\omega}^2 x^2 + 2 i s \bar{\omega} x - {}_{s}\mathscr{F}_{\ell m}) \,{}_{s}R_{\ell m} \sim 0 ~,
% \end{align}
allows to substitute a power law ${}_{s}R_{\ell m}(x)\sim x^\alpha e^{- i \bar{\omega} x} M(x)$. In order for $M(x)$ to be analytic everywhere, then we must have $\alpha^2+(2s+1)\alpha = {}_{s}\mathscr{F}_{\ell m}$. We are left with Kummer's differential equation for $M(x)$, where $\alpha = \ell-s$ or $\alpha=-1-\ell-s$. The general solution is the combination
\begin{align}
    \begin{split}
    {}_{s}R_{\ell m}(x) &\sim   C\,  e^{-i \bar{\omega} x} x^{\ell-s} \, {}_{1}F_1(1+\ell-s, \,2 \ell+2 ; \,-2 i \bar{\omega} x) \\
    &\qquad\qquad + D\,  e^{-i \bar{\omega} x} x^{-1-\ell-s} \, {}_{1}F_1(-\ell-s, \,-2 \ell; \,-2 i \bar{\omega} x) ~,
    \end{split}
    \label{eq3:RfarSolution}
\end{align}
where ${}_{1}F_1(a,b;x)$ are Olver's confluent hypergeometric functions and $B$, $C$ are integration constants.

Altought useful, these solutions do not provide enough physical insight. Another way of obtaining the same assympotic behavior is by transforming the radial equation into a Schrodinger-like potential problem
\begin{align}
    \label{eq3:D2PlusVeff}
    \left[ \frac{\dd^2}{\dd r_{*}^2} + V_\mathrm{eff} \right] {}_{s}U_{\ell m} = 0 ~,
\end{align}
where ${}_{s}U_{\ell m}=\sqrt{\Delta^s (r^2 + a^2)} \,{}_{s}R_{\ell m}$ and the tortoise coordinate $r_*$ was defined in \eqref{eq2:tortoise}.
We write the potential as a function of the radial coordiante,
\begin{align}
    V_\mathrm{eff} = \frac{K^2 - 2 i s (r-M) K + \Delta ( 4 i s \omega r - {}_{s}\mathscr{F}_{\ell m} )}{(r^2+a^2)^2} - G^2 - \frac{\dd G}{\dd r_{*}}
\end{align}
where $G = s(r-M)/(r^2+a^2)+ r \Delta/(r^2+a^2)^2$.
Even tought $r_*$ is related to the radial coordinate through an non-invertable relation, we use this coordinate to remove all the $\Delta$ singularities from the diferential equation. 

Taking $r\to\infty$ ($r_{*}\to\infty$), the potential becomes
\begin{align}
    \label{eq3:asymptoticVeff}
    V_\mathrm{eff} \sim \omega^2 + \frac{2 i s \omega}{r} ~,
\end{align}
with assympotic solution ${}_{s}U_{\ell m}\sim r^{\pm s} e^{\mp i \omega r_{*}}$.
The combination of both solutions corresponds to
\begin{align}
    {}_{s}R_{\ell m}(r) \sim A_\mathrm{in}\, \frac{e^{-i \omega r_*}}{r} + A_\mathrm{out}\, \frac{e^{i \omega r_*}}{r^{2s+1}} ~.
    \label{eq3:asymptoticR}
\end{align}
The ingoing and outgoing wave coeficients can be related to the integration coeficients $B, C$ by expanding the hypergeometric function in \eref{eq3:RfarSolution} at infinity,
\begin{align}
    \begin{split}
    A_\mathrm{in} &= \left[ C (-2 i \bar{\omega})^{-\ell+s-1} \frac{\Gamma(2\ell+2)}{\Gamma(\ell+s+1)} + D (-2 i \bar{\omega})^{\ell+s} \frac{\Gamma(-2\ell)}{\Gamma(-\ell+s)} \right] r_{+} ~, \\
    A_\mathrm{out} &= \left[ C (2 i \bar{\omega})^{-\ell-s-1} \frac{\Gamma(2\ell+2)}{\Gamma(\ell-s+1)} + D (2 i \bar{\omega})^{\ell-s} \frac{\Gamma(-2\ell)}{\Gamma(-\ell-s)} \right] (r_{+})^{2s+1}  ~.
    \end{split}
\end{align}

We may notice the ratio of gamma functions with negative integer valued arguments, which can be misinterpreted as a divergence due to existing poles of $\Gamma$.
This is a mere artifact of the asymptotic expantion for general confluent hypergeometric arguments.
A way to circumvent this problem is to consider Euler's reflection formula for any value of $\ell$ and then take the limit to the integer set,
\begin{align}
    \label{eq3:negativeGammaRatio}
    \lim_{\ell\in\mathbbl{Z}} \frac{\Gamma(-2\ell)}{\Gamma(-\ell\pm s)} &= \frac{\Gamma(\ell\mp s+1)}{\Gamma(2\ell+1)} \, \lim_{\ell\in\mathbbl{Z}} \frac{\sin(\ell\pi)\cos(\mp s \pi)}{\sin(2\ell\pi)} = \frac{(-1)^{\ell+s}}{2} \frac{(\ell \mp s)!}{(2\ell)!} ~.
\end{align}
 
At the event horizon $r=r_+$, $r_*\to-\infty$ and $\Delta=0$.
The effective radial potential is simplified to a constant
\begin{align}
    V_\mathrm{eff} \simeq \left( \kappa- i s \frac{r_+ - M}{2 M r_+} \right)^2 ~.
\end{align}
Due to the logarithmic behavior of $r_*$ at the horizon, the solution takes the form ${}_{s}U_{\ell m}\sim e^{\pm i \kappa r_{*} } (r-r_{+})^{\pm s/2} \sim \Delta^{\pm s/2} \,e^{\pm i \kappa r_{*} }$. The boundary conditions at $r=r_{+}$ state that the horizon solution must only have the ingoing solution
\begin{align}
    \label{eq3:boundaryR}
    {}_{s}R_{\ell m} \simeq A_\mathrm{hole} \,\Delta^{-s} \,e^{- i \kappa r_{*} } ~.
\end{align}
Expanding solution \eref{eq3:RnearSolution}, the integration constants relate through $A_\mathrm{hole} = (r_{+})^{-s} A$.

With both approximations it is possible to extend the solutions of small frequency waves to overlapping regions and perform a matching of coefficients, which can be used to find how much of the wave is reflected/amplified.
The near region $r-r_{+}\ll \omega^{-1}$ and the asymptotic region $r-r_{+}\gg r_{+}$ overlap when $\omega r_{+} \ll 1$ ($\bar{\omega}\ll 1$). The overlapping region becomes larger as $\omega r_{+}$ becomes smaller.
We proceed by expanding the far region solution \eref{eq3:RfarSolution} at the horizon, $x=0$, where the lowest order terms are simply
\begin{align}
    {}_{s}R_{\ell m} \simeq C \, x^{\ell-s} + D \, x^{-1-\ell-s} ~.
\end{align}
On the other hand, expanding the near region solution \eref{eq3:RfarSolution} at infinity we get
\begin{align}
    \begin{split}
        {}_{s}R_{\ell m} &\sim A \, \tau^{-\ell} \frac{ \Gamma(2\ell+1)}{\Gamma(\ell+1)} \frac{\Gamma(1-s - 2 i \varpi/\tau)}{\Gamma(\ell+1 - s - 2 i \varpi/\tau)} x^{\ell-s} \\
        &\qquad\qquad + A \, \tau^{\ell+1} \frac{\Gamma(-2\ell-1)}{\Gamma(-\ell)} \frac{\Gamma(1 - s - 2 i \varpi/\tau)}{ \Gamma(-\ell - s - 2 i \varpi/\tau)} x^{-\ell-1-s} ~.
    \end{split}
\end{align}
The matching is possible because the solutions when expanded in regions in the limit of their validity are given in terms of two monomials of $x^{\ell-s}$ and $x^{-1-\ell-s}$.
A combination of these results yields
\begin{align}
    \frac{D}{C} = \frac{(-1)^{\ell+1}}{2} \frac{(\ell !)^2}{(2\ell)! (2\ell+1)!} \frac{\Gamma(\ell+1 - s - 2 i \varpi/\tau)}{\Gamma(-\ell - s - 2 i \varpi/\tau)} \, \tau^{2\ell+1} ~,
\end{align}
where we used the same identification as in \eqref{eq3:negativeGammaRatio} for the $\Gamma$ functions with negative arguments.
With this result it is possible to find the ratio between the incoming and the outgoing energy from the BH.

%----------------------------------------------------------------------------------------

\section{Amplification factor ${}_{s}Z_{\ell m}$}

Potential barrier problems are heavilly associated with reflection and absortion of radiation. 
Central potentials have waves scattered diferently for each mode $(\omega, \ell, m)$, depending on the incident angle of the wave.
The stress-energy tensor allows to define conserved currents, which can be used to compute the flow of energy and angular momentum.
In particular, we will be interested in calculating the asymptotic energy flow going inward and outward of the BH.

Different Killing vectors have distinct currents, due to different possible projections of the stress-energy tensor.
These currents are conserved due to $\nabla_\mu T^{\mu\nu}=0$ and the Killing \eqref{eq2:killing}.
The energy flux is defined as
\begin{align}
    \label{eq3:fluxE}
    \dd E = T^{\mu}{}_\nu \, k^\nu \,\dd\Sigma_\mu
\end{align}
where $\dd \Sigma_\mu$ is defined as the 3-surface element.
An asymptotically flat geometry such as the Kerr metric has infinity $r$-constant hypersurface with induced 3-metric $\bm{h} = h_{\alpha\beta} \,\dd y^\alpha \dd y^\beta$, where $h_{\alpha\beta} = g_{\alpha\beta}$ for $y^\alpha \in (t, \theta, \varphi)$.
In BL coordinates, the normal to the surface is the outgoing radial vector $\bm{n}=(\dd r)^\sharp$, while the other vectors form the tangent basis.
By computing the highest order term when $r\to\infty$, the surface element is asymptotically spherically symmetric, given by
\begin{align}
    \label{eq3:radial3SurfaceElement}
    \dd\Sigma_\mu = n_\mu \,\sqrt{\det{\bm{h}}} \,\dd t \,\dd \theta \,\dd \varphi \sim n_\mu \,r^2 \sin\theta \,\dd t \,\dd \theta \,\dd \varphi ~.
\end{align}

We are obiously interested in obtaining an expression relating the flow of energy at infinity and the Maxwell NP scalar. Thus, will be convenient to describe the stress-energy tensor using symmetric tetrad combinations and NP scalars and their conjugates. Much like the Weyl and the Maxwell tensor, this composition is uniquelly defined by Eqs. \eref{eq2:stressenergyEM} and \eref{eq3:maxwellNPphi},
\begin{align}
    \begin{split}
    2 T_{\mu\nu} &= \phi_0^* \phi_0 \, \mathfrak{n}_\mu \mathfrak{n}_\nu + \phi_2^* \phi_2 \, \mathfrak{l}_\mu \mathfrak{l}_\nu + 2 \phi_1^* \phi_1 \, [ \mathfrak{l}_{(\mu} \mathfrak{n}_{\nu)} + \mathfrak{m}_{(\mu} \bar{\mathfrak{m}}_{\nu)} ] \\
    &\qquad - 4 \phi_0^* \phi_1 \, \mathfrak{n}_\mu \mathfrak{m}_\nu - 4 \phi_1^* \phi_2 \, \mathfrak{l}_\mu \mathfrak{m}_\nu + 2 \phi_0^* \phi_2 \, \mathfrak{m}_\mu \mathfrak{m}_\nu + \text{c.c.}
    \end{split}
    \label{eq3:stressEnergyPhi}
\end{align}
Energy flow is thus computed by taking the series expansion of
\begin{align}
    r^2 \,T^{r}{}_t = - r^2 \frac{\Delta}{\rho^2} \,T_{rt} = - r^2 \left(\frac{1}{4} |\phi_0|^2 - |\phi_2|^2\right) + \mathscr{O}\left(\frac{1}{r}\right) ~,
\end{align}
recalling definition \eref{eq3:phiBarRhoToPhi} when considering the asymptotic form of ${}_{s}R_{\ell m}$ in \eqref{eq3:asymptoticR}.
We can clearly identify the ingoing and outgoing flows as
\begin{align}
    \frac{\dd^2 E_\mathrm{in}}{\dd t \dd\Omega} = \lim_{r\to\infty} \,\frac{r^2}{4} |\phi_0|^2 ~,\qquad \frac{\dd^2 E_\mathrm{out}}{\dd t \dd\Omega} = \lim_{r\to\infty} \, r^2 |\phi_2|^2 ~,
\end{align}

In order to obtain the full conservation law we must find the absorbed radiation by the BH. We turn now to the horizon null hypersurface, for which the normal vector $n$ in BL coordinates is the zero vector, due to $g^{rr}=0$.
Similarly, the stress-energy tensor in this tetrad basis is ill-defined since $\bm{\mathfrak{l}}$ is singular at the horizon, where $\Delta=0$.
The Kinnersley tetrad keeps its properties, by applying a boost in the null directions
\begin{align}
   \bm{\tilde{\mathfrak{l}}} = \frac{\Delta}{2 (r^2+a^2)} \bm{\mathfrak{l}} ~,\qquad \bm{\tilde{\mathfrak{n}}} = \frac{2 (r^2+a^2)}{\Delta} \bm{\mathfrak{n}} ~,
\end{align}
while removing the singularity at the horizon. The NP field quantities are now given by $\tilde{\Upsilon}_s = [\Delta/2(r^2+a^2)]^s \,\Upsilon_s$.
In addition, we shall use the ingoing EF coordinates, defined in \eqref{eq2:InEFtoBL}, as the chart is the indicated to consider inward future directed waves, because $(\bm{\mathfrak{l}}\cdot\partial_v)$ is a positive constant,
\begin{align}
    \bm{\tilde{\mathfrak{l}}} = \left(1, \,\frac{\Delta}{2(r^2+a^2)}, \,0, \,\frac{a}{r^2 + a^2} \right) ~,\qquad \bm{\tilde{\mathfrak{n}}} = \left(0, -\frac{r^2+a^2}{\rho^2}, \,0, \,0\right) ~.
\end{align}
If we set $r=r_{+}$, we obtain that $\bm{\tilde{\mathfrak{l}}} = \bm{\xi}$.
This implies that $\bm{\tilde{\mathfrak{l}}}$ is the normal vector to the event horizon, just like $\bm{n}$, but they are opposite to each other as $n^v = g^{rv}<0$.

The radial 3-surface element cannot be of the form in \eqref{eq3:radial3SurfaceElement}, since the induced metric at the horizon is now singular, $\sqrt{\det{\bm{h}}}=\Delta \rho^2 \sin\theta = 0$.
Special considerations must be taken when taking the induced metric of a null hypersurface.
We usually choose $\bm{k}=\partial_v$ as one of the surface tangent vectors, due to $\bm{\xi}\cdot\bm{k}=0$.
Then we compute the induced metric $\bm{\sigma}$ of a 2-surface space spanned by the vectors $\partial_\theta$ and $\partial_\chi$.
The general 3-surface element for a null horizon, normal to the inward radial direction, is given by
\begin{align}
    \dd\Sigma_\mu = \tilde{\mathfrak{l}}_\mu \,(\sqrt{\det \bm{\sigma}} \,\dd\theta \,\dd\chi ) \,\dd v = \tilde{\mathfrak{l}}_\mu  \, 2 M r_+ \sin\theta \,\dd\theta \,\dd\varphi \,\dd t ~,
\end{align}
where $\det\sigma = g_{\theta\theta} \,g_{\chi\chi} - (g_{\theta\chi})^2$. In the last equality we used the fact that the jacobian $\partial(v,\chi)/\partial(t,\varphi)=1$.
The resultant energy flux going inside the BH is then computed by
\begin{align}
    \frac{\dd^2 E_\mathrm{hole}}{\dd t \dd\Omega} = 2 M r_{+} T_{\mu\nu} \tilde{\mathfrak{l}}^\mu k^\nu \qquad (r=r_{+})~.
\end{align}
Generalizing \eqref{eq3:fluxE}, we may define the the flow of angular momentum using the axisymmetic Killing vector, $\dd L = - T^{\mu}{}_\nu \, m^\nu \,\dd\Sigma_\mu$, and combining previous results to write
\begin{align}
    \frac{\dd^2 E_\mathrm{hole}}{\dd t \dd\Omega} - \Omega_H \frac{\dd^2 L_\mathrm{hole}}{\dd t \dd\Omega} = 2 M r_{+} T_{\mu\nu} \,\tilde{\mathfrak{l}}^\mu \tilde{\mathfrak{l}}^\nu \qquad (r=r_{+})~.
\end{align}

The computation of the flow of the energy into the BH requires finding the ratio between the energy and angular momentum carried by waves.
For a scalar wave $\Phi\sim e^{-i\omega t+ i m\varphi}$, we can easily find the ratio by computing $\dd L/\dd E = -T^r{}_\varphi/T^r{}_t = - \partial_\varphi \Phi / \partial_t \Phi = m/\omega$, using the standard scalar energy-stress tensor \cite{Bekenstein1973}.
Another simpler argument was made in \eref{eq2:spinMassRatio}, obtaining the same result.
Since this ratio holds for any type of perturbation \cite{Teukolsky1974},
\begin{align}
    \frac{\dd^2 E_\mathrm{hole}}{\dd t \dd\Omega} = \frac{2 M r_{+} \omega}{\omega - m \Omega_H}  \tilde{\phi}_0 \tilde{\phi}_0^* = \frac{\omega}{8 M r_{+} \kappa} |\Delta \phi_0 |^2 \qquad (r=r_{+})~.
\end{align}
Double projection of the future-directed inward vector $\bm{\tilde{\mathfrak{l}}}$ onto the energy-stress tensor gives us $|\tilde{\phi}_0|^2$, due to the decomposition \eref{eq3:stressEnergyPhi}.
At the horizon, the boosted NP scalar can be written as $\tilde{\phi}_0 = (\Delta \phi_0)/(4 M r_{+})$, where $\Delta \phi_0$ is regular at the horizon by construction and also by checking with the boundary solution \eref{eq3:boundaryR}.

Now, we are prepared to define the amplification factor as 
\begin{align}
    \label{eq3:Zdef}
    \uu[s]{Z}{\ell m} = \frac{\dd E_\mathrm{out} / \dd t}{\dd E_\mathrm{in} / \dd t} - 1 ~,
\end{align}
where we integrated over the polar angles.
The factor is defined as the overall gain/loss effect for each mode $(\omega,\ell,m)$, therefore it measures how much of the wave was \emph{globally} reflected ($\uu[s]{Z}{\ell m}=0$), absorbed ($\uu[s]{Z}{\ell m}<0$) or amplified ($\uu[s]{Z}{\ell m}>0$).
Assuming a single mode decomposition and remebering that $\phi_2 = \Phi_2/(2 \bar{\rho}^2) \sim {}_{-1}R_{\ell m}/(2 r^2)$, we show that
\begin{align}
    \label{eq3:ZfactorAout2Ain0}
    \uu[\pm1]{Z}{\ell m} +1= \left| \frac{\lim_{r\to\infty} (\tfrac{1}{r} \; {}_{-1}R_{\ell m} )}{\lim_{r\to\infty} ( r \; {}_{+1}R_{\ell m} )} \right|^2 = \left| \frac{A_\mathrm{out}(s=-1)}{A_\mathrm{in}(s=+1)} \right|^2 \equiv \left| \frac{\mathscr{Z}_\mathrm{out}}{\mathscr{Y}_\mathrm{in}} \right|^2 ~,
\end{align}
Naturally, in order to give use to the $D/C$ ratio from the matching of coefficients, we need to obtain the ratio $A_\mathrm{out}/A_\mathrm{in}$ for the same spin weight. We redefine these constants in \tref{tb3:approximatedRsolutionsYZ}.
\begin{table}[h]
	\centering
	\tabulinesep=1.5mm
    \begin{tabu}{@{\hskip 0.25cm}c@{\hskip 0.75cm}c@{\hskip 0.75cm}c@{\hskip 0.25cm}}
        \hline
         & $\omega r \gg 1$ & $\omega r \ll 1$ \\
		\hline\hline
        $\uu[1]{R}{\ell m}$ & $\mathscr{Y}_\mathrm{in}\,\cfrac{e^{-i \omega r_*}}{r} + \mathscr{Y}_\mathrm{out}\, \cfrac{e^{i \omega r_*}}{r^3}$ & $\mathscr{Y}_\mathrm{hole} \,\Delta^{-1} e^{-i \kappa r_{*}}$  \\
		\hline
        $\uu[-1]{R}{\ell m}$ & $\mathscr{Z}_\mathrm{in}\,\cfrac{e^{-i \omega r_*}}{r} + \mathscr{Z}_\mathrm{out}\, r \,e^{i \omega r_*}$ & $\mathscr{Z}_\mathrm{hole} \,\Delta \,e^{-i \kappa r_{*}}$ \\
        \hline
    \end{tabu}
    \caption{Solutions near horzion far horizon \cite{Teukolsky1974}}
    \label{tb3:approximatedRsolutionsYZ}
\end{table}
Using \eqref{eq3:separationB}, we can relate the \emph{ingoing} integration constants from both $\phi_0$ and $\phi_2$.
In the large $r$ limit, this equation is simplified into $(\partial_r-i \omega)(\partial_r-i \omega) \,{}_{-1}R_{\ell m} \sim \mathscr{B} \,( {}_{+1}R_{\ell m} )$, and considering terms only up to $\mathscr{O}(\tfrac{1}{r})$ we substitute $\mathscr{Y}_\mathrm{in}$,
\begin{align}
    \label{eq3:amplificationBAoutAin}
    \uu[\pm1]{Z}{\ell m} +1 = \frac{\mathscr{B}^2}{16 \omega^4} \left| \frac{\mathscr{Z}_\mathrm{out}}{\mathscr{Z}_\mathrm{in} } \right|^2 ~,
\end{align}
where now the expression is only using $s=-1$ coefficients.
Still, the amplification should not depend on the sign of spin-weight, \emph{i.e.} the amplification should be the same for all EM waves, the same holding true for GW perturbations.
The $\emph{in-out}$ ratio is given by
\begin{align}
    \begin{split}
        \frac{\mathscr{B}^2}{16 \omega^4} \left| \frac{\mathscr{Z}_\mathrm{out}}{\mathscr{Z}_\mathrm{in} } \right|^2 &= 
        \frac{\mathscr{B}^2}{\ell^2(\ell+1)^2} 
        \left|\cfrac{ 
            1+ \frac{(-1)^{\ell+1}}{2} \frac{D}{C} 
            \frac{\Gamma(\ell) \Gamma(\ell+2)}{\Gamma(2\ell+1) \Gamma(2\ell+2)} 
            (2 i\bar{\omega})^{2\ell+1}
            }{
            1 - \frac{(-1)^{\ell+1}}{2} \frac{D}{C}
            \frac{\Gamma(\ell) \Gamma(\ell+2)}{\Gamma(2\ell+1) \Gamma(2\ell+2)}
            (2 i\bar{\omega})^{2\ell+1}
        }\right|^2 \\[0.15cm]
        & \simeq  1 - (2 \bar{\omega})^{2\ell+1}
        \frac{\Gamma(\ell) \Gamma(\ell+2)}{\Gamma(2\ell+1) \Gamma(2\ell+2)} \,\Re\left\{2 i \,\frac{D}{C}\right\} ~,
    \end{split}
\end{align}
where we approximated the relative normalization $\mathscr{B} = \lambdabar + \mathscr{O}(a\omega) = \ell(\ell+1) + \mathscr{O}(a\omega)$, by considering a small deviation from spherical symmetry.

The $D/C$ ratio has a non-trivial expression in terms of $\Gamma$ functions of non-integers arguments. The factorial property of $\Gamma$ allows the approximation ($y = - 2 i \varpi/\tau$)
\begin{align}
    \frac{\Gamma(\ell+1-s + y)}{\Gamma(-\ell - s + y)} = \prod_{n=-\ell-s}^{\ell-s} (n + y) \stackrel[(s=\pm 1)]{}{\simeq} (-1)^{\ell+1} \frac{\ell+1}{\ell} \, y \,\prod_{n=1}^{\ell} (n^2 - y^2) ~.
\end{align}
Combining all results, the amplification factor for EM perturbations yields
\begin{align}
    \label{eq3:ZfactorEM}
    \uu[\pm1]{Z}{\ell m} \simeq  - 4 \bar{\omega} (\bar{\omega} - m \bar{\Omega}_H) \,(2-\tau)(2\bar{\omega} \tau)^{2\ell} \left[\frac{(\ell-1)! (\ell+1)!}{(2\ell)! (2\ell+1)!}\right]^2 \prod_{n=1}^{\ell} \left(n^2 + \frac{4 \varpi^2}{\tau^2} \right) ~.
\end{align}
A very similar expression can be obtained for GW perturbations (see e.g. \cite{Rosa2016}).
The validity of this result is mainly based on the overlapping of the far-region and near-region solutions.
When the black hole is extremal, $\tau=0$, the factor is regular and proportional to ${}_{\pm1}Z_{\ell m}\propto - (4 \varpi \bar{\omega})^{2\ell+1}$, while the amplification occurs mostly when $\ell=m=1$, due to the dampening of the quickly growing factorials in the denominator of $\uu[\pm1]{Z}{\ell m}$.

We know that the EM fields must be real quantities, therefore physical waves must also include negative valued $\omega$ in the mode decomposition \eref{eq3:multipoleExpansion}.
The amplification factor explicity demonstrates that superradiance occurs when
\begin{align}
    \label{eq3:superradiance}
    \omega(\omega-m\Omega_H)<0 ~.
\end{align}
For $\omega>0$, amplification occurs for $m>0$ modes, in the region \eref{eq1:superradiance}, while for $\omega<0$, only modes with $m<0$ can be amplified.
The circular symmetry of the spacetime garantees that superradiance phenomena is invariant under the change of $(\omega,m)\to(-\omega,-m)$.
In other other words,
\begin{align}
    {}_{s}Z_{\ell,-m}(-\omega) = {}_{s}Z_{\ell m}(\omega) ~,
\end{align}
which is clear from the EM case in \eref{eq3:ZfactorEM}.

%----------------------------------------------------------------------------------------

\cleardoublepage