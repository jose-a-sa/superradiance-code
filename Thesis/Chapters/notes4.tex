Therefore to compute the amplification factor we use
\begin{align}
	\uu[\pm 1]{Z}{\ell m} = \frac{\mathscr{B}^2 \tau^4 }{4 \varpi^2 (\tau^2 + 4 \varpi^2)} \left| \frac{f_{-}(x_\infty)}{f_{+}(x_\infty)} \right|^2 - 1
\end{align}

It might seem that we have the liberty of choosing this normalization at the boundary, but we may relate the solutions $f_{\pm}(x)$ at $x\simeq0$ using \eqref{eq3:separationB},
\begin{align}
	f_{-}(0) = -\frac{\mathscr{B} \tau^2 }{2 \varpi ( 2 \varpi + i \tau) } f_{+}(0) ~,
\end{align}
where $r_{+} \mathscr{D}_0 \simeq \partial_x - i \varpi / \tau x$.

We expect to $|f_\pm(x)|$ to become approximately constant to large $x$, because this
form is written in way that also matches the behavior of the radial function at infinity, $\uu[\pm 1]{R}{\ell m}\sim r^{\mp 1}$.
Comparing \eref{eq4:numericalRansatz} with the asymptotic form at large $x$, we obtain
\begin{align}
	\begin{split}
	Y_\mathrm{in} &= f_{+}(x) \exp\left[ - i \bar{\omega} x - i \bar{\omega} (2-\tau)\log x   -i \frac{\varpi}{\tau} \log x \right] ~, \\
	Z_\mathrm{out} &= f_{-}(x) \exp\left[ + i \bar{\omega} x + i \bar{\omega} (2-\tau)\log x   -i \frac{\varpi}{\tau} \log x \right] ~,
	\end{split}
\end{align}
where we used that $r_{*}/r_{+} = x + (2-\tau)\log x + \mathscr{O}(\tfrac{1}{x})$.




Knowing these ratios we may compute the amplification in three several ways
\begin{align}
	\uu[\pm 1]{Z}{\ell m} = \frac{\mathscr{B}^2 \tau^4}{4 \varpi^2 (\tau^2 + 4 \varpi^2)} \frac{Z_\mathrm{out}}{Y_\mathrm{hole}} - 1 =  
\end{align}

(REDOING TEXT)

Knowing the irregularities of the solution at the horizon, we propose the ansatz
\begin{align}
	\uu[\pm 1]{R}{\ell m} = (r_+)^{\mp 1} \, x^{ \mp 1 - i \varpi / \tau} f_{\pm}(x) ~,
\end{align}
where $f(x)$ is a new function obeys a regular differential equation.

It might seem that we have the liberty of choosing any two normalization at the boundary, but we know that these solution are not independent.
From \eqref{eq3:separationB}, we may relate the solutions $f_{\pm}(x)$ at $x\simeq0$,
\begin{align}
	f_{-}(0) = -\frac{\mathscr{B} \tau^2 }{2 \varpi ( 2 \varpi + i \tau) } f_{+}(0) ~,
\end{align}
where $r_{+} \mathscr{D}_0 \simeq \partial_x - i \varpi / x \tau$.
Being a second order differential equation, we need also to restrain $f_{\pm}'(0)$. Thus we replace the

This particular ansatz is optimal for EM perturbations because $|f_\pm(x)|$ becomes approximately constant to large $x$.
This occurs because this form is written in way that matches the behavior of the radial function at infinity, $\uu[\pm 1]{R}{\ell m}\sim r^{\mp 1}$.
There


We redefine the integration constants in~\tref{tb4:solutionsTeukolskyEq} in order to distinguish both solutions.
\begin{table}[h]
	\centering
	%\renewcommand{\arraystretch}{1.5}
	\tabulinesep=1.5mm
    \begin{tabu}{@{\hskip 0.25cm}c@{\hskip 0.75cm}c@{\hskip 0.75cm}c@{\hskip 0.25cm}}
        \hline
         & $\omega r \gg 1$ & $\omega r \ll 1$ \\
		\hline\hline
        $\uu[1]{R}{\ell m}$ & $Y_\mathrm{in}\,\cfrac{e^{-i \omega r_*}}{r} + Y_\mathrm{out}\, \cfrac{e^{i \omega r_*}}{r^3}$ & $Y_\mathrm{hole} \,\Delta^{-1} e^{-i \kappa r_{*}}$  \\
		\hline
        $\uu[-1]{R}{\ell m}$ & $Z_\mathrm{in}\,\cfrac{e^{-i \omega r_*}}{r} + Z_\mathrm{out}\, r \,e^{i \omega r_*}$ & $Z_\mathrm{hole} \,\Delta \,e^{-i \kappa r_{*}}$ \\
        \hline
    \end{tabu}
    \caption{Solutions near horzion far horizon~\cite{Teukolsky1974}}
    \label{tb4:solutionsTeukolskyEq}
\end{table}
Using Eqs. \eref{eq3:separationB} and \eref{eq3:separationBdagger}, we may approximate $\mathscr{D}_0 \sim \partial_r - i \omega$ ($\omega r \gg 1$) and $\mathscr{D}_0 \simeq \partial_r - i M r_{+} \kappa/(r_{+}-M)(r-r_{+}) $ ($\omega r \ll 1$) to deduce 
\begin{align}
	\begin{split}
		\mathscr{B} Y_\mathrm{in} + 4 \omega^2 Z_\mathrm{in} = 0 ~,  \\
		\omega^2 Y_\mathrm{out} + \mathscr{B} Z_\mathrm{out} = 0 ~,  \\
		\mathscr{B} Y_\mathrm{hole} + 16 i \kappa M^2 r_+^2 \left( - i \kappa + \frac{r_{+} - M}{2 M r_{+}} \right) Z_\mathrm{hole} = 0 ~.
	\end{split}
\end{align}