% !TEX root = ../main.tex

\chapter{Mathematical preliminaries} % Main chapter title
\label{Chapter2}


\section{General Relativity}

General Relativity is the theory of space, time and gravitation developed by Einstein in 1915. 
It introduced a new viewpoint on gravity and it's relation with the fabric of spacetime, a \emph{manifold} that bounded our three spatial dimensions with dimension of time, which was a concept that challenged our deeply ingrained and intuitive notions of nature partially because the mathematical background need to understand the precise formulation of theory was unfamiliar to much of the Physics community at the time.

This formulation corresponds to a field theory which the main object of study is the metric of the manifold, $g=g_{\mu\nu} \dd x^\mu \dd x^\nu$ and inherited diffeomorphism invariance, which was at the core of definition of differential manifolds. Firstly, the theory was left aside because of the numerous complicated coupled nonlinear equations, but the astronomical discovery of compact and highly energetic objects in the 1950s breaded new interest into the somewhat dormant GR, mainly because it was thought that these quasars and compact X-ray sources had suffered some form of gravitational collapse or that strong gravitational fields were present.
Soon after, the modern theory of gravitational collapse was developed and the first solutions of BHs were discovered in the mid-1960s, including the Schwarzschild and Kerr BHs.

The theory of GR can be elegantly described in the form of the Hilbert action
\begin{align}
    S_{H} = \frac{1}{16\pi} \int \dd^4 x \sqrt{-g} \,R ~,
    \label{eq2:actionGR}
\end{align}
where $g=\det(g_{\mu\nu})$ and $R$ corresponds to the Ricci scalar.
Naturally, the first solutions corresponded to pure gravity, usually designated as vacuum solutions, which obey
\begin{align}
    R_{\mu\nu} = 0 ~.
    \label{eq2:vacuumGR}
\end{align}
Despite their simplicity, they enjoy some very fascinating nontrivial properties. 
One of which is the existence of an event horizon, a surface that separates two causally disconnected regions of spacetime.

Particularly, in this work we will also include electromagnetic (massless, neutral) wave interactions, which are described by the Maxwell action
\begin{align}
    S_{EM} = - \frac{1}{4} \int \dd^4 x \sqrt{-g} \,F_{\mu\nu} F^{\mu\nu} ~,
     \label{eq2:actionEM}
\end{align}
where $F_{\mu\nu}$ is the Maxwell tensor. Variation of both actions, \emph{i.e.} $\delta(S_H + S_{EM}) = 0$, give rise to two field equations
\begin{align}
    \nabla_\mu F^{\mu\nu} &= 0 ~, \label{eq2:maxwellEM} \\
    R_{\mu\nu} - \frac{R}{2} g_{\mu\nu} &= 8 \pi T_{\mu\nu} \label{eq2:EM+GR}  ~.
\end{align}
The first equation is just the usual of Maxwell equation in curved spacetime.
The second equation reflects the backreaction of the electromagnetic waves into the geometry through the presence of a energy momentum tensor
\begin{align}
    T_{\mu\nu} = F_{\mu\lambda} F_{\nu}{}^{\lambda} - \frac{1}{4} g_{\mu\nu} F^2  ~.
    \label{eq2:stressenergyEM}
\end{align}
These equation completely describe the system, but they are coupled and nonlinear so will be specializing to perturbation theory, considering the field $A^\mu$ to be small. 
Because the stress-energy tensor is quadratic in the fields, $T_{\mu\nu}\sim\mathcal{O}(A^2)$, then we can ignore the backreaction and the field equations for the metric $g_{\mu\nu}$ reduce to~\eqref{eq2:vacuumGR}.
This is a very good approximation, since near black holes of stellar-mass proportions the gravitational field is considerably strong compared with radiation emitted by nearby sources.
Therefor we need only to focus on the Maxwell equation in a static background.
In order to to able to solve this equations we will resort to the Newman-Penrose formalism, which is suited to study any kind of radiation in curved spacetime.


\section{Kerr black hole}

\textbf{(More on stationary and Kerr).} \lipsum[2] 

\subsection{Spacetime symmetries}

If we represent our spacetime by $(\mathcal{M}, g_{\mu\nu}, \psi)$, then the pullbash $f^*$ of the diffeomorphism $f:\mathcal{M}\rightarrow\mathcal{M}$, would give us the same physical system $(\mathcal{M}, f^* g_{\mu\nu}, f^* \psi)$.
Since diffeomorphisms are just active coordinate transformations, such concept may raise some confusion, as we don't seam to obtain no new information to work with. 
Almost all physics theories are coordinate invariant, as is Newtonian mechanics and Special Relativity, but in such theories there is a preferable coordinate system, while the same does not hold true for GR.
An analogies can be made with the path integral formalism in QFT, where special consideration is taken when summing all field configurations in order to not overcount indistinguishable configurations, as is the case of gauge field theories.
A similar ambiguity can occur in GR, where two apparently different solutions which can be related by a diffeomorphism and are actually ``the same'', so we must be careful when deriving and analyzing any geometries.

Despite the added complexity of Einstein's field equations, it is still possible to find exact nontrivial solutions in a systematic way by considering spacetimes with symmetries with the use of Killing vector fields.
A vector field $\xi$ that obeys
\begin{align}
    \mathcal{L}_\xi  g = 0  
    \label{eq2:killing}
\end{align}
is called a Killing field. Locally, this expression reduces to $\nabla_\mu \xi_\nu + \nabla_\nu \xi_\mu = 0$.

A \emph{stationary} solution implies the existence of a Killing vector $k$ that is asymptotically timelike, $k^2<0$, therefore allows us to normalize our vector such that $k^2 \rightarrow -1$. 
Unlike the case of the static spacetime, a stationary metric does not show invariance under reversal of the time coordinate, which is natural considering a system with angular momentum. 
Futhermore, a solution is also \emph{axisymmetric}, due to the presence of a asymptotically spacelike Killing field $m$ whose integral curves are closed. A solution is stationary and axisymmetric if both symmetries are present, along with commuting fields, $[k , m] = 0$, \emph{i.e.} rotations along with the axis of symmetry commute with time translations. The commutativity of the fields implies the existence of a set of coordinates, $(t,r,\theta,\phi)$, such that
\begin{align}
    k = \frac{\partial}{\partial t} ~, \qquad m = \frac{\partial}{\partial \phi} ~.
    \label{eq2:tPhiKilling}
\end{align}
As for direct implication of this choice of chart, components of the metric stay independent of $(t,\phi)$, in virtue of~\eqref{eq2:killing},
\begin{align}
    (\mathcal{L}_m g)_{\mu\nu} = \frac{\partial g_{\mu\nu}}{\partial \phi} = 0 ~,
    \label{eq2:lieMetricTPhi}
\end{align}
with the same holding true for $k$, hence we can write $g_{\mu\nu} = g_{\mu\nu}(r,\theta)$. 

One of the major applications of Killing vectors is to find conserved charges associated with the motion along a geodesic spanned by field.
These quantities are defined by taking the geodesics to regions space that are asymptotical flat, where the geometry does not affect the observer.
In the case of Kerr solution, we have two Killing vectors, $k$ and $m$, which are naturally associated with the total mass $M$ and angular momentum $J$ of the BH, respectively.
This is usually done by evaluating the Komar integrals~\cite{Heusler1996, Wald2010}, which can be written a covariant way as
\begin{alignat}{6}
    M = &&\, -\frac{1}{8 \pi} & \int_{S^2_\infty} \star \dd k^\flat \,& = &&\, \frac{1}{4} & \lim_{r\to\infty}  \int_0^\pi \dd\theta \sqrt{-g} \, g^{t\alpha} g^{r\beta} g_{t[\alpha,\beta]} ~, \label{eq2:komarMass} \\
    J = &&\, \frac{1}{16 \pi} & \int_{S^2_\infty} \star \dd m^\flat \,& = &&\, - \frac{1}{8} & \lim_{r\to\infty}  \int_0^\pi \dd\theta \sqrt{-g} \, g^{t\alpha} g^{r\beta} g_{\phi[\alpha,\beta]} ~,
\end{alignat}
where the usual notation $k^\flat = g(k, \,\cdot\,) = g_{\mu\nu} k^\mu \dd x^\nu$ transforms a vector into a 1-form and $\star : \Omega^{p}(\mathcal{M})\to\Omega^{n-k}(\mathcal{M})$ is the Hodge dual map. In order to complete the integration in the last step is assumed (\ref{eq2:tPhiKilling}) and (\ref{eq2:lieMetricTPhi}). According to the widely accepted of \emph{no-hair conjecture}~\cite{Carter1971}, these two quantities completely define a stationary (neutral) BH. 


\subsection{Metric and ring singularity}

In little assumption, we were able to fix some conditions for the family stationary rotating BH solutions. 

\subsection{Ergoregion and the Penrose process}

\section{Newman-Penrose formalism}

\subsection{Kinnersly tetrad}
\subsection{Spin coefficients}
\subsection{Maxwell equations}


\cleardoublepage