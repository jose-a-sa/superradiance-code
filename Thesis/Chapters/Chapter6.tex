% !TEX root = ../main.tex

\chapter{Discussion and future work} % Main chapter title
\label{Chapter6}

In this work we sought to understand the effect of superradiance scattering of EM waves in the Kerr spacetime.
The objective was to demonstrate if superradiance occurred in the case of scattering of an EM wave radiated from a physically realistic source by a rotating BH.
General waves are a superpositions of modes ($\omega, \ell, m$), for which we know superradiance occurs when $\omega(\omega - m \Omega_H)<0$.
In this region the modes are either reflected or amplified, with the maximum amplification in the case of EM waves being of approximately 4.4\%.
This occurs when the BH is extremal, $a\to M$ and on the lowest multipole $\ell=1$, with the percentage dropping quickly to zero as $\ell$ increases.
Modes with large $|\omega|$ are quickly absorbed by the BH since they can ``cross'' the centrifugal barrier in the effective potential, reaching the event horizon.
To compute these amplification factors we need: (i) to compute the angular eigenvalues that enter the radial equation; (ii) to obtain the coefficients $\mathscr{Z}_\mathrm{in}$ and $\mathscr{Z}_\mathrm{out}$, by solving the radial equation.
In second step we devised a way of not rewriting the radial equation using the tortoise coordinate $r_*$, removing the singularities by considering a clever ansatz, without sacrificing precision or computational speed.
We showed that it is possible to find the amplification factor of each mode either using only $|\mathscr{Z}_\mathrm{in}|^2$ or $|\mathscr{Z}_\mathrm{out}|^2$, or a combination of both coefficients.
In this work we discuss why it is more advantageous to write the gain/loss factor using only one of the previous coefficients.
Although the routine is defined for EM perturbations, it can be quickly updated to accommodate GW perturbations for future work studies.
The resultant complex coefficients also play an integral part in the computation of phase shifts, which are present for each mode.
Particularly, for a plane wave with superradiant frequency we know that most of the multipole modes are deflected with no change in the amplitude.
This effect is characteristic of long-ranged potentials that fall as $1/r$.
When observing the BH from a particular direction, the effect of these phase shifts will dominate and conceal the effects of superradiance.
In principle we could remove these interference effects, integrating over all the solid angle if we could gather information about the EM wave scattering in all directions, leaving only global amplification/absorption effects, but we do not know if that will be ever possible.
Therefore we need to find a way to isolate the lower superradiant modes. A possible idea would be use a source orbiting the BH, with possibility for variation of distance to the BH and incidence angle, due to the chaotic orbits of the Kerr geometry.

In summary, in this thesis we have developed a computational routine that numerically yields the outcome of scattering of any EM wave mode in the Kerr spacetime, including both amplification/absorption factors and phase shifts. 
This code can be used to determine the scattered wave corresponding to any realistic incident wave if its mode decomposition is known, as we illustrated for the case of a plane wave produced by a distant precessing magnetic dipole.
The tools developed in this work will thus play a key role in future studies of superradiant scattering off astrophysical black holes, which may potentially yield an important probe of general relativity in the strong gravity regime.