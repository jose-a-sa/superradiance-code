% !TEX root = ../main.tex

\chapter{Discussion and future work} % Main chapter title
\label{Chapter6}

In this work we soughed to understand the effect of superradiance scattering in EM waves on Kerr BHs.
The objective was to demonstrate if superradiance occurred in case of scattering of a EM wave radiated from a physically realistic source by a rotating BH.
General waves are a superpositions of modes ($\omega, \ell, m$), which we know that occurs superradiance when $\omega(\omega - m \Omega_H)<0$.
In this region the modes are either reflected or amplified, with the maximum amplification in the case of EM waves being of approximately 4.4\%.
This occurs when the BH is extremal, $a\to M$ and on the lowest multipole $\ell=1$, with the percentage dropping quickly to zero as $\ell$ increases.
Modes with large $|\omega|$ are quickly absorbed by the BH since they can ``cross'' the gravitational potential barrier, reaching the event horizon.
To compute these amplification factor we need: (i) to compute the angular eigenvalues in order to substitute into the radial equation; (ii) obtain the coefficients of $\mathscr{Z}_\mathrm{in}$ and $\mathscr{Z}_\mathrm{out}$, by solving the radial equation.
The second step we devised way of not rewriting the radial equation using the tortoise coordinate $r_*$, removing the singularities by considering a cleaver ansatz, without sacrificing precision or computational speed.
We showed that is possible to find the amplification factor of each mode either using $|\mathscr{Z}_\mathrm{in}|^2$ or $|\mathscr{Z}_\mathrm{out}|^2$ our both.
In this work we discuss why it is more advantageous to write the gain/loss factor only using only one of the previous coefficients.
Although the routine is defined for EM perturbations, it can be quickly updated to accommodate GW perturbations for future work studies.
The resultant complex coefficients also play an integral part in the computation of phase shifts, which are present in for each mode.
Particularly, for a plane wave with superradiant frequency we know that most of the multipole modes are deflected with no change in the amplitude.
This effect is characteristic of long-ranged potential that fall as $1/r$.
When observing the BH from a particular direction, the effect of these phase shifts will dominate and conceal the effects of superradiance.
In principle we could remove these interference effects, integrate all the solid angle by taking making observations from all directions, leaving only global amplification/absorption effects, but we do not know if that will be ever possible.
Therefore we need to find a way to isolate the lower superradiant modes. A possible idea would be use a source orbiting the BH, with possibility for variation of distance to the BH and incidence angle, due to the chaotic orbits of the Kerr geometry. 