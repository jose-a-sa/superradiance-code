\section{Klein paradox as a first example} 

\subsection{Fermions}

For a minimally coupled electromagnetic potential, the usual partial derivative in the Dirac equation becomes $D_\mu = \partial_\mu + i e A_\mu$ in order to preserve gauge invariance of the theory. Thus
\begin{align}
    ( i \gamma^\mu D_\mu - m ) \Psi = 0
    \label{eq:dirac}
\end{align}
where $m$ is the fermion mass.
The problem is greatly simplified by considering flat space-time in (1+1)-dimentions, for which a valid representation of the gamma matrices is
\begin{align}
    \gamma^0 = \left(\begin{array}{cr} 1 & 0 \\  0 & -1 \end{array}\right) \qquad 
    \gamma^1 = \left(\begin{array}{cr} 0 & 1 \\ -1 & 0 \end{array}\right)
    \label{eq:gamma1+1}
\end{align}

Following chronologically, Klein~\cite{Klein1929} used Dirac equation to study electrons in a step potential $A(x) = V\,\theta(x) ~\dd t$, for $V>0$ constant and plane wave solutions $\Psi= e^{-i E t} \psi$. For $x<0$, the solution can be divided as incident and reflected, taking the form
\begin{align}
    \psi_\mathrm{inc}(x) = \mathcal{I}\, e^{i k x} \left(\begin{array}{c} 1 \\ \cfrac{k}{E+m} \end{array}\right) \qquad
    \psi_\mathrm{refl}(x) = \mathcal{R}\, e^{- i k x} \left(\begin{array}{c} 1 \\ \cfrac{-k}{E+m} \end{array}\right)
\end{align}
while for $x>0$, the transmitted wave function is written as 
\begin{align}
    \psi_\mathrm{trans}(x) = \mathcal{T}\, e^{i q x} \left(\begin{array}{c} 1 \\ \cfrac{q}{ E - e V + m} \end{array}\right)
\end{align}
where $q = [(E-e V)^2 - m^2]^{1/2}$, by solving the eigenvalue problem. 
Writing the continuity condition for the complete solution at the barrier $x=0$, determines the coefficients
\begin{align}
    \mathcal{I} + \mathcal{R} &= \mathcal{T} \\
    \mathcal{I} - \mathcal{R} &= r\, \mathcal{T}
\end{align}
with
\begin{align}
    r = \frac{q}{k}\,\frac{E+m}{E-e V+m}
\end{align}