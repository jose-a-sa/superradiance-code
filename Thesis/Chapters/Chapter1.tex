% !TEX root = ../main.tex

\chapter{Superradiance} % Main chapter title
\label{Chapter1} 

%----------------------------------------------------------------------------------------

\section{Introduction}

The first appearance of the concept of \emph{supperradiance} was in 1954, when Dicke~\cite{Dicke1954} showed that a gas could be excited by a pulse into ``superradiant states'' from thermal equilibrium and then emit coherent radiation. 

Actually, radiation amplification can be traced to birth of Quantum Mechanics, in the beginnings of the 20th century. 
First studies of the Dirac equation by Klein~\cite{Klein1929} revealed the possibility of electrons propagating in a region with a sufficiently large potential barrier without the expected dampening from non-relativistic QM tunnel effect.
Due to some confusion, this result was wrongly interpreted by some authors as fermionic superradiance, as if the reflected current by the barrier could be greater than the incident current. 
The problem was named \emph{Klein paradox} by Sauter~\cite{Sauter1931} and this misleading result was due to a incorrect calculation of the group velocities of the reflected and transmitted waves. 
Today, it is known that fermionic currents cannot be amplified for this particular problem~\cite{Manogue1988, Brito2015}, which was correctly calculated in Klein's original paper. 
On the contrary, superradiant scattering could indeed occur for bosonic fields. 

Further calculations from Zel'dovich~\cite{Zeldovich1971,Zeldovich1972} showed that a absorbing surface rotating with an angular velocity $\Omega$ could scatter incident wave with frequency $\omega$ which satisfies
\begin{align}
    \omega - m \Omega < 0
    \label{eq:superradiant-condition}
\end{align}
where $m$ is the usual azimuthal number of the monochromatic plane wave relative to the rotation axis. 
Condition~\eqref{eq:superradiant-condition} was to become one of the most important results of (rotational) superradiance as it presents itself in multiple examples in the literature, \emph{i.e.} Vavilov-Cherekov effect and anomalous Doppler effect.

%----------------------------------------------------------------------------------------

\section{Klein paradox as a first example} 

\subsection{Fermions}

For a minimally coupled electromagnetic potential, the usual partial derivative in the Dirac equation becomes $D_\mu = \partial_\mu + i e A_\mu$ in order to preserve gauge invariance of the theory. Thus
\begin{align}
    ( i \gamma^\mu D_\mu - m ) \Psi = 0
    \label{eq:dirac}
\end{align}
where $m$ is the fermion mass.
The problem is greatly simplified by considering flat space-time in (1+1)-dimentions, for which a valid representation of the gamma matrices is
\begin{align}
    \gamma^0 = \left(\begin{array}{cr} 1 & 0 \\  0 & -1 \end{array}\right) \qquad 
    \gamma^1 = \left(\begin{array}{cr} 0 & 1 \\ -1 & 0 \end{array}\right)
    \label{eq:gamma1+1}
\end{align}

Following chronologically, Klein~\cite{Klein1929} used Dirac equation to study electrons in a step potential $A(x) = V\,\theta(x) ~\dd t$, for $V>0$ constant and plane wave solutions $\Psi= e^{-i E t} \psi$. For $x<0$, the solution can be divided as incident and reflected, taking the form
\begin{align}
    \psi_\mathrm{inc}(x) = \mathcal{I}\, e^{i k x} \left(\begin{array}{c} 1 \\ \cfrac{k}{E+m} \end{array}\right) \qquad
    \psi_\mathrm{refl}(x) = \mathcal{R}\, e^{- i k x} \left(\begin{array}{c} 1 \\ \cfrac{-k}{E+m} \end{array}\right)
\end{align}
while for $x>0$, the transmitted wave function is written as 
\begin{align}
    \psi_\mathrm{trans}(x) = \mathcal{T}\, e^{i q x} \left(\begin{array}{c} 1 \\ \cfrac{q}{ E - e V + m} \end{array}\right)
\end{align}
where $q = [(E-e V)^2 - m^2]^{1/2}$, by solving the eigenvalue problem. 
Writing the continuity condition for the complete solution at the barrier $x=0$, determines the coefficients
\begin{align}
    \mathcal{I} + \mathcal{R} &= \mathcal{T} \\
    \mathcal{I} - \mathcal{R} &= r\, \mathcal{T}
\end{align}
with
\begin{align}
    r = \frac{q}{k}\,\frac{E+m}{E-e V+m}
\end{align}




\subsection{Bosons}

Klein and Gordon develop their equation in which describes scalar fields


%----------------------------------------------------------------------------------------

\section{Black hole superradiance}


%----------------------------------------------------------------------------------------


\cleardoublepage
