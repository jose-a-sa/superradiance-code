% !TEX root = ../main.tex

\chapter{Superradiance} % Main chapter title
\label{Chapter1}

%----------------------------------------------------------------------------------------

\section{Introduction}

What is superradiance ? 

Why superradiance in Kerr ? Gravitational waves \ldots \\

Historically, the first appearance of the concept of \emph{superradiance} was in 1954, in a publication by Dicke~\cite{Dicke1954}, and it is defined as the assemble of processes which result in amplified radiation. In particular, he showed that a gas could be excited by a pulse into ``superradiant states'' from thermal equilibrium and then emit coherent radiation. Almost two decades later, Zel'dovich~\cite{Zeldovich1971,Zeldovich1972} showed that a absorbing cylinder rotating with an angular velocity $\Omega$ could scatter incident wave with frequency $\omega$ if
\begin{align}
    \omega < m\, \Omega
    \label{eq:superradiance}
\end{align}
would be satisfied, where $m$ is the usual azimuthal number of the monochromatic plane wave relative to the rotation axis.
In his study, he observed that superradiance was associated with dissipation of rotational energy from the absorbing object, possibly due to spontaneous pair creation at the surface. Condition~\eqref{eq:superradiance} was to become one of the most important results of rotational superradiance, as it presented itself in multiple examples, including in black hole (BH) physics, particularly in the case of the Kerr~\cite{Kerr19XX} solution. 
Furthermore, attempts of quantising (scalar/fermionic/\ldots) fields in the Kerr geometry by Starovinsky and others, as well as thermodynamic analysis of the problem, laid seminal grounds to the discovery of BH evaporation by Hawking. 

%----------------------------------------------------------------------------------------

\section{Klein paradox as a first example}

Actually, radiation amplification can be traced to birth of Quantum Mechanics, in the beginnings of the 20th century. 
First studies of the Dirac equation by Klein~\cite{Klein1929} revealed the possibility of electrons propagating in a region with a sufficiently large potential barrier without the expected dampening from non-relativistic QM tunnel effect.
Due to some confusion, this result was wrongly interpreted by some authors as fermionic superradiance, as if the reflected current by the barrier could be greater than the incident current. 
The problem was named \emph{Klein paradox} by Sauter~\cite{Sauter1931} and this misleading result was due to a incorrect calculation of the group velocities of the reflected and transmitted waves. 

Today, it is known that fermionic currents cannot be amplified for this particular problem~\cite{Klein1929,Manogue1988}, result that was correctly obtained by Klein in is original paper. 
On the contrary, superradiant scattering can indeed occur for bosonic fields.

\subsection{Bosons}

The equation that governs bosonic wave function is the Klein-Gordon equation, which for a minimally coupled electromagnetic potential takes the form
\begin{align}
    (D^\mu D_\mu - m^2) \Phi = 0 ~,
    \label{eq:KleinGordon}
\end{align}
where the usual partial derivative becomes $D_\mu = \partial_\mu + i e A_\mu$. 

The problem is greatly simplified by considering flat space-time in (1+1)-dimensions and step potential $A(x) = V\,\theta(x) ~\dd t$, for $V>0$ constant and wave solutions $\Phi= e^{-i \omega t} \phi$.
For $x<0$, the solution can be divided as incident and reflected, taking the form
\begin{align}
    \phi_\mathrm{inc}(x) = \mathcal{I}\, e^{i k x}  ~, \qquad
    \phi_\mathrm{refl}(x) = \mathcal{R}\, e^{- i k x} ~,
    \label{eq:KGsolneg}
\end{align}
in which the dispersion relation states that $k=\sqrt{\omega^2-m^2}$. For $x>0$, the transmitted wave is naturally given by 
\begin{align}
    \psi_\mathrm{inc}(x) = \mathcal{T}\, e^{i q x}  ~,
    \label{eq:KGsolpos}
\end{align}
but in this case the root sign for the momentum must be carefully chosen so that the group velocity sign of the transmitted wave matches of the incoming wave~\cite{Manogue1988}, \textit{i.e.}
\begin{align}
    \left.\frac{\partial \omega}{\partial p}\right|_{p=q} = \frac{q}{\omega - e V} > 0 ~,
    \label{eq:KGphasev}
\end{align}
therefore we must have that
\begin{align}
    q = \mathrm{sgn}(\omega - e V) \sqrt{ (\omega - e V)^2 - m^2 } ~.
    \label{eq:KGq}
\end{align}

After obtaining the continuity relations at the barrier, $x=0$, we follow by computing the ratios of the transmitted and reflected currents relative to the incident one, which yield
\begin{align}
    \frac{j_\mathrm{refl}}{j_\mathrm{inc}} = - \left|\frac{\mathcal{R}}{\mathcal{I}}\right|^2 = -\left|\frac{1-r}{1+r}\right|^2  ~, \qquad \frac{j_\mathrm{trans}}{j_\mathrm{inc}} = \Re(r) \left|\frac{\mathcal{T}}{\mathcal{I}}\right|^2 = \frac{4\, \Re(r)}{|1+r|^2} ~,
    \label{eq:KGcurrents}
\end{align}
written as a function of the coefficient 
\begin{align}
    r = \frac{q}{k} = \mathrm{sgn}(\omega - e V)  \sqrt{\frac{(\omega - e V)^2 - m^2}{\omega^2 - m^2}} ~.
    \label{eq:KGr}
\end{align}
Hence, in the case of strong potential limit, $e V > \omega + m > 2 m$, we may have $r<0$ real and the reflected current is larger (in magnitude) than the incident wave and therefore we have amplification.

\subsection{Fermions}

In the case of fermions, knowing that no spin flip occurs when simply considering a scalar potential~\cite{Itzykson2012}, we need only to consider Dirac equation
\begin{align}
    ( i \gamma^\mu D_\mu - m ) \Psi = 0 ~,
    \label{eq:Dirac}
\end{align}
where $m$ is the fermion mass, for a single spinor component, for which a valid representation of the gamma matrices is
\begin{align}
    \gamma^0 = \left(\begin{array}{cr} 1 & 0 \\  0 & -1 \end{array}\right) ~,\qquad 
    \gamma^1 = \left(\begin{array}{cr} 0 & 1 \\ -1 &  0 \end{array}\right) ~.
    \label{eq:Gamma1+1}
\end{align}

Probing wave solutions $\Psi= e^{-i\omega t} \psi$, the incident and reflected solutions are
\begin{align}
    \psi_\mathrm{inc}(x) = \mathcal{I}\, e^{i k x} \left(\begin{array}{c} 1 \\ \cfrac{k}{\omega+m} \end{array}\right) ~,\qquad
    \psi_\mathrm{refl}(x) = \mathcal{R}\, e^{- i k x} \left(\begin{array}{c} 1 \\ \cfrac{-k}{\omega+m} \end{array}\right) ~,
    \label{eq:Diracsolneg}   
\end{align}
while for $x>0$, the transmitted wave function is written as 
\begin{align}
    \psi_\mathrm{trans}(x) = \mathcal{T}\, e^{i q x} \left(\begin{array}{c} 1 \\ \cfrac{q}{ \omega - e V + m} \end{array}\right) ~,
    \label{eq:Diracsolpos} 
\end{align}
where was followed the same procedure as before, obtaining the same result through ~\ref{eq:KGphasev,eq:KGq,eq:KGcurrents}. As a result of the structure of the spinor components, the coefficient at~\eqref{eq:KGr} is modified to
\begin{align}
    r = \mathrm{sgn}(\omega - e V)  \frac{\omega+m}{\omega-e V+m} \sqrt{\frac{(\omega - e V)^2 - m^2}{\omega^2 - m^2}} ~,
    \label{eq:Diracr} 
\end{align}
and now, in the same region, $\omega > m$, superradiance does not occur.

One may think that this difference between bosons and fermions arises from the potential barrier shape, but Sauter~\cite{Sauter1931} showed that only the difference between the asymptotic values of the potential at infinity was essential for the process. The difference comes from intrinsic properties of these particles. Fermionic current densities are always positive definite while bosons can change sign, as we see in EQ.X for the transmitted part.  

Even though superradiance and spontaneous pair creation are two distinct phenomena, this result is usually interpreted using the latter, from a QFT stand point. All incident particles are fully reflected as well as some extra due to pair creation at the barrier due to a presence of a strong electromagnetic field, while the resultant anti-particles are transmitted in the opposite direction, accounting for the change of sign in the transmitted current, due to the opposite charge they carry. The minimum necessary energy for pair creation, $2m$, leaves evidence that superradiance may be accompanied with this process and some sort of dissipation at potential barrier, in order to maintain energy balance.   


%----------------------------------------------------------------------------------------

\section{Black hole superradiance}


%----------------------------------------------------------------------------------------


\cleardoublepage
