%%%%%%%%%%%%%%%%%%%%%%%%%%%%%%%%%%%%%%%%%
% Masters/Doctoral Thesis
%%%%%%%%%%%%%%%%%%%%%%%%%%%%%%%%%%%%%%%%%

%----------------------------------------------------------------------------------------
%	PACKAGES AND OTHER DOCUMENT CONFIGURATIONS
%----------------------------------------------------------------------------------------

\documentclass[11pt, twoside]{Thesis} % The default font size and one-sided printing (no margin offsets)

\graphicspath{{Figures/}} % Specifies the directory where pictures are stored

\usepackage[square, numbers, comma, sort&compress]{natbib} % Use the natbib reference package - read up on this to edit the reference style; if you want text (e.g. Smith et al., 2012) for the in-text references (instead of numbers), remove 'numbers' 
\bibliographystyle{apsrev4-1-etal} % Use the utphys, apsrev4-1, kp and more BibTeX style for formatting the Bibliography

\hypersetup{colorlinks,citecolor=green,urlcolor=blue} % Colors hyperlinks in blue - change to black if annoying

%%% FONT PACKAGES
% - fourier
% - mathpazo ( palatino for Computer Modern on math )
\usepackage{mathpazo}

\usepackage{pdfpages}

\usepackage{lipsum}

\newcommand*{\defaultlinespacing}{\setstretch{1.5}}
\newcommand*{\listlinespacing}{\setstretch{1.7}}


%----------------------------------------------------------------------------------------
%	DOCUMENT VARIABLES
%----------------------------------------------------------------------------------------
\thesistitle{Superradiance} % Your thesis title \ttitle
\thesistype{Master's Thesis} % Your thesis type Doctoral Thesis or Masters Thesis \ttype
\supervisor[mailto:joao.rosa@fc.up.pt]{João \textsc{Rosa}} % Your supervisor's name \supname \supnamenolink
\cosupervisor[mailto:orfeu.bertolami@fc.up.pt]{Orfeu \textsc{Bertolami}} % Your supervisor's name \cosupname \cosupnamenolink
\degree{Master of Science} % Your degree name \degreename 
\authors[mailto:jose.sa@fc.up.pt]{José \textsc{Sá}} % Your name \authornames \authornamesnolink
\addresses{} % Your address \addressname
\subject{} % Your subject area \subjectname
\keywords{Superradiance} % Keywords for your thesis \keywordnames 
\university[http://www.up.pt]{Universidade do Porto} % Your university's name \univname \univnamenolink
\UNIVERSITY[http://www.up.pt]{UNIVERSIDADE DO PORTO} % Your university's name in capitals \UNIVNAME \UNIVNAMEnolink             
\department[http://dfa.fc.up.pt]{Departamento de Física e Astronomia} % Your department's name \deptname \deptnamenolink
\DEPARTMENT[http://dfa.fc.up.pt]{DEPARTAMENTO DE FÍSICA E ASTRONOMIA} % Your department's name in capitals \DEPTNAME \DEPTNAMEnolink            
%\group{} % Your research group's name \groupname \groupnamenolink            
%\GROUP[]{} % Your research group's name in capitals \GROUPNAME \GROUPNAMEnolink   
\faculty[http://www.fc.up.pt]{Faculdade de Ciências da Universidade do Porto} % Your faculty's name \facname \facnamenolink
\FACULTY[http://www.fc.up.pt]{FACULDADE DE CIÊNCIAS DA UNIVERSIDADE DO PORTO} % Your faculty's name in capitals \FACNAME \FACNAMEnolink         


%----------------------------------------------------------------------------------------
%	NOTATION VARIABLES
%----------------------------------------------------------------------------------------
\newcommand{\dd}{\mathrm{d}}
\newcommand{\sts}[3][]{ {}_{#1} #2_{#3} }


%----------------------------------------------------------------------------------------
%	DOCUMENT
%----------------------------------------------------------------------------------------
\begin{document}

\frontmatter % Use roman page numbering style (i, ii, iii, iv...) for the pre-content pages

\defaultlinespacing % Line spacing of 1.5


%----------------------------------------------------------------------------------------
%	TITLE PAGE
%----------------------------------------------------------------------------------------

\pagestyle{empty}
\includepdf[page={1},pagecommand={},scale=1]{Front/fcup}
\cleardoublepage
\includepdf[page={2},pagecommand={},scale=1]{Front/fcup}
\cleardoublepage

\maketitle


%----------------------------------------------------------------------------------------
%	QUOTATION PAGE
%----------------------------------------------------------------------------------------

\quotepage{Palavra do Dominum}
{
	Naquele tempo, depois da lição Parental, Dominum falou aos seus projectos de apóstolos das criaturas que vaguavam: ``\emph{As tipas andam de relationship em relationship. Aquilo é falta de Pai}''. Um dos apóstolos, de seu nome Pedro, perguntou-lhe ``\emph{Pai? Como assim, Pai?}'', ao que Dominum lhe respondeu: ``\emph{Pai? Pai há só um!}''. E, com isto, Dominum revelou-se aos seus projectos de apóstolos como Pai de todos, e estes prosseguiram a sua vida felizes com o que lhes foi dito.~\cite{Rosa2017}
}


%----------------------------------------------------------------------------------------
%	ACKNOWLEDGEMENTS
%----------------------------------------------------------------------------------------

\acknowledgements{\addtocontents{toc}{\vspace{1em}} % Add a gap in the Contents, for aesthetics
	
	The acknowledgements and the people to thank go here, don't forget to include your project advisors\ldots

}


%----------------------------------------------------------------------------------------
%	ABSTRACT PAGE
%----------------------------------------------------------------------------------------

\addtotoc{Abstract} % Add the "Abstract" page entry to the Contents

\abstract
{
	\addtocontents{toc}{\vspace{1em}} % Add a gap in the Contents, for aesthetics
	
	The Thesis Abstract is written here (and usually kept to just this page). The page is kept centered vertically so can expand into the blank space above the title too\ldots
	
	\lipsum[2-3]
}

\cleardoublepage

%----------------------------------------------------------------------------------------
%	ABSTRACT PAGE (PORTUGUESE)
%----------------------------------------------------------------------------------------

\addtotoc{Resumo}

\abstract[
	thesistitle={Nome da tese em português},
	title={Resumo},
	degree={Mestre de Ciência},
	nameconnector={por}]
{	
	\addtocontents{toc}{\vspace{1em}} % Add a gap in the Contents, for aesthetics
	
	Tradução em português do ``Abstract'' escrito em inglês mais a cima.~\cite{Teukolsky1974} A página é centrada vertical  e horizontalmente, podendo espandir para o espaço superior da página em branco \ldots
	
	\lipsum[2-3]
}

\cleardoublepage

%----------------------------------------------------------------------------------------
%	LIST OF CONTENTS/FIGURES/TABLES PAGES
%----------------------------------------------------------------------------------------

\tableofcontents % Write out the Table of Contents

\listoffigures % Write out the List of Figures

%\listoftables % Write out the List of Tables

\addtocontents{toc}{\vspace{1em}}

%----------------------------------------------------------------------------------------
%	ABBREVIATIONS
%----------------------------------------------------------------------------------------

\listlinespacing % This makes the following tables easier to read

\listofabbreviations{r@{\hskip 0.4in}l} % Include a list of Abbreviations (a table of two columns)
{
	\textbf{GR}		& \textbf{G}eneral \textbf{R}elativity \\
	\textbf{BH}		& \textbf{B}lack \textbf{H}ole \\
	\textbf{SWSH}	& \textbf{S}pin-\textbf{W}eighted \textbf{S}pheroidal \textbf{H}armonic \\
	\hspace*{0.9in} & \hspace*{5in}
}

\defaultlinespacing % Return the line spacing back to 1.3



%----------------------------------------------------------------------------------------
%	PHYSICAL CONSTANTS/OTHER DEFINITIONS
%----------------------------------------------------------------------------------------

% \listlinespacing % This makes the following tables easier to read

% \listofconstants{r@{\hskip 0.4in}lcl} % a four column table
% {
% 	Speed of Light & $c$ & $=$ & $2.997\ 924\ 58\times10^{8}\ \mbox{ms}^{-\mbox{s}}$ (exact)\\
% 	% Constant Name & Symbol & $=$ & Constant Value (with units) \\
% }

% \defaultlinespacing % Return the line spacing back to default


%----------------------------------------------------------------------------------------
%	SYMBOLS
%----------------------------------------------------------------------------------------

% \listlinespacing % This makes the following tables easier to read

% \listofsymbols{l@{\hskip 0.4in}l@{\hskip 0.4in}l} % a three column table
% {
% 	$a$ & distance & m \\
% 	$P$ & power & W (Js$^{-1}$) \\
% 	% Symbol & Name & Unit \\

% 	& & \\ % Gap to separate the Roman symbols from the Greek

% 	$\omega$ & angular frequency & rads$^{-1}$ \\

% 	% Symbol & Name & Unit \\
% }

% \defaultlinespacing % Return the line spacing back to default


%----------------------------------------------------------------------------------------
%	NOTATION
%----------------------------------------------------------------------------------------

\newcommand\notationname{Notation and Conventions}

\addtotoc{\notationname}
\chapter*{\notationname}
\fancyhead[LO]{\textsc{\notationname}}

\section*{Units}

Text random and a new citation

\section*{Tensors and relativity related}

\lipsum[1-10] % Random text

\cleardoublepage


%----------------------------------------------------------------------------------------
%	DEDICATORY
%----------------------------------------------------------------------------------------

% \dedicatory{For/Dedicated to/To my\ldots} % Dedication text

% \addtocontents{toc}{\vspace{2em}} % Add a gap in the Contents, for aesthetics

% \cleardoublepage


%----------------------------------------------------------------------------------------
%	THESIS CONTENT - CHAPTERS
%----------------------------------------------------------------------------------------

\addtocontents{toc}{\vspace{1em}}

\mainmatter % Begin numeric (1,2,3...) page numbering

\pagestyle{fancy}
\renewcommand{\chaptermark}[1]{\markboth{\thechapter. \textsc{#1}}{}}
\fancyhead[LO]{\leftmark}

% !TEX root = ../main.tex

\chapter{Superradiance} % Main chapter title
\label{Chapter1} 

%----------------------------------------------------------------------------------------

\section{Introduction}

The first appearance of the concept of \emph{superradiance} was in 1954, when Dicke~\cite{Dicke1954} showed that a gas could be excited by a pulse into ``superradiant states'' from thermal equilibrium and then emit coherent radiation.

Zel'dovich~\cite{Zeldovich1971,Zeldovich1972} showed that a absorbing surface rotating with an angular velocity $\Omega$ could scatter incident wave with frequency $\omega$ which satisfies
\begin{align}
    \omega - m\, \Omega < 0
    \label{eq:superradiance}
\end{align}
where $m$ is the usual azimuthal number of the monochromatic plane wave relative to the rotation axis. 
Condition~\eqref{eq:superradiance} was to become one of the most important results of (rotational) superradiance as it presents itself in multiple examples in the literature, \emph{i.e.} Vavilov-Cherekov effect and anomalous Doppler effect.

Actually, radiation amplification can be traced to birth of Quantum Mechanics, in the beginnings of the 20th century. 
First studies of the Dirac equation by Klein~\cite{Klein1929} revealed the possibility of electrons propagating in a region with a sufficiently large potential barrier without the expected dampening from non-relativistic QM tunnel effect.
Due to some confusion, this result was wrongly interpreted by some authors as fermionic superradiance, as if the reflected current by the barrier could be greater than the incident current. 
The problem was named \emph{Klein paradox} by Sauter~\cite{Sauter1931} and this misleading result was due to a incorrect calculation of the group velocities of the reflected and transmitted waves. 
Today, it is known that fermionic currents cannot be amplified for this particular problem~\cite{Manogue1988}, result that was correctly obtained by Klein in is original paper. 
On the contrary, superradiant scattering can indeed occur for bosonic fields.

%----------------------------------------------------------------------------------------




%----------------------------------------------------------------------------------------

\section{Black hole superradiance}


%----------------------------------------------------------------------------------------


\cleardoublepage

% !TEX root = ../main.tex

\chapter{Mathematical preliminaries} % Main chapter title
\label{Chapter2}


\section{General Relativity}

General Relativity is the theory of space, time and gravitation developed by Einstein in 1915. 
It introduced a new viewpoint on gravity and it's relation with the fabric of spacetime, a \emph{manifold} that bounded our three spatial dimensions with dimension of time, which was a concept that challenged our deeply ingrained and intuitive notions of nature, partially because the mathematical background needed to understand the precise formulation of theory was unfamiliar to much of the Physics community at the time.

This formulation corresponds to a field theory with the dynamical object of study being the metric of the manifold, $g=g_{\mu\nu} \dd x^\mu \dd x^\nu$, and inherits diffeomorphism invariance, \emph{i.e.} remains the same theory by a active change of coordinates, which was at the core of definition of differential manifolds.

Immediately after, in 1916, Schwarzschild found the first solution.
Then, the theory was left aside because of the numerous coupled nonlinear equations, but the astronomical discovery of compact and highly energetic objects in the 1950s breaded new interest into the somewhat dormant GR, mainly because it was thought that these quasars and compact X-ray sources had suffered some form of gravitational collapse or that strong gravitational fields were present.
Soon after, the modern theory of gravitational collapse was developed in the mid-1960s, including other BHs solutions, including Kerr's.

The theory of GR can be elegantly described in the form of the Hilbert action
\begin{align}
    S_{H} = \frac{1}{16\pi} \int \dd^4 x \sqrt{-g} \,R ~,
    \label{eq2:actionGR}
\end{align}
where $g=\det(g_{\mu\nu})$ and $R$ corresponds to the Ricci scalar.
Naturally, the first solutions corresponded to pure gravity, usually designated as vacuum solutions, which obey
\begin{align}
    R_{\mu\nu} = 0 ~.
    \label{eq2:vacuumGR}
\end{align}
Despite their simplicity, they enjoy some very fascinating nontrivial properties. 
One of which is the existence of an event horizon, a surface that separates two causally disconnected regions of spacetime.

The underlying techniques behind the study of superradiance is the linearization of Einstein and/or Maxwell equations around know BHs in stationary equilibrium.
These perturbation will obey a series of partial equations whose dynamical variables are components of the Weyl tensor, $C_{\mu\nu\rho\sigma}$, or the Maxwell field tensor. 
Thanks to the Newman-Penrose (NP) formalism we will be able to decouple and separate the equations for both GWs and EM waves, revealing decoupled variables which contain all the information needed about the nontrivial perturbations, instead of working with all components of the field tensors.

For the gravitational case, a straightforward way of obtaining a linearized theory is to consider a background stationary BH solution, $g_{\mu\nu}{}^B$, and then expanding the field equations~\eref{eq2:vacuumGR} using the metric $\tilde{g}_{\mu\nu} = g_{\mu\nu}{}^B + h_{\mu\nu}$, keeping only terms with are $\mathcal{O}(h_{\mu\nu})$, which leads to a wave equation in the particular background chosen. 

Particularly, in this work we will focus on (massless, neutral) electromagnetic waves and perturbations are performed including EM interactions through the Maxwell action
\begin{align}
    S_{EM} = - \frac{1}{4} \int \dd^4 x \sqrt{-g} \,F_{\mu\nu} F^{\mu\nu} ~,
     \label{eq2:actionEM}
\end{align}
where $F_{\mu\nu}$ is the Maxwell tensor.
Variation of both actions, $\delta(S_H + S_{EM}) = 0$, result in two field equations
\begin{align}
    \nabla_\mu F^{\mu\nu} &= 0 ~, \label{eq2:maxwellEM} \\
    R_{\mu\nu} - \frac{R}{2} g_{\mu\nu} &= 8 \pi T_{\mu\nu} \label{eq2:EM+GR}  ~.
\end{align}
The first equation is just the usual of Maxwell equation in curved spacetime.
The latter is the Einstein field equation, which reflects the backreaction of the electromagnetic waves into the geometry through the presence of EM stress-energy tensor
\begin{align}
    T_{\mu\nu} = F_{\mu\lambda} F_{\nu}{}^{\lambda} - \frac{1}{4} g_{\mu\nu} F^2  ~.
    \label{eq2:stressenergyEM}
\end{align}
These equation completely describe the system, but the problem is analytically untreatable, so will be resorting to perturbation theory, considering the field $A^\mu$ to be small. 
This is a very good approximation, since near the gravitational field of stellar-mass BHs  is considerably strong compared with radiation emitted by nearby astrophysical sources.
As the stress-energy tensor is quadratic in the fields, $T_{\mu\nu}\sim\mathcal{O}(A^2)$, then we can ignore the backreaction and the field equations for the metric $g_{\mu\nu}$ reduce to~\eqref{eq2:vacuumGR}.


\section{Kerr black hole}

It was generally accepted that a perfectly spherical symmetrical star would collapse to a Schwarzschild BH. 
Although, at the time it was not known the effect of a slightest amount o angular momentum on a gravitational collapse of a star.
Finding a metric with intrinsic rotation could give insight to such problem. Due to the lack of spherical symmetry, the problem became much harder, and took roughly 50 years after Schwarzschild's discovery to find a metric for a rotating body.
Imposing symmetries to the final metric were essential to solve the field equation.

\subsection{Spacetime symmetries}

If we represent our spacetime by $(\mathcal{M}, g_{\mu\nu}, \psi)$, then the pullbash $f^*$ of the diffeomorphism $f:\mathcal{M}\rightarrow\mathcal{M}$, would give us the same physical system $(\mathcal{M}, f^* g_{\mu\nu}, f^* \psi)$.
Since diffeomorphisms are just active coordinate transformations, such concept may raise some confusion, as we don't seam to obtain no new information to work with. 
Almost all physics theories are coordinate invariant, as is Newtonian mechanics and Special Relativity, but in such theories there is a preferable coordinate system, while the same does not hold true for GR.
An analogies can be made with the path integral formalism in QFT, where special consideration is taken when summing all field configurations in order to not overcount indistinguishable configurations, as is the case of gauge field theories.
A similar ambiguity can occur in GR, where two apparently different solutions which can be related by a diffeomorphism and are actually ``the same'', so we must be careful when deriving and analyzing any geometries.

Despite the added complexity of Einstein's field equations, it is still possible to find exact nontrivial solutions in a systematic way by considering spacetimes with symmetries with the use of Killing vector fields.
A vector field $\xi$ that obeys
\begin{align}
    \mathcal{L}_\xi  g = 0  
    \label{eq2:killing}
\end{align}
is called a Killing field. Locally, this expression reduces to $\nabla_\mu \xi_\nu + \nabla_\nu \xi_\mu = 0$.

A \emph{stationary} solution implies the existence of a Killing vector $k$ that is asymptotically timelike, $k^2<0$, therefore allows us to normalize our vector such that $k^2 \rightarrow -1$. 
Unlike the case of the static spacetime, a stationary metric does not show invariance under reversal of the time coordinate, which is natural considering a system with angular momentum. 
Futhermore, a solution is also \emph{axisymmetric}, due to the presence of a asymptotically spacelike Killing field $m$ whose integral curves are closed.
A solution is stationary and axisymmetric if both symmetries are present, along with commuting fields, $[k , m] = 0$, \emph{i.e.} rotations along with the axis of symmetry commute with time translations. The commutativity of the fields implies the existence of a set of coordinates, $(t,r,\theta,\phi)$, such that
\begin{align}
    k = \frac{\partial}{\partial t} ~, \qquad m = \frac{\partial}{\partial \phi} ~.
    \label{eq2:tPhiKilling}
\end{align}
As for direct implication of this choice of chart, components of the metric stay independent of $(t,\phi)$, in virtue of~\eqref{eq2:killing},
\begin{align}
    (\mathcal{L}_m g)_{\mu\nu} = \frac{\partial g_{\mu\nu}}{\partial \phi} = 0 ~,
    \label{eq2:lieMetricTPhi}
\end{align}
with the same holding true for $k$, hence we can write $g_{\mu\nu} = g_{\mu\nu}(r,\theta)$. 

One of the major applications of Killing vectors is to find conserved charges associated with the motion along a geodesic spanned by field.
These quantities are defined by taking the geodesics to regions space that are asymptotical flat, where the geometry does not affect the observer.
In the case of Kerr solution, we have two Killing vectors, $k$ and $m$, which are naturally associated with the total mass $M$ and angular momentum $J$ of the BH, respectively.
This is usually done by evaluating the Komar integrals~\cite{Heusler1996, Wald2010}, which can be written a covariant way as
\begin{alignat}{6}
    M = &&\, -\frac{1}{8 \pi} & \int_{S^2_\infty} \star \dd k^\flat \,& = &&\, \frac{1}{4} & \lim_{r\to\infty}  \int_0^\pi \dd\theta \sqrt{-g} \, g^{t\alpha} g^{r\beta} g_{t[\alpha,\beta]} ~, \label{eq2:komarMass} \\
    J = &&\, \frac{1}{16 \pi} & \int_{S^2_\infty} \star \dd m^\flat \,& = &&\, - \frac{1}{8} & \lim_{r\to\infty}  \int_0^\pi \dd\theta \sqrt{-g} \, g^{t\alpha} g^{r\beta} g_{\phi[\alpha,\beta]} ~, \label{eq2:komarSpin}
\end{alignat}
where the usual notation $k^\flat = g(k, \,\cdot\,) = g_{\mu\nu} k^\mu \dd x^\nu$ transforms a vector into a one-form and $\star : \Omega^{p}(\mathcal{M})\to\Omega^{4-p}(\mathcal{M})$ is the Hodge dual map.
In order to complete the integration in the last step is assumed~\eref{eq2:tPhiKilling} and~\eref{eq2:lieMetricTPhi}, keeping $(t,r)$ constant. 
According to the widely accepted of \emph{no-hair conjecture}~\cite{Carter1971}, these two quantities completely define a stationary (neutral) BH. 


\subsection{Kerr-Child coordinates}

Naturally, Kerr wasn't the only after such solution.
Many presented other metrics to approximately describe a rotating star. 
Most of the solutions were modified one-parameter modification to Schwarzschild that were not flat in case of the standard case. 
Simply using stationary and axisymmetric symmetries and then solving the Einstein's equations clearly wouldn't suffice.

Kerr success originated in of Petrov's classification of spacetimes, which used the algebraic properties of the Weyl tensor to distinguish the solutions in 3 types, along with some subcases.
He assumed that his solution would have the same classification as Schwarzschild's, which associated with the gravitational fields of isolated massive objects, such as stars and BHs. 
From this assumption, using GR spinor techniques in Newman-Penrose formalism, a only then imposing the Killing vectors in~\eqref{eq2:tPhiKilling}, was possible to find a new solution. 
Kerr's metric appear in his original paper in the form
\begin{align}
    \begin{split}
        g = &- \left(1 - \frac{2 M r}{\rho^2} \right) (\dd v - a \sin^2\theta \dd \chi )^2 \\
        &+ 2  (\dd v - a \sin^2\theta \dd \chi )  (\dd r - a \sin^2\theta \dd \chi ) \\
        &+ \rho^2 (\dd \theta^2 + \sin^2\theta \dd \chi^2 ) ~,
    \end{split}
    \label{eq2:KerrIngoingEF}
\end{align}
where $a$ is a parameter, $M$ is the Komar mass and $\rho^2 = r^2 + a^2 \cos^2\theta$. Naturally the stationary Killing vector is $\partial_v$ and $\partial_\chi$ is the axial field, which implies that $J = a M$.

Taking the limit of $a\to0$, we reduce the metric to the Schwarzschild solution in ingoing Eddington-Finkelstein coordinates, $(v,r,\theta,\chi)$, which are useful to study ingoing (to the horizon) geodesics and remove the horizon coordinate singularity.
If a given metric has singularities, then it is not trivial to identify if is a physical singularity or just an artifact resultant of choice of the chart, which can simply be removed by a better choice of coordinates. 
That being said, this raises the difficulty of finding the essential singularities.
The best way to look to these singularities is to compute curvature scalar quantities, and if they diverge in one chart then they diverge on all charts.
Since any BH is just a vacuum solution, then the Ricci scalar vanishes, $R=0$, so we resort to the Kretschmann scalar,
\begin{align}
    R_{\mu\nu\rho\sigma} R^{\mu\nu\rho\sigma} = \frac{48 M (r^2 - a^2\cos^2\theta) \left[ (r^2 - a^2\cos^2\theta) ^2 - 16 r^2 M^2 a^2\cos^2\theta \right] }{(r^2 + a^2\cos^2\theta)^6} ~,
    \label{eq2:KerrKretschmann}
\end{align}
that diverges for $\rho^2=0$.
The Schwarzschild singularity, $r=0$, is replaced with the Kerr singularity $(r,\theta)=(0,\pi/2)$.
It is not clear what is the geometry of the Kerr singularity if we interpret $r$ and $\theta$ as being part of the ordinary spherical coordinates.
Although the metric is singular, considering $(t,r,\theta)$ constant and then the limit of $r\to0$ through the equatorial plane, the metric
\begin{align}
    g_{\rvert\mathrm{singularity}} \sim a^2 \dd \chi^2 ~
    \label{eq2:KerrKretschmann}
\end{align}
is reduced to the line element of the circle, $S^1$, confirming a \emph{ring singularity} of radius $a$.
This result implies that only approaching the a Kerr BH through the equatorial plane we may reach the singularity $\rho^2=0$. 

The Kerr-Child ``cartesian'' form, 
\begin{align}
    \begin{split}
        g = &- \dd \tilde{t}^2 + \dd x^2 + \dd y^2 + \dd z^2 \\
        &+ \frac{2 M r^3}{r^4 + a^2 z^2} \left[ \dd \tilde{t} + \frac{r (x \dd x + y \dd y) - a (x \dd y - y \dd x)}{r^2+a^2} + \frac{z}{r} \dd z \right]^2 ~,
    \end{split}
    \label{eq2:KerrChild}
\end{align}
is useful to really observe the ring singularity.
In this metric, $r$ is no longer a coordinate but a function of this chart coordinates $(\tilde{t},x,y,z)$.
We can relate the The Kerr-Child metric to the original Kerr solution, using
\begin{align}
    \tilde{t} = v - r ~, \qquad x+ i y = (r -i a) e^{i \chi} \sin\theta ~,\qquad z=r\cos\theta ~,
    \label{eq2:InEFtoKChild}
\end{align}
which implies that $r(x,y,z)$ is implicitly given by
\begin{align}
    r^4 - (x^2+y^2+z^2-a^2)r^2 -a^2 z^2 = 0 ~.
    \label{eq2:rConditionKChild}
\end{align}
This condition deserves a more in-depth analysis.
For increasing $r$, the surfaces obeying~\eqref{eq2:rConditionKChild} approximates perfect spheres as the geometry get more and more flat, as is observed in~\eref{eq2:KerrChild}. Minkowsky flat space is also guaranteed for $M=0$.
On the other hand, as we approach the singularity on $z=0$ and $x^2+y^2 = a^2$, rotation effects deform the surfaces into oblate spheroids (for the strict inequality, the singularity is removable).
Such remarks are visually demonstrated in~\fref{fig2:kerrchild}.

\begin{figure}[h]
    \centering
    \begin{subfigure}[c]{0.45\textwidth}
        \includegraphics[width=\textwidth]{kerrchild2d}
    \end{subfigure}
    \hspace{1cm}
    \begin{subfigure}[c]{0.35\textwidth}
        \includegraphics[width=\textwidth]{kerrchild3d}
    \end{subfigure}
    \caption{Contour plots of the surface $r(x,y,z)$ for constant values of $0,\,1/2,\,1,\,3/2$, in the Kerr-Child ``cartesian'' coordinates. The left plot is the intersection with $z=0$ plane with the 3D representation (right) that spotlights the ring singularity. Dashed curves, representing orthogonal constant $\theta(x,y,z)$ hypersufaces, become asymptotically affine.}\label{fig2:kerrchild}
\end{figure}

Even thought both metrics $r>0$, there is no mathematical reason to restrict $r$ strictly to positive values.
Particularly for Kerr-Child, hypersurfaces of constant $r$ can be represented also by $-r$. 
This means that this chart can be analytically extended to regions where $r<0$.
From this procedure and a proper collage of charts it is possible to achieve a \emph{maximally extended} solution, with gives mathematical access to new spacetime regions, even tough most of them show unphysical properties.


\subsection{Boyer-Linquist coordinates}

Considering the problem in hand, the most suitable coordinates for work with the NP formalism, are the Boyer-Linquist coordinates
\begin{align}
    \begin{split}
        g = &- \left(\dd t - \frac{2 M r}{\rho^2} \right) \dd t^2 - 2 a \sin^2\theta \frac{(r^2+a^2-\Delta)}{\rho^2} \dd t \dd \phi \\
        &+ \frac{(r^2+a^2)^2- \Delta a^2 \sin^2\theta}{\rho^2} \sin^2\theta \dd\phi^2 + \frac{\rho^2}{\Delta} \dd r^2 + \rho^2 \dd \theta^2 ~,
    \end{split}
    \label{eq2:KerrBL}
\end{align}
where we define $\Delta=r^2-2 M r + a^2$. In order to show that these corresponds to the same solution, the change of coordinates
\begin{align}
    \dd v= \dd t + \frac{r^2+a^2}{\Delta} \dd r ~, \qquad \dd\chi = \dd\phi + \frac{a}{\Delta} \dd r ~.
    \label{eq2:InEFtoBL}
\end{align}
This coordinates are usually referred as ``Schwarzschild like'', as it takes the spherical static case in standard curvature coordinates when setting $a=0$. 
Time inversion symmetry is characteristic of static Schwarzschild spacetime, but not for Kerr.
Nevertheless, this specific form is invariant under the inversion $(t,\phi)\to(-t,-\phi)$, also known as the \emph{circular condition}, an intuitive notion from physical systems with angular momentum.
This discrete symmetry eliminates most of the off-diagonal components of the BL metric, $g_{tr} = g_{\phi r} = g_{t \theta} = g_{\phi \theta} = 0$, making it the simplest to perform calculations.

The one-form, $n = \dd r$, defines normal vectors to constant radial surfaces.
It is easy to show that $n^2 = g^{rr}$, which implies that $n$ is null when $\Delta=0$, defining hypersurfaces at 
\begin{align}
    r_\pm = M \pm \sqrt{M^2 - a^2} ~,
    \label{eq2:KerrRadius}
\end{align}
singularities at $g_{rr}$ which we know to be removable.
As a consequence, from a stationary observer point of view, a massless particle on an ingoing null geodesic would spiral around the BH for a infinite time, as the coordinate $t\to\infty$, never reaching $r=r_{+}$.
This surface is the event horizon of the Kerr BH, as it separates two causally disconnected regions of spacetime, \emph{i.e} any information from the inside this surface, will never reach any asymptotic observer. 
The expression for the event horizon surface also raises limitations for the amount of angular momentum a physical BH can have.
We must have 
\begin{align}
    |a| < M ~,
    \label{eq2:spinLimit}
\end{align}
otherwise $\Delta$ would lack any real roots and would lead to a essential \emph{naked singularity}, reachable in a finite observable time, which is forbidden by the \emph{Weak Cosmic Censorship}.  

The surface at $r=r_{-}$, on the other hand, is called a Cauchy horizon.
In GR, a spacelike surface (Cauchy surface) containing all initial conditions of spacetime would suffice to predict all past and future events, but a Cauchy horizon separates the domain of validity of such initial conditions.
Despite no information ever escaping the event horizon, it is still possible to predict events inside $r_{-} < r < r_{+}$, but such thing it is not guaranteed after crossing the Cauchy horizon.
Due to this and some other unphysical features (for example, closed timelike curves), we need only to focus on the region outside the event horizon $r>r_{+}$, since only information on that region is physically reachable.

Event tough most of the Kerr BH properties were shown, there is was no result so far showed some kind of rotation.
First, consider the quantity $\xi_\mu u^\mu$, where $u^\mu$ is the four-velocity vector and $\xi^\mu$ is a Killing field.
Being aware of the geodesic equation, $u^\nu \nabla_\nu u^\mu = 0$, it is easy to show that this quantity is conserved along geodesics, \emph{i.e.}
\begin{align}
    u^\nu \nabla_\nu ( \xi_\mu u^\mu ) = u^\mu u^\nu \nabla_\nu \xi_\mu = \frac{u^\mu u^\nu }{2} ( \nabla_\mu \xi_\nu + \nabla_\nu \xi_\mu ) = 0 ~,
    \label{eq2:geodesicKilling}
\end{align}
due to Killing~\eqref{eq2:killing}.
As a result, geodesics of a free particle in Kerr geometry will be characterized by two constants
\begin{align}
    -\epsilon &= k^\mu g_{\mu\nu} \frac{\dd x^\nu}{\dd \tau} ~, \qquad \ell = m^\mu g_{\mu\nu} \frac{\dd x^\nu}{\dd \tau} ~,
    \label{eq2:geodesicConsts}
\end{align}
where $\tau$ is the affine parameter fo the geodesic.
These quantities can be interpreted the energy per mass and angular momentum per mass of the particle, respectively.
Due to the circular form of the BL metric, the metric components of the coordinates $(t,\phi)$ define a product decomposition, providing the separation of previous equations, 
\begin{align}
    \dot{t} &= \frac{1}{\Delta} \left[ (r^2+a^2 +\frac{2 M a^2}{r})\epsilon - \frac{2 M a}{r} \ell \right] ~, \\
    \dot{\phi} &= \frac{1}{\Delta} \left[ \frac{2 M a}{r} \epsilon +\left( 1- \frac{2 M}{r} \right) \ell \right]  ~,
    \label{eq2:geodesicPhiT}
\end{align}
specified for the equatorial plane $\theta=\pi/2$. The final equation for the geodesic is provided by the line element (\ref{eq2:KerrBL}), which becomes also a first order ODE, after the substitution of $\dot{t}$ and $\dot{\phi}$. If a test particle would start with $\ell=0$ relative to a zero angular momentum observer (ZAMO), then we can get the angular velocity $\Omega$, as measured at infinity
 \begin{align}
    \Omega = \frac{\dot{\phi}}{\dot{t}} = - \frac{g_{t\phi}}{g_{\phi\phi}} = \frac{2 a M}{r^3 + a^2 (2 M+r)} ~.
    \label{eq2:geodesicKilling}
\end{align}
Asymptotically we obtain $\Omega\to0$, consistent with the ZAMO measurements, but for a finite distance, infalling geodesics are forced to co-rotate with the BH. 
Particularly, at the event horizon, $r=r_+$, on finds that
 \begin{align}
    \Omega_H = \frac{a}{2 M r_+} = \frac{a}{2 M (M+\sqrt{M^2-a^2})} ~.
    \label{eq2:geodesicKilling}
\end{align}

\subsection{Ergoregion and Penrose process}

One of the main characteristic that distinguishes Kerr BH from other spherical solutions is the existence of a $Ergoregion$ which allows for energy extraction from th BH an therefore superradiance. The surface is characterized by being the stationary limit, \emph{i.e.} when $k^2=g_{tt}=0$

Much like spontaneous pair creation and amplification at discontinuities are related but distinct effects, the Penrose process allows for a 


From now on, all results will be provided using BL coordinates.

\section{Newman-Penrose formalism}

\subsection{Kinnersly tetrad}
\subsection{Spin coefficients}
\subsection{Maxwell equations}


\cleardoublepage 
%% !TEX root = ../main.tex

\chapter{Teukolsky master equation} % Main chapter title
\label{Chapter3}

\section{Newman-Penrose formalism}

Study of gravitational and electromagnetic perturbations in a BH background were performed long before Kerr found his solution, for other spacetimes such as Schwarzschild's. Despite it's simplicity, the procedure involved was already algebraically tedious. In the Kerr case, the metric was far more complicated, making the problem almost untreatable.

Fortunately, the NP formalism~\cite{Newman1962} provides an alternative method of studying perturbations.
Results as a natural introduction of spinor techniques into GR, after the choice of a null complex tetrad basis,
\begin{align}
    e_a = (e_a)^\mu \frac{\partial}{\partial x^\mu} \qquad (a = 1, 2, 3, 4) ~,
\end{align}
where all quantities will be projected, \emph{i.e.} for the Weyl tensor we define
\begin{align}
    C_{abcd} =  (e_a)^\alpha  (e_b)^\beta  (e_c)^\gamma  (e_d)^\delta  C_{\alpha\beta\gamma\delta} ~.
\end{align}
Penrose believed that the light-cone was the essential element of the spacetime, thus it was of importance to find null directions. The basis consisted in two real vectors, $l$ and $n$, and two complex conjugate vectors $m$ and $\bar{m}$. Besides satisfying
\begin{align}
    l^2 = n^2 = m^2 = \bar{m}^2 = 0 ~,
\end{align}
orthogonality conditions of NP formalism require
\begin{align}
    l \cdot m = l \cdot \bar{m} = n \cdot m = n \cdot \bar{m} = 0 ~.
\end{align}
Still we are left with the ambiguity raised by multiplication of scalar functions to each vector, therefore its customary to impose normalization conditions to the basis,
\begin{align}
    l \cdot n = 1 ~, \qquad m \cdot \bar{m} = -1 ~.
\end{align}
This formalism is a special case of tetrad calculus, where we can identify the new basis as $(l,n,m,\bar{m})$. The ``metric'' for manipulating tetrad indices, $\eta_{ab}$, is defined by all restrictions provided above,
\begin{align}
    g^{\mu\nu} = \eta^{ab} (e_a)^\mu (e_b)^\nu = l^{\mu} n^{\nu} + n^{\mu} l^{\nu} - m^{\mu} \bar{m}^{\nu} - \bar{m}^{\mu} m^{\nu} ~.
\end{align}
Additionally this vectors define new directional derivatives.
We will depart shortly from standard notation~\cite{Teukolsky1972,Teukolsky1973,Teukolsky1974}, by redefining these derivatives as
\begin{align}
    \mathbbl{D}=\nabla_l ~,\qquad \mathbbl{\Delta}=\nabla_n ~,\qquad \bbdelta=\nabla_m ~,\qquad \bar{\bbdelta}=\nabla_{\bar{m}} ~.
    \label{eq3:tetradCovDer}
\end{align}

More details and definitions on the tetrad formalism can be found in~\aref{AppendixTetradFormalism}.

\subsection{Kinnersley tetrad}

The Riemann tensor may have up to twenty non-vanishing components.
We know that ten of these are present in the symmetric Ricci tensor, that is intrinsically connected to matter and energy.
The other components are pure gravitational degrees of freedom and are encoded in the Weyl tensor.
It becomes the most useful object when the Ricci tensor vanishes, such as vacuum solutions and source-free gravitational waves.
In order to remove the Ricci tensor degrees of freedom, the tensor must be constructed trace-free,
\begin{align}
    \eta^{ad} C_{abcd} = C_{1bc2} + C_{1bc2} - C_{3bc4} - C_{4bc3} = 0 ~. 
\end{align}
Together with the other symmetries inherited from the Riemann tensor, for instance the first Bianchi identity, $C_{a[bcd]}=0$, it is possible to vanish some components and rewrite others such that only ten degrees of freedom remain.
As a result, in NP formalism the Weyl tensor can be represented by five complex scalars, usually chosen as
\begin{align}
    \begin{split}
        \psi_0 &= - C_{1313} = - C_{\alpha\beta\gamma\delta}\, l^\alpha m^\beta l^\gamma m^\delta ~,\qquad
        ~\psi_1 = - C_{1213} = - C_{\alpha\beta\gamma\delta}\, l^\alpha n^\beta l^\gamma m^\delta ~,\\
        \psi_2 &= - C_{1342} = - C_{\alpha\beta\gamma\delta}\, l^\alpha m^\beta \bar{m}^\gamma n^\delta ~,\qquad
        \psi_3 = - C_{1242} = - C_{\alpha\beta\gamma\delta}\, l^\alpha n^\beta \bar{m}^\gamma n^\delta ~,\\
        \psi_4 &= - C_{2424} = - C_{\alpha\beta\gamma\delta}\, n^\alpha \bar{m}^\beta n^\gamma \bar{m}^\delta ~.
    \end{split}
\end{align}
The complex conjugates can be obtained by doing the replacement $3 \rightleftarrows 4$, by exchanging $m$ with $\bar{m}$ and vice-versa. 
Weyl tensor has a unique decomposition in therms of a linear combination of NP scalars and tensorial product of two-forms such as $l_{[\mu} n_{\nu]}$ and $m_{[\rho} \bar{m}_{\sigma]}$. 
It is clear that the values that these five complex scalars take is completely dependent on the choice of tetrad frame. 

BH solutions are ``type D'' spacetimes according to Petrov's classification, which was a major restriction necessary to the discovery of Kerr's metric.
For these spacetimes it is possible to find two different doubly-degenerate principal directions of the Weyl tensor, which we choose to be the real vectors of the tetrad, $l$ and $n$~\cite{Chandrasekhar1998}.
These yield
\begin{align}
    C_{\mu\alpha\beta[\nu} l_{\rho]} l^\alpha l^\beta = 0 ~, \qquad C_{\mu\alpha\beta[\nu} l_{\rho]} n^\alpha n^\beta = 0  ~.
\end{align}
In NP formalism terms, this implies, respectively,
\begin{align}
    \psi_0=\psi_1=0 ~,\qquad \psi_3=\psi_4=0 ~.
\end{align}

Finding the principal directions may not be trivial, but we can apply successive local transformations of the six-parameter Lorentz group in order to rotate the tetrad vectors. 
This procedure allows for the simplification of the Weyl tensor by vanishing NP scalars, ``locking'' the orientation of the tetrad frame.
Weyl scalar $\psi_2$ becomes invariant under boosts in the principal directions.
These keep the light-cone structure intact by maintaining the direction of $l$ and $n$ unchanged (up to multiplication of scalar functions), being useful to change between ingoing and outgoing frames~\cite{Teukolsky1974}. 
Kinnersly solved type D vaccum field equations~\cite{Kinnersley1969}, finding a suitable tetrad 
\begin{align}
    \begin{split}
        l =& \left(\frac{r^2+a^2}{\Delta}, \, 1, \,0, \,\frac{a}{\Delta} \right) ~, \\
        n =& \frac{1}{2 \rho^2} \Bigr( r^2+a^2, \,-\Delta, \,0 , \,a \Bigr) ~, \\
        m =& \frac{1}{ \sqrt{2} \bar{\rho}^2 } \Bigr( i a \sin\theta, \,0, \,1, \, i \csc\theta \Bigr) ~,
    \end{split}
    \label{eq3:kinnerslytetrad}
\end{align}
where $\bar{\rho} = r + i a \cos\theta$ and $\rho^2 = \bar{\rho} \bar{\rho}^*$.

The NP formalism provides a full set of first-order coupled diferential equations, relating the NP scalar (Weyl and Maxwell tensors) and the spin-coeficients resultant of the Kinnersley tetrad. These equations result from second Bianchi identity, $C_{\mu\nu[\rho \sigma ; \lambda ]} = 0$, and~\eqref{eq2:maxwellEM}.
To study GWs, instead of perturbing the background metric, the NP formalism provides a natural way of performing perturbations by modification of the tetrad, $l=l^B+l^P$, $n=n^B+n^P$, etc., and also the NP scalars, $\psi_a = \psi_a{}^B + \psi_a{}^P$, maintaining only first-order terms.
The formalism reveals decoupled equations for $\psi_0{}^P$ and $\psi_4{}^P$, which implies that these dynamic variables are the only independent degrees of freedom of the GWs.

\subsection{Maxwell equations}

We focus with more detail on EM perturbations with a fixed background because is a simpler procedure and then we will tie with the same master equation that also describes GW perturbations. 

In NP formalism, all Maxwell equations, $F_{[\mu\nu ; \rho]}=0$ and~\eqref{eq2:maxwellEM}, reduce to
\begin{align}
    F_{[ab \rvert c]} = 0 ~,\qquad \eta^{bc} F_{ab \rvert c} = 0 ~.
    \label{eq3:maxwellFabEqs}
\end{align}
The Mawell tensor $F_{\mu\nu}$ has a total of six components which encodes the vector quantities of the electric and the magnetic fields. We may reduce the equation using three complex NP scalars,
\begin{align}
    \begin{split}
        \phi_0 &= F_{13} = F_{\alpha\beta} l^\alpha m^\beta ~,\qquad
        \phi_1 = \tfrac{1}{2} (F_{12} + F_{43}) = \tfrac{1}{2} F_{\alpha\beta} (l^\alpha n^\beta + \bar{m}^\alpha m^\beta) ~,\\
        \phi_2 &= F_{42} = F_{\alpha\beta} \bar{m}^\alpha n^\beta ~.
    \end{split}
    \label{eq3:maxwellNPphi}
\end{align}
Considering all posible combinations of NP indices in~\eref{eq3:maxwellFabEqs}, we gather eight equations, double the amount of necessary relations. 
This occurs because the conjugates $\phi_0^*$, $\phi_1^*$, $\phi_2^*$ are coupled in these equations. Eliminating every term of the form $F_{23\rvert a}$ or $F_{14\rvert b}$, 
\begin{subequations}
    \begin{align}
        \phi_{2\rvert 1} &= \phi_{1\rvert 4} ~, \label{eq3:phi21phi14}\\
        \phi_{1\rvert 2} &= \phi_{2\rvert 3} ~, \label{eq3:phi12phi23}\\
        \phi_{1\rvert 1} &= \phi_{0\rvert 4} ~, \label{eq3:phi11phi04}\\
        \phi_{0\rvert 2} &= \phi_{1\rvert 3} ~. \label{eq3:phi02phi13}
    \end{align}
\end{subequations}
We may expand explicitly the left-hand side of~\eqref{eq3:phi21phi14},
\begin{align}
    \begin{split}
        \phi_{2\rvert 1} &= \phi_{2, 1} - \eta^{ab}( \gamma_{a41} F_{b2} + \gamma_{a21} F_{4b} ) \\
        &= \phi_{2, 1} - (\gamma_{241} F_{12} + \gamma_{121} F_{42}) + ( \gamma_{341} F_{42} + \gamma_{421} F_{43} ) \\
        &= \phi_{2, 1} + 2 F_{42} \left(\frac{\gamma_{341} + \gamma_{211}}{2}\right) + 2 \gamma_{421} \left(\frac{F_{12} + F_{43}}{2}\right) \\
        &= \mathbbl{D} \phi_2 + 2 \varepsilon \phi_2 - 2 \pi \phi_1 ~,
    \end{split}
    \label{eq3:phi21SpinCoef}
\end{align}
where we used the antisymmetry of the spin connection, $\gamma_{abc}=-\gamma_{bac}$. The right-hand side yields
\begin{align}
    \begin{split}
        \phi_{1\rvert 4} &= \phi_{1, 4} - \tfrac{1}{2} \eta^{ab}( \gamma_{a14} F_{b2} + \gamma_{a24} F_{1b} + \gamma_{a44} F_{b3} + \gamma_{a34} F_{4b} ) \\
        &= \phi_{1, 4} - \tfrac{1}{2} ( \gamma_{144} F_{23} + \gamma_{134} F_{42} + \gamma_{214} F_{12} + \gamma_{234} F_{41} ) \\ 
        &\qquad\qquad + \tfrac{1}{2} ( \gamma_{314} F_{42} + \gamma_{414} F_{42} + \gamma_{324} F_{14} + \gamma_{424} F_{13} ) \\
        &= \phi_{2, 1} - \gamma_{244} F_{13} + \gamma_{314} F_{42} \\
        &= \bar{\bbdelta} \phi_1 - \lambda \phi_0 + \tau \phi_2  ~.
    \end{split}
    \label{eq3:phi21SpinCoef}
\end{align}
The spin coefficients $\varepsilon$, $\pi$, $\lambda$, $\tau$ along with other NP definitions are found in~\aref{AppendixNPSpinCoef}. If we repeat the same expansion for the other Maxwell equations, we gather the set
\begin{subequations}
    \begin{align}
        \label{eq3:phi21phi14SpinCoef}
        \mathbbl{D} \phi_2 - \bar{\bbdelta} \phi_1 &= -\lambda \phi_0 + 2 \pi \phi_1 + (\varrho - 2 \varepsilon) \phi_2 ~,\\
        \label{eq3:phi12phi23SpinCoef}
        \mathbbl{\Delta} \phi_1 - \bbdelta \phi_2 &= \nu \phi_0 - 2 \mu \phi_1 + (2\beta-\tau) \phi_2 ~,\\
        \label{eq3:phi11phi04SpinCoef}
        \mathbbl{D} \phi_1 - \bar{\bbdelta} \phi_0 &= (\pi - 2 \alpha) \phi_0 + 2 \varrho \phi_1 - \kappa \phi_2 ~,\\
        \label{eq3:phi02phi13SpinCoef}
        \mathbbl{\Delta} \phi_0 - \bbdelta \phi_1 &= (2 \gamma - \mu) \phi_0 - 2 \tau \phi_1 + \sigma \phi_2 ~.
    \end{align}
\end{subequations}
The Kinnersley tetrad guarantees that $\kappa = \sigma = \lambda = \nu = 0$, decoupling all equations above. After substitution of all spin coefficients,
\begin{subequations}
    \begin{align}
        \label{eq3:phi21phi14Expand}
        \left( \mathbbl{D} + \frac{1}{\bar{\rho}^*} \right) \phi_2 &= 
        \left( \bar{\bbdelta} + \frac{2 i a \sin\theta}{\sqrt{2} (\bar{\rho}^*)^2} \right) \phi_2 ~,\\
        \label{eq3:phi12phi23Expand}
        \left( \mathbbl{\Delta} - \frac{\Delta}{\rho^2 \bar{\rho}^*} \right) \phi_1 &= 
        \left[ \bbdelta + \frac{1}{\sqrt{2} \bar{\rho}} \left( \cot\theta - \frac{i a \sin\theta}{\bar{\rho}^*} \right) \right] \phi_2 ~,\\
        \label{eq3:phi11phi04Expand}
        \left( \mathbbl{D} + \frac{2}{\bar{\rho}^*} \right) \phi_1 &=
        \left[ \bar{\bbdelta} + \frac{1}{\sqrt{2} \bar{\rho}^*}\left( \cot\theta - \frac{i a \sin\theta}{\bar{\rho}^*} \right) \right] \phi_0 ~, \\
        \label{eq3:phi02phi13Expand}
        \left[ \mathbbl{\Delta} + \frac{\Delta}{2 \rho^2} \left( \frac{1}{\bar{\rho}^*} - \frac{2(r-M)}{\Delta} \right) \right] \phi_0 &=
        \left(\bbdelta + \frac{2 i a \sin\theta}{\sqrt{2} \bar{\rho} \bar{\rho}^*}\right) \phi_1  ~.
    \end{align}
    \label{eq3:phiAllExpand}
\end{subequations}

An important consequence of the symmetries of Kerr spacetime allows for a wave decomposition of the form $\phi_0, \phi_1, \phi_2 \sim e^{- i \omega t + i m \varphi}$.
Therefore, the four differential operators group into radial $(\mathbbl{D}, \mathbbl{\Delta})$ and angular $(\bbdelta, \bar{\bbdelta})$. The procedure for separation of the Maxwell equations can be further simplified by introducing new operators
\begin{equation}
    \begin{alignedat}{3}
        \mathscr{D}_n &= \partial_r - \frac{i \mathscr{K}}{\Delta} + 2n \frac{r-M}{\Delta} ~,\qquad && \mathscr{D}_n^\dagger = \partial_r + \frac{i \mathscr{K}}{\Delta} + 2n \frac{r-M}{\Delta} ~,\\
        \mathscr{L}_n &= \partial_\theta - \mathscr{Q} + n \cot\theta ~,\qquad && \mathscr{L}_n^\dagger = \partial_\theta + \mathscr{Q} + n \cot\theta ~,
    \end{alignedat}
\end{equation}
where we define the functions $\mathscr{K}=(r^2+a^2)\omega - m a$, $\mathscr{Q} = a \omega \sin\theta - m \csc\theta$.
In this definition, $n$ is a non-negative integer.
These operators are related to the tetrad by
\begin{align}
    \mathbbl{D} = \mathscr{D}_0 ~,\qquad \mathbbl{\Delta} = - \frac{\Delta}{2 \rho^2 }\mathscr{D}^\dagger_0 ~,\qquad \bbdelta = \frac{1}{\sqrt{2} \bar{\rho}} \mathscr{L}^\dagger_0 ~,\qquad \bar\bbdelta = \frac{1}{\sqrt{2} \bar{\rho}^*} \mathscr{L}_0 ~,
\end{align}
as a result of the substitutions $\partial_t\rightarrow-i \omega$, $\partial_\varphi\rightarrow i m$.
We may use the fact that $\mathscr{D}_n$ and $\mathscr{L}_n$ act mostly as radial and angular derivatives, respectively, to deduce the properties
\begin{subequations}
    \begin{align}
        \label{eq3:propDeltaD}
        \mathscr{D}_n \Delta &= \Delta \mathscr{D}_{n+1} ~, \\[0.15cm]
        \label{eq3:propSinL}
        \mathscr{L}_n \sin\theta &= \sin\theta\, \mathscr{L}_{n+1} ~, \\[0.15cm]
        \label{eq3:propBarRhoD}
        \left(\mathscr{D}_n + \frac{q}{\bar{\rho}^*} \right) \frac{1}{(\bar{\rho}^*)^p} &= 
        \frac{1}{(\bar{\rho}^*)^p} \left(\mathscr{D}_n + \frac{q-p}{\bar{\rho}^*} \right) ~, \\[0.15cm]
        \label{eq3:propBarRhoL}
        \left(\mathscr{L}_n + \frac{i q a \sin\theta}{\bar{\rho}^*} \right) \frac{1}{(\bar{\rho}^*)^p} &= 
        \frac{1}{(\bar{\rho}^*)^p} \left(\mathscr{L}_n + \frac{i (q-p) a \sin\theta}{\bar{\rho}^*} \right) ~, \\[0.15cm]
        \label{eq3:propCommutLD}
        \left(\mathscr{D}_n + \frac{q}{\bar{\rho}^*} \right) 
        \left(\mathscr{L}_n + \frac{i q a \sin\theta}{\bar{\rho}^*} \right) &= 
        \left(\mathscr{L}_n + \frac{i q a \sin\theta}{\bar{\rho}^*} \right)
        \left(\mathscr{D}_n + \frac{q}{\bar{\rho}^*} \right) ~,
    \end{align}
\end{subequations}
for any integers $p,q,n$, holding also for either $\mathscr{D}^\dagger_n$ or $\mathscr{L}^\dagger_n$.

In order to achieve the separable form, we still need to perform a replacement of the Maxwell NP scalars by new dynamical variables
\begin{align}
    \label{eq3:phiBarRhoToPhi}
    \Phi_0 = \phi_0 ~,\qquad \Phi_1 = \sqrt{2} \bar{\rho}^* \phi_1 ~,\qquad \Phi_2 = 2 (\bar{\rho}^*)^2 \phi_2  ~,
\end{align}
and using properties~\eref{eq3:propBarRhoD} and~\eref{eq3:propBarRhoL}, we go from Eqs.~\eref{eq3:phiAllExpand} to
\begin{subequations}
    \begin{align}
        \label{eq3:D0Phi2L0Phi1}
        \left( \mathscr{D}_0 - \frac{1}{\bar{\rho}^*} \right) \Phi_2 &=
        \left( \mathscr{L}_0 + \frac{i a \sin\theta}{\bar{\rho}^*} \right) \Phi_1 ~, \\
        \label{eq3:Dd0Phi1Ld1Phi2}
        \Delta \left( \mathscr{D}^\dagger_0 + \frac{1}{\bar{\rho}^*} \right) \Phi_1 &= 
        -\left( \mathscr{L}^\dagger_1 - \frac{i a \sin\theta}{\bar{\rho}^*} \right) \Phi_2 ~, \\
        \label{eq3:D0Phi1L1Phi0}
        \left( \mathscr{D}_0 + \frac{1}{\bar{\rho}^*} \right) \Phi_1 &= 
        \left( \mathscr{L}_1 - \frac{i a \sin\theta}{\bar{\rho}^*} \right) \Phi_0 ~, \\
        \label{eq3:Dd1Phi0Ld0Phi1}
        \Delta \left( \mathscr{D}^\dagger_1 - \frac{1}{\bar{\rho}^*} \right) \Phi_0 &= 
        -\left( \mathscr{L}^\dagger_0 + \frac{i a \sin\theta}{\bar{\rho}^*} \right) \Phi_1 ~.
    \end{align}
    \label{eq3:AllDPhiLPhi}
\end{subequations}
Now we may use commutatively property~\eref{eq3:propCommutLD} together with~\eref{eq3:propDeltaD} to separate the equations for $\Phi_0$ and $\Phi_2$.
In order to obtain the first equation, we must first apply the operator $(\mathscr{L}^\dagger_0 + i a \sin\theta/\bar{\rho}^*)$ to~\eqref{eq3:D0Phi1L1Phi0} and then use the commutativity relation to substitute~\eqref{eq3:Dd1Phi0Ld0Phi1}.
Similarly, applying $(\mathscr{L}_0 + i a \sin\theta/\bar{\rho}^*)$ to~\eqref{eq3:Dd0Phi1Ld1Phi2} we obtain the final equation.
Together yield
\begin{align}
    \label{eq3:DDLLPhi0}
    \left[ \Delta \mathscr{D}_1 \mathscr{D}^\dagger_1 + \mathscr{L}^\dagger_0 \mathscr{L}_1 + 2 i \omega (r+i a \cos\theta) \right] \Phi_0 = 0 ~, \\
    \label{eq3:DDLLPhi2}
    \left[ \Delta \mathscr{D}^\dagger_0 \mathscr{D}_0 + \mathscr{L}_0 \mathscr{L}^\dagger_1 - 2 i \omega (r+i a \cos\theta) \right] \Phi_2 = 0 ~.
\end{align}
Still, there is another way of combining equations, \emph{i.e} Eq.~\eref{eq3:Dd0Phi1Ld1Phi2} with~\eref{eq3:Dd1Phi0Ld0Phi1} and the remaining two form the set
\begin{align}
    \label{eq3:LLPhi0DDPhi2}
    \mathscr{L}_0 \mathscr{L}_1 \Phi_0 &= \mathscr{D}_0 \mathscr{D}_0 \Phi_2 ~,\\
    \label{eq3:LLPhi2DDPhi0}
    \mathscr{L}^\dagger_0 \mathscr{L}^\dagger_1 \Phi_2 &= \Delta \mathscr{D}^\dagger_0 \mathscr{D}^\dagger_0 \Delta \Phi_0 ~.
\end{align}
Thus, we went from four first-order differential equations relating three NP scalars to four second-order differential equations, two of each decoupled, eliminating the need for the scalar $\Phi_1$.
The last two equations imply that each one of the complex NP scalars contains all the information necessary to describe a EM wave (two polarizations).
One may think that we only need one of each group of equations to solve all perturbations, as of today no closed form solution has been found.
Thus the problem has to be tackled using approximations or numerical methods, recurring to all last four equations.

Due to the nature of the operators $\mathscr{D}_n$ and $\mathscr{L}_n$, we may separate the equations for $\Phi_0 \sim R_{+1}(r) S_{+1}(\theta)$ and $\Phi_2 \sim R_{-1}(r) S_{-1}(\theta)$ into two pairs of equations, 
\begin{subequations}
    \begin{align}
        \label{eq3:separationDDRp}
        \left(\Delta \mathscr{D}_0 \mathscr{D}^\dagger_0 + 2 i \omega r \right) \Delta R_{+1} 
        &= \lambdabar \Delta R_{+1} ~, \\
        \label{eq3:separationLLSp}
        \left( \mathscr{L}^\dagger_0 \mathscr{L}_1 - 2 a \omega \cos\theta \right) S_{+1}
        &= - \lambdabar S_{+1}  ~, 
    \end{align}
    \label{eq3:separationRSp}
\end{subequations}
and
\begin{subequations}
    \begin{align}
        \label{eq3:separationDDRm}
        \left( \Delta \mathscr{D}^\dagger_0 \mathscr{D}_0 - 2 i \omega r \right) R_{-1}
        &= \lambdabar R_{-1} ~, \\
        \label{eq3:separationLLSm}
        \left( \mathscr{L}_0 \mathscr{L}^\dagger_1 + 2 a \omega \cos\theta \right) S_{-1}
        &= - \lambdabar S_{-1}  ~,
    \end{align}
    \label{eq3:separationRSm}
\end{subequations}
where $\lambdabar$ is a separation constant.
We use the property~\eref{eq3:propDeltaD} in to obtain~\eqref{eq3:separationDDRp}.
The constant $\lambdabar$ must be real, as the angular differential operators $\mathscr{L}_n$ are also real.
Notice that we not distinguish the separation constants of both equations. Performing the transformation $\theta \rightarrow \pi-\theta$, the angular operators transforms as $\mathscr{L}^\dagger_0 \mathscr{L}_1 \rightarrow \mathscr{L}_0 \mathscr{L}^\dagger_1$. Then if $S_{+1}(\theta)$ is a solution for~\eqref{eq3:separationLLSp} for a given separation constant $\lambdabar$, this implies that $\tilde{S}_{-1}(\theta)=S_{+1}(\pi-\theta)$ is a solution for~\eqref{eq3:separationLLSm} for the same constant. 
In other words, the separation constant must me the same for both equations. 
Also, solutions $R_{-1}$ and $\Delta R_{+1}$ obey the same complex conjugate equations due to $\mathscr{D}^\dagger_n=(\mathscr{D}_n)^*$.

(RELATIVE NORMALIZATION)

\subsection{Mode decomposition}

Will be more profitable to study Eqs.~\eref{eq3:DDLLPhi0} and~\eref{eq3:DDLLPhi2} as a special case of the Teukolsky master equation which describes all the linearized perturbations arround the Kerr geometry.
The generality of this equation is the primary reason for the focus on the EM case.
The treatment for GWs differs in the perturbation formalism only in algebraic complexity, resulting in the same master equation.
With the Teukolsky master equation we can proceed considering general perturbations, but there are several numerical and analytical details that make EM waves and GWs differ later on.

The general equation reads
\begin{align}
    \begin{split}
        & \frac{1}{\Delta^s} \frac{\partial}{\partial r} \left( \Delta^{s+1} \frac{\partial \Upsilon_s}{\partial r} \right) 
        + \frac{1}{\sin\theta} \frac{\partial}{\partial\theta} \left( \sin\theta \frac{\partial \Upsilon_s}{\partial \theta} \right) 
        - \left[ \frac{(r^2+a^2)^2}{\Delta} - a^2 \sin^2\theta \right]\frac{\partial^2 \Upsilon_s}{\partial t^2} \\[0.15cm]
        - & \frac{4 M a r}{\Delta}\frac{\partial^2 \Upsilon_s}{\partial t \partial \varphi} 
        - \left( \frac{a^2}{\Delta} -\frac{1}{\sin^2\theta} \right)\frac{\partial^2 \Upsilon_s}{\partial \varphi^2} 
        + 2s\left[ \frac{M(r^2-a^2)}{\Delta} - r - i a \cos\theta \right] \frac{\partial \Upsilon_s}{\partial t} \\[0.15cm]
        + & 2s\left[ \frac{a(r-M)}{\Delta}+\frac{i \cos\theta}{\sin^2\theta}\right] \frac{\partial \Upsilon_s}{\partial \varphi}
        - (s^2 \cot^2\theta - s) \Upsilon_s = 0 ~,
    \end{split}
    \label{eq3:teukolsky}
\end{align}
where $s$ is the field \emph{spin weight} and each field quantity $\Upsilon_s$ is related to the NP scalars as shown in the~\tref{tb3:solutionsTeukolskyEq}.
Depending on the spin weight, the equation may describe massless scalar ($s=0$) or Dirac fields ($s=\pm \tfrac{1}{2}$), as well as electromagnetic ($s=\pm 1$) or gravitational waves ($s=\pm 2$). Substituting the spin-weight for the EM waves we obtain Eqs.~\eref{eq3:DDLLPhi0} and~\eref{eq3:DDLLPhi2}.
\begin{table}[h]
    \centering
    \renewcommand{\arraystretch}{1.25}
    \begin{tabular}{ | c | l | }
        \hline
        $s$ & $\qquad ~ \Upsilon_s$ \\
        \hline\hline
        $+1$ & $\Phi_0 = \phi_0$ \\
        \hline
        $-1$ & $\Phi_2 = 2 (\bar{\rho}^*)^2 \phi_2$ \\
        \hline
        $+2$ & $\Psi_0 = \psi_0$  \\
        \hline
        $-2$ & $\Psi_4 = (\bar{\rho}^*)^4 \psi_4$ \\
        \hline
    \end{tabular}
    \caption{Performance at peak F-measure}
    \label{tb3:solutionsTeukolskyEq}
\end{table}

Obviously, Teukolsky equation is explicitly independent of $t$ and $\varphi$, thus $\Upsilon_s$ accepts a decomposition in $e^{-i \omega t + i m \varphi}$, which we already assumed in the EM case to separate the equations.
Stationary and axisymmetry of the spacetime geometry guarantees this form.
The azimuthal wave number $m$ must be an integer, due to periodic boundary conditions on the BL coordinate $\varphi$.
We may separate all perturbations in a completely general mode decomposition
\begin{align}
    \Upsilon_s = \sum_{\ell,m,\omega} e^{-i \omega t + i m \varphi} \, {}_{s}S_{\ell m}(\theta) {}_{s}R_{\ell m}(r) ~.
\end{align}
The integer $\ell$ plays a role in labelling all possible solutions for the eigenvalue problem of both radial and angular equations,
\begin{align}
    \label{eq3:teukolskyRadial}
    \frac{1}{\Delta^s} \frac{\dd}{\dd r} \left( \Delta^{s+1} \frac{\dd\, {}_{s}R_{\ell m}}{\dd r} \right)
    + \left[ \frac{\mathscr{K}^2 - 2 i s (r-M)\mathscr{K}}{\Delta} + 4 i s \omega r -{}_{s}\mathscr{C}_{\ell m}  \right] {}_{s}R_{\ell m} = 0 ~, \\[0.15cm]
    \label{eq3:teukolskyAngular}
    \frac{1}{\sin\theta} \frac{\dd}{\dd\theta} \left( \sin\theta \frac{\dd\, {}_{s}S_{\ell m}}{\dd \theta} \right)
    + \left[ a^2 \omega^2 \cos^2\theta - 2 s a \omega \cos\theta - \frac{(m + s \cos\theta)^2}{\sin^2\theta} + s + {}_{s}\mathscr{A}_{\ell m} \right] {}_{s}S_{\ell m}  = 0~.
\end{align}
The radial and angular eigenvalues are related to the separation constant on Eqs.~\eref{eq3:separationRSp} and~\eref{eq3:separationRSm} through
\begin{align}
    \mathcal{C}_{s\ell m} + 2 m a \omega = {}_{s}\mathscr{A}_{\ell m} + a^2 \omega^2 \stackrel[(s=\pm 1)]{}{=} \lambdabar - s(s+1) ~.
    \label{eq3:relationEVs}
\end{align}
Due to the form of the angular equation, the eigenvalues ${}_{s}\mathscr{C}_{\ell m}$, ${}_{s}\mathscr{A}_{\ell m}$ as well as the function ${}_{s}S_{\ell m}(\theta)$ depends also on the coupling $a \omega$.
Clearly, the same does not hold for the radial function ${}_{s}R_{\ell m}(r)$. 

\section{Spin-Weighted Spheroidal Harmonics}

To shed some light into the explicit form of $\lambdabar$, we will need to dive into the eigenvalue problem for the angular equation. We may transform the~\eqref{eq3:teukolskyAngular} into a more familiar form using the change of coordinate $z=\cos\theta$ and renaming the BH-wave coupling $c=a \omega$, obtaining
\begin{align}
    \frac{\dd}{\dd z} \left[ (1-z^2) \frac{\dd\, {}_{s}S_{\ell m}}{\dd z} \right] + \left[ (c z)^2 - 2 c s z  -\frac{(m + s z)^2}{1 - z^2} + s  + {}_{s}\mathscr{A}_{\ell m} \right] {}_{s}S_{\ell m} = 0 ~.
    \label{eq3:swshEq}
\end{align}
We may also use freely ${}_{s}S_{\ell m}(\cos\theta) \equiv {}_{s}S_{\ell m}(\theta)$.
We will consider $c$ real as we are analyzing superradiance of EM waves in the vacuum, although we could generalize the spin-weighted spheroidal harmonic (SWSH) equation to imaginary $c$ values to describe waves in a particular medium.

Spherically symmetric problems allow for a decomposition using spherical harmonics $Y_{\ell m}(\theta,\varphi)$ of angular dependent functions with finite boundary conditions.
These have innumerable applications in physics such as the hydrogen atom or the description of anisotropies in the cosmic microwave background.
By setting $s=0$ and $c=0$ (spherical), then it is clear that solution for~\eqref{eq3:swshEq} are given by the associated Legendre polynomials, $P^m_\ell(z)$, and the therefore it is a generalization of the spherical harmonics, \emph{i.e.}
\begin{align}
    e^{i m \varphi} \,{}_{0}S_{\ell m}(0; \theta) = Y_{\ell m}(\theta,\varphi) ~,\qquad {}_{0}\mathscr{A}_{\ell m}(0) = \ell (\ell + 1) ~,
\end{align}
where we explicitly wrote the dependency in the coupling $c$.
The value of $\ell$ are non-negative integers, with the restriction of $\ell \ge |m|$. 

Perturbations of any type in Schwarzschild spacetime are described using spin-weighted \emph{spherical} harmonics. The generalization from the $s=0$ case is well studied problem. Due to the shared symmetries with spherical harmonics it is possible to find a closed form for $s\ne0$ harmonics.

Since this generalization must hold for the spherical case, we require that spin-weighted spheroidal harmonics (SWSHs) are normalized for any spin $s$ and coupling $c$,
\begin{align}
    \int_{0}^\pi \dd\theta \sin\theta \, | \,{}_{s}S_{\ell m}(\theta) |^2 =
    \int_{-1}^{1} \dd z  \, | \,{}_{s}S_{\ell m}(z) |^2 = \frac{1}{2\pi} ~.
\end{align}

\subsection{Eigenvalues}

(TALK ABOUT THE ANALYTICAL PROBLEM AND PERTURBATION)

The major advances on the study of the eigenvalues of the SWSHs was performed by Leaver in 1985. He worked out the asymptotical and critical behavior of the equation. We observe that equation the equation diverges at the poles, $z = \pm 1$, where the it takes the form $(1 \mp z) \,{}_{s}S_{\ell m}{}'(z) \sim \mp \tfrac{1}{4} (m \pm s)^2 \, {}_{s}S_{\ell m}(z)$. In order to guarantee that SWSH is everywhere analytical, he proposed the series expansion at $x=-1$,
\begin{align}
    {}_{s}S_{\ell m}(z) = e^{c z} (1+z)^{k_-} (1-z)^{k_+} \sum_{p=0}^\infty a_p (1+z)^p ~,
\end{align}
where $k_{\pm} = \tfrac{1}{2}|m \pm s|$.
The exponential in the ansatz accounts for the large $z$ behavior of the equation.
Substituting in the angular equation, we obtain a three-term recurrence relation between the expansion coefficients $a_p$ and the boundary condition at $x=-1$,
\begin{align}
    \alpha_p a_{p+1} + \beta_p a_p + \gamma_p a_{p-1} = 0 ~,\qquad
    \alpha_0 a_1 + \beta_0 a_0 = 0 ~,
\end{align}
where
\begin{align}
    \begin{split}
        \alpha_p &= -2 (1 + p) (1 + 2 k_{+} + p) ~,\\
        \beta_p  &= (k_{-} + k_{+} + p - s) (1 + k_{-} + k_{+} + p + s) \\
        &\qquad\qquad\qquad - 2 c (1 + 2 k_{-} + 2 p + s) - c^2 - {}_{s}\mathscr{A}_{\ell m} ~,\\
        \gamma_p &= 2 c (k_{-} + k_{+} + p + s) ~.
    \end{split}
\end{align}
We then find an equation for the eigenvalue ${}_{s}\mathscr{A}_{\ell m}$ with explicit dependence on $m$, $s$ and $c$, by combining the previous relations into a continued fraction,
\begin{align}
    \beta_0 = \frac{\gamma_1 \alpha_0}{\beta_1 -} \frac{\gamma_2 \alpha_1}{\beta_2 -} \frac{\gamma_3 \alpha_2}{\beta_3 -} \cdots \equiv  \frac{\gamma_1 \alpha_0}{\beta_1 - \frac{\gamma_2 \alpha_1}{\beta_2 - \frac{\gamma_3 \alpha_2}{\beta_3 - \dots}}}  ~.
\end{align}
Just like the spherical case, for each fixed $c$, $m$, $s$, will have an infinite set of eigenfunctions each with its corresponding eigenvalue (LEAVER ARTICLE). These eigenvalues are roots of this equation

\subsection{Eigenfunctions}







\section{Analytic radial expansions}

\subsection{Near horizon approximation}
\subsection{Assymptotic expansion}

\section{Amplification factor $Z_{slm}$}


\cleardoublepage
%% !TEX root = ../main.tex

\chapter{Numerical methods} % Main chapter title
\label{Chapter4}

In this chapter will will develop the necessary method to compute the necessary coefficients to compute the gain/loss factor, using Mathematica\texttrademark.
We will go beyond the spherical approximation and calculate the SWSHs eigenvalues for any BH angular momentum.
With the eigenvalue defined for a particular mode, we will compute the asymptotic radial coefficients, which in turn are used to compute the amplification factor in three diferent ways.

%----------------------------------------------------------------------------------------

\section{Eigenvalues}

The need for obtaining the eigenvalues $\uu[s]{\mathscr{E}}{\ell m}$ rests on the dependency to solve the radial equation numerically with no spherical approximation.
Additionally, the relative normalization constant $\mathscr{B}$, which depends explicitly on the eigenvalue, will be rather important in one the methods used to calculate the gain/loss factor $\uu[s]{Z}{\ell m}$ for each mode $(\omega,\ell,m)$.
There is no reason to differentiate the eigenvalue for given BH angular momentum and a particular frequency, since the parameter dependence of the eigenvalue is $c=a\omega$.
Considering the focus in superradiant modes, we only will need eigenvalues in the range $0<c<3$. 
Even for extremal BHs, the typical frequency value for a superradiant mode, $\bar{\omega}\sim 1/2$, so this margin is sufficient even for observing the effects in non-superradiant modes.
Due to the circular symmetry, $\uu[s]{\mathscr{E}}{\ell, -m}(c) = \uu[s]{\mathscr{E}}{\ell m}(-c)$, instead of computing for negative values of $c$, we will consider all integer azimuthal numbers $|m|\le \ell$.

%----------------------------------------------------------------------------------------

\subsection{Leaver method}

The first method implemented was Leaver's.
Consists is using the three recursion relation obtained for SWSHs and correspondent continued fraction \eref{eq3:evInversion0th} and it's inversions.
Since the problem is now numerical, we have to stop the continued fraction at some particular $p=N$.
By substitution of the parameters $s$ and $m$ and $c$, we are left with an equation with $N$ roots for $\uu[s]{\mathscr{E}}{\ell m}$.
A root-finding algorithm is a method that allows to approximate roots of some equation $f(x)=0$, by suggestion of a connected region were $f$ has different signs at the boundary.
The method ``FindRoot'' in Mathematica\texttrademark~ allows to distinguish the roots of equation by finding the closest to a particular input value.
Firstly, we use the the expansion coefficients for $c\ll 1$~(\aref{AppendixEigenvalues}) to suggest a value of the eigenvalue $\uu[s]{\mathscr{E}}{\ell m}$ that is close to $\ell(\ell+1)$.
We improved on this method by starting the curve at $c=0$, and then obtaining the eigenvalue numerically for small increments in $c$ and then using the last eigenvalue solution as the initial guess for the next increment.
This is particularly useful to generate and save a complete table of eigenvalues for given range and then use interpolation methods to guess eigenvalues for intermediate $c$ values.

For both methods the obtained curves are well behaved for $\ell=1$, but for bigger $\ell$ we start to observe some discontinuities, especially when we increase the range of $c$.
\begin{figure}[h]
	\centering
	\vspace{0.2cm}
	\begin{subfigure}[c]{0.6\textwidth}
        \includegraphics[width=\textwidth]{discontinuitiesEV}
    \end{subfigure}
	\caption{Showcasing discontinuities in the values of $\uu[\pm 1]{\mathscr{E}}{\ell 1}$ for $m=1,2$ when using incorrect implementation of the Leaver method. Real values of the eigenvalues are shown as dashed lines ($m=1,2,3$) of the same color.}
	\label{fig4:discontinuitiesEV}
\end{figure}
For a fixed $s$ and $m$, we have an infinite number of curves labeled by $\ell$ and in some cases the root finding algorithm selecting roots from adjacent curves, either from the branch $\ell-1$, $\ell+1$ or even distant values.
These solutions cannot ever intersect, otherwise the eigenvalue would be degenerate and the SWSHs would not be a orthogonal basis of functions.
The issue rests on the lack of accuracy when identifying of the $\ell$-th root.
We lose accuracy when trying to obtaining roots on levels further down in the continued fraction.
We solve the problem by considering the inversion \eref{eq3:evInversionRth}, choosing $r=\ell+\max\{|m|,|s|\}$, as the main information in the value taken by the $\ell$-th root is in the $\beta_r$, with the continuous fractions providing higher order contributions in $c$.

Once the eigenvalue root is know, one can find any number of the series expansions coefficients $a_p$, for a particular eigenfunction \eref{eq3:SWSHseriesLeaver}, by using the three-coefficient recursion relation \eref{eq3:ap3CoefRecursion}.

%----------------------------------------------------------------------------------------

\subsection{Spectral method}

Due to initial problems with the Leaver method, we decided to use the spectral method.
Because the spheroidal \eqref{eq3:teukolskyAngular} can be seen as a perturbed version of the spherical case, $c=0$.
We may rewrite the equation using three operators depending on their order in $c$, 
\begin{align}
	\label{eq4:angularEqH012}
	( \mathscr{H}^{(0)} +  \mathscr{H}^{(1)} + \mathscr{H}^{(2)} ) \uu[s]{S}{\ell m} = - \uu[s]{\mathscr{E}}{\ell m}  \;\uu[s]{S}{\ell m} 
\end{align}
The zeroth order operator $\mathscr{H}^{(0)}$, defines the eigenvalue problem for the spin-weighted spherical harmonics, which will provide the complete non-perturbed basis, $\mathscr{H}^{(0)} \,\uu[s]{Y}{\ell m} = -\ell(\ell+1) \,\uu[s]{Y}{\ell m}$.
The other two operators are quickly identified from the angular equation as $\mathscr{H}^{(1)} = - 2 s c \cos\theta$ and $\mathscr{H}^{(2)} = c^2 \cos^2\theta$.
Simple perturbation theory states that
\begin{align}
	\begin{split}
		\uu[s]{\mathscr{E}}{\ell m} &= \ell(\ell+1) - \int \dd\Omega \,(\uu[s]{Y}{\ell m})^* \,\mathscr{H}^{(1)} \,\uu[s]{Y}{\ell m} + \mathscr{O}(c^2) ~, \\
		\uu[s]{S}{\ell m} &= \uu[s]{Y}{\ell m} - \sum_{j\ne\ell} \frac{\int \dd\Omega \,(\uu[s]{Y}{j m})^* \,\mathscr{H}^{(1)} \,\uu[s]{Y}{\ell m}}{j(j+1)-\ell(\ell+1)} \, \uu[s]{Y}{j m} +\mathscr{O}(c^2) ~.
	\end{split}
\end{align}
We may include $\mathscr{H}^{(2)}$ by using a higher order expansion, which can be found in any Quantum Mechanics textbook.
The integrals $\int \dd\Omega \,(\uu[s]{Y}{j m})^* \,\mathscr{H}^{(1)} \,\uu[s]{Y}{\ell m}$ and $\int \dd\Omega \,(\uu[s]{Y}{j m})^* \,\mathscr{H}^{(2)} \,\uu[s]{Y}{\ell m}$ may be computed using Clebsch-Gordon coefficients decomposition generalized for spin-weighted harmonics \eref{eqC:clebschGordanDecomp}.
These operators can be written in terms of general matrix elements in the basis of spin-weighted spherical harmonics,
\begin{align}
	\begin{split}
	h^{(1)}_{j\ell} &= \int \dd\Omega \,\cos\theta \,(\uu[s]{Y}{j m})^* \,\uu[s]{Y}{\ell m} = \sqrt{\frac{2 \ell+1}{2 j+1}} \langle \ell,m; 1,0 | j,m \rangle \langle \ell, -s; 1,0 | j, -s \rangle ~, \\
	h^{(2)}_{j\ell} &= \int \dd\Omega \,\cos^2\theta \,(\uu[s]{Y}{j m})^* \,\uu[s]{Y}{\ell m} = \frac{\delta_{j\ell}}{3} + \frac{2}{3}\sqrt{\frac{2 \ell+1}{2 j+1}} \langle \ell,m; 2,0 | j,m \rangle \langle \ell, -s; 2,0 | j, -s \rangle ~,
	\end{split}
\end{align}
remembering that $\cos\theta$ and $\cos^2\theta$ can be rewritten using $\uu[0]{Y}{10}(\theta,\varphi)$ and $\uu[0]{Y}{20}(\theta,\varphi)$.
The first integral is proportional Leaver series coefficient $f_1$ defined in~\aref{AppendixEigenvalues}.

Perturbation theory shows that the SWSHs can be expanded using of spherical harmonics.
This should not be a surprising fact as any angular function $f(\theta,\varphi)$ with a particular spin-weight can be represented using a decomposition using spin-weighted spherical harmonics.
Having this idea in mind, we write 
\begin{align}
	\label{eq4:SWSHsphericalExpansion}
	\uu[s]{S}{\ell m}(c; \theta, \varphi) = \sum_{j} b_{j}^{(\ell)} \,\uu[s]{Y}{j m}(\theta,\varphi) \qquad \Big( \ell, j \ge \max\{|s|,|m|\} \Big) ~.
\end{align}
Replacing the expansion in \eqref{eq4:angularEqH012}, we can take advantage of the orthogonality of the harmonics, $\int \dd\Omega \,(\uu[s]{Y}{j m})^* \,\uu[s]{Y}{\ell m} = \delta_{\ell j}$, by multiplying the the equation by $(\uu[s]{Y}{\ell m})^*$ and integrating the solid angle.
The angular equation is replaced by an eigenvalue matrix equation $\sum_{j} a_{ij} \, b_{j}^{(\ell)} = - \uu[s]{\mathscr{E}}{\ell m} \,b_{i}^{(\ell)}$, such that
\begin{equation}
	\label{eq4:spectralMatrix}
	a_{ij} =
	\begin{cases} 
		~c^2 \,h^{(2)}_{ii} - 2 s c \,h^{(1)}_{ii} - i(i+1) & ~ i=j \\[-0.5ex]
		~c^2 \,h^{(2)}_{ij} - 2 s c \,h^{(1)}_{ij} & ~ |i-j|=1 \\[-0.5ex]
		~c^2 \,h^{(2)}_{ij} & ~ |i-j|=2 \\[-0.5ex]
		~0 & ~\text{otherwise}
	\end{cases}  \qquad \Big( i, j \ge \max\{|s|,|m|\} \Big) ~,
\end{equation}
where the the eigenvalues of this matrix are $-\uu[s]{\mathscr{E}}{\ell m}$ and the correspondent eigenvector is given by $b_j^{(\ell)}$.

Like the Leaver method, we will have to truncate the matrix at some finite size.
From \eqref{eq4:spectralMatrix}, we know that the zeroth order contribution to the $\ell$-th eigenvalue will be the element $a_{\ell\ell}$.
We opted to implement a $N\times N$ centered submatrix such that $i,j\ge\ell_\mathrm{\min}\equiv\max\{|s|,|m|\}$ and truncating the matrix at $N > \ell+1 - \ell_\mathrm{\min}$, in order to include the $a_{\ell\ell}$ terms in the approximation.
In reality, we must implement a variable $N\equiv N(c)$, so that it increases the size of the taken submatrix in order to include extra corrections for larger values of $c$. The size of submatrix also increases linearly with $\ell$.
The best way to approximate $\uu[s]{\mathscr{E}}{\ell m}$ would be to not to construct the submatrix including all values of $\ell_\mathrm{\min}$ but center the submatrix at $a_{\ell\ell}$ and increase its size to fine-tune the eigenvalue.
Since we will take very large $\ell$ values, we will use the first method for simplicity.

\begin{figure}[h]
	\centering
	\vspace{0.2cm}
	\begin{subfigure}[c]{0.48\textwidth}
        \includegraphics[width=\textwidth]{plotEV1}
	\end{subfigure}
	\hfill
	\begin{subfigure}[c]{0.47\textwidth}
		\includegraphics[width=\textwidth]{plotEV2}
	\end{subfigure}
	\hfill
	\caption{Eigenvalues for $\ell=1$ (left) and $\ell=2$ (right) for typical values of $c$, using the spectral method.}
	\label{fig4:plotsEV12}
\end{figure}
Optimized numerical methods allow for fast computation of the eigenvalues and eigenvectors of a band-diagonal matrix.
The ``Eigensystem'' method found in Mathematica\texttrademark~ returns a array of eigenvalues and their correspondent normalized eigenvectors, guaranteeing \eqref{eq3:SWSHorthogonality}. Since the result is positively sorted list of $-\uu[s]{\mathscr{E}}{\ell m}$, with $0\le \ell-\ell_\mathrm{min}\le N-1$, of which we need to select the negative of the ($N-\ell+\ell_\mathrm{min}$)-th element.
We show EM eigenvalues form lower $\ell$ values in \fref{fig4:plotsEV12}.
\begin{figure}[!b]
	\centering
	\vspace{0.2cm}
	\begin{subfigure}[c]{0.48\textwidth}
        \includegraphics[width=\textwidth]{plotSWSH1}
	\end{subfigure}
	\hfill
	\begin{subfigure}[c]{0.48\textwidth}
		\includegraphics[width=\textwidth]{plotSWSH2}
	\end{subfigure}
	\hfill
	\caption{Plots of all spin-weighted spheroidal harmonics $\uu[-1]{S}{\ell m}(\theta,0)$ with $\ell=1$ (left) and $\ell=2$ (right) for $s=-1$ and $c=0.4$. This plot shares the same legend coloring as the above (\fref{fig4:plotsEV12}). Dotted curves represent the values of the $\uu[-1]{Y}{\ell m}$, when $c\to0$.}
	\label{fig4:plotSWSH12}
\end{figure}
This procedure also returns correct eigenvector for approximating the eigenfunction using \eref{eq4:SWSHsphericalExpansion}.
In order to ensure that the SWSHs have the same phase convections of their spherical counterparts, we must ensure that the correspondent eigenvector has $b_\ell^{(\ell)}>0$, by mapping the obtained vector components as $b_j^{(\ell)}\mapsto \mathrm{sgn}(b_\ell^{(\ell)}) \,b_j^{(\ell)}$.
This process is computationally more stressful, since it requires the computing and combining $N$ different spherical harmonics with the same $s$ and $m$.
In the superradiance range, the values of $c$ can be small $\left(<\tfrac{1}{2}|m|\right)$, therefore the SWSHs may not differ greatly from $\uu[s]{Y}{\ell m}(\theta,\varphi)$, with smaller deviations as $\ell$ increases (e.g. see \fref{fig4:plotSWSH12}).
In this regime, we may the use pure spin-weighted spherical harmonics as approximations, as they not change the qualitative behaviour of the $\uu[s]{S}{\ell m}(\theta,\varphi)$.

%----------------------------------------------------------------------------------------

\section{Amplification factor}

Finding non-approximate form to the amplification factor $\uu[\pm 1]{Z}{\ell m}$ requires the numerical solving of the radial \eqref{eq3:radialTeukolskyAdimensional}, which is already in an adimensional form.
We computed the angular eigenvalues beforehand, which depend on the mode $(\ell,m)$ as well as the coupling $c=a\omega$.
Additionally the equation depends on the BH parameters $(M,J)$ and $\omega$ explicitly, but it is possible to normalize all variables so that we only need to specify $(\mathscr{J}, \ell, m, \bar{\omega})$, where $\mathscr{J}=a/M=J/M^2$.
We choose to work with barred frequencies because $\bar{\Omega}_H = \mathscr{J}/2$, which makes it easier to numerically select superradiant modes.

We need to obtain numerical interpolations for $\uu[\pm 1]{R}{\ell m}$, by integrating the solution outwards from the horizon, at $x=0$, up to a sufficiently large $x_\infty \gg |\bar{\omega}|^{-1}$. 
The solutions for $s=\pm 1$ contain the all the EM field information, but they have different asymptotic behaviors.
For $\phi_0$, \eqref{eq3:asymptoticR} tells us that the ingoing coefficient tends to overshadow the outgoing coefficient, while the opposite occurs for $\phi_2$.
This way seems natural to try to solve the both equations, where we can obtain $\mathscr{Y}_\mathrm{in}$ and $\mathscr{Z}_\mathrm{out}$ separately (\tref{tb3:approximatedRsolutionsYZ}).

Knowing the irregularities of the solution at the horizon, we propose the ansatz
\begin{align}
	\label{eq4:numericalRansatz}
	\uu[\pm 1]{R}{\ell m} = (r_+)^{\mp 1} \, x^{ \mp 1 - i \varpi / \tau} f_{\pm}(x) ~,
\end{align}
where $f(x)$ is a new function obeys a regular second-order differential equation.
Thus we need to set two initial conditions at the horizon, $f_{\pm}(0)$ and $f_{\pm}{\!}'(0)$.
We expect to $|f_\pm(x)|$ to become approximately constant to large $x$, because this
form is written in way that also matches the behavior of the radial function at infinity, $\uu[\pm 1]{R}{\ell m}\sim r^{\mp 1}$.

Comparing \eqref{eq4:numericalRansatz} with the asymptotic form at large $x$ as well as near the horizon, we obtain
\begin{align}
	\label{eq4:YinZoutFInf}
	\begin{split}
		r_{+} \tau \,\frac{\mathscr{Y}_\mathrm{in}}{\mathscr{Y}_\mathrm{hole}} &= \frac{f_{+}(x_\infty)}{f_{+}(0)} \exp\left[ - i \bar{\omega} x_\infty - i \bar{\omega} (2-\tau)\log(x_\infty)   -i \frac{\varpi}{\tau} \log(x_\infty) \right] ~, \\[0.15cm]
		\frac{1}{r_{+} \tau} \,\frac{\mathscr{Z}_\mathrm{out}}{\mathscr{Z}_\mathrm{hole}} &= \frac{f_{-}(x_\infty)}{f_{-}(0)} \exp\left[ + i \bar{\omega} x_\infty + i \bar{\omega} (2-\tau)\log(x_\infty) -i \frac{\varpi}{\tau} \log(x_\infty) \right] ~.
	\end{split}
\end{align}
If both solutions are normalized such that $f_{\pm}(0) = 1$, then we have to deal with the relative normalization of $\mathscr{Z}_\mathrm{hole}/\mathscr{Y}_\mathrm{hole}$. We can obtain such ratio in terms of known parameters by considering \eqref{eq3:separationB} at $x\simeq0$,
\begin{align}
	(r_{+} \tau)^2 \,\frac{\mathscr{Z}_\mathrm{hole}}{\mathscr{Y}_\mathrm{hole}} = -\frac{\mathscr{B} \tau^2 }{2 \varpi (i \tau + 2 \varpi)} ~.
\end{align}
Therefore to compute the amplification factor we use ($f_{\pm}(0)=1$)
\begin{align}
	\label{eq4:ZFPlusMinus}
	\uu[\pm 1]{Z}{\ell m} = \frac{\mathscr{B}^2 \tau^4 }{4 \varpi^2 (\tau^2 + 4 \varpi^2)} \left| \frac{f_{-}(x_\infty)}{f_{+}(x_\infty)} \right|^2 - 1 ~.
\end{align}

Another way of dealing with the relative normalization would be to select different initial conditions at the horizon $x=0$.
We could cancel the relative normalization of $\mathscr{Z}_\mathrm{hole}/\mathscr{Y}_\mathrm{hole}$ if we would set any normalization which results in $f_{-}(0)/f_{+}(0) = -\mathscr{B} \tau^2 /[2 \varpi (i \tau + 2 \varpi)]$, eliminating the dependence of $\mathscr{B}$, $\tau$, $\varpi$ in \eqref{eq4:ZFPlusMinus}.

The differential equation obtained from substituting \eref{eq4:numericalRansatz} into \eqref{eq3:radialTeukolskyAdimensional} is identically zero for $x=0$.
Therefore no matter what initial conditions set for $f_{\pm}{\!}'$ the system would not evolve due to stiffness, which makes the step size of the integrator effectively zero.
The usual solution for stiff differential equation is to start the solver a small distance from the horizon $\epsilon>0$.
We adjust the initial conditions by substituting the series expansion of $f_{\pm}(x) = \sum_{n=0}^{N_H} a_n x^n$ in the radial equation, discarding a terms higher than $\mathscr{O}(x^{N_H})$ and obtaining the coefficients $a_n \propto a_0$, $1 \le n \le N_H$. 
Therefore we may set the initial conditions as
\begin{align}
	f_{\pm}(\epsilon) = f_{\pm}(0) \sum_{n=0}^{N_H} \left(\frac{a_n}{a_0}\right) \epsilon^n ~,\qquad f_{\pm}{\!}'(\epsilon) = f_{\pm}(0) \sum_{n=1}^{N_H} \left(\frac{a_n}{a_0}\right) \epsilon^n ~,
\end{align}
$f_{\pm}(0)=1$ are the original horizon conditions consider.
We found $\epsilon=10^{-12}$, $N_H=6$ and $x_\infty = 200 \times 2\pi/|\bar{\omega}|$ working perfectly for the ``NDSolve'' integrator.
Effectively we will have $|f_{\pm}{\!}'(\epsilon)|\simeq \epsilon$, but this contribution is sufficient to remove stiffness from the system and has important contributions in case of extremal BHs ($\mathscr{J}\to 1$).

This previous method requires us to call the integrator twice which is not very effective numerically.
Exploring the conservation of the wronskian (conserved current) of \eqref{eq3:D2PlusVeff}, we can obtain
\begin{align}
	\label{eq4:InOutHole}
	\frac{\dd E_\mathrm{out}}{\dd t} - \frac{\dd E_\mathrm{in}}{\dd t} = \frac{\dd E_\mathrm{hole}}{\dd t} ~,
\end{align}
which simply states total energy conservation.
Therefore we can rewrite \eqref{eq3:Zdef} only using \emph{hole-in} ratio, thus we are able to get the amplification factor with just with the $s=+1$ solution,
\begin{align}
	\label{eq4:ZFPlusPlus}
	\uu[\pm 1]{Z}{\ell m} = - \frac{\bar{\omega} \tau^2}{\varpi} \left|\frac{f_{+}(0)}{f_{+}(x_\infty)}\right|^2 ~.
\end{align}
If we use the \emph{out-hole} ratio in the amplification factor we only need to solve for $s=-1$,
\begin{align}
	\label{eq4:ZFMinusMinus}
	\uu[\pm 1]{Z}{\ell m} = - \left( 1 + \frac{\mathscr{B}^2 \tau^2}{4 \bar{\omega} \varpi(\tau^2 + 4 \varpi^2)} \left|\frac{f_{-}(x_\infty)}{f_{-}(0)}\right|^2 \right)^{-1} ~.
\end{align}
Results from computing the mode $\ell=m=1$ at $\mathscr{J}=0.9999$, we obtain a maximum amplification of about $4.36\%$ for a frequency of about $\omega M \simeq 0.436$.
Modes in the region \eref{eq3:superradiance} have always $\uu[\pm 1]{Z}{\ell m}>0$, but decreases heavily in orders of magnitude as $\ell$ increases (\fref{fig4:logZ}).
\begin{figure}[h]
	\centering
	\vspace{0.2cm}
	\begin{subfigure}[c]{0.6\textwidth}
        \includegraphics[width=\textwidth]{precisionZ1}
	\end{subfigure}
	\caption{Amplification factor of an extremal BH ($\mathscr{J}=0.9999$) for modes with $\ell=1$. In figure, superradiance occurs only for $m=1$ as predicted.}
	\label{fig4:plotZ53}
\end{figure}

Thus we have three ways of computing $\uu[\pm 1]{Z}{\ell m}$, which only two of them are independent.
We rename the these different forms in Eqs. \eref{eq4:ZFPlusPlus}, \eref{eq4:ZFMinusMinus}, \eref{eq4:ZFPlusMinus} as $Z^{(1)}$, $Z^{(2)}$, $Z^{(3)}$, respectively.
We can rearrange the expressions, so that we have
\begin{align}
	Z^{(3)} = Z^{(1)} \left[ 1 + \frac{1}{Z^{(2)}} \right] - 1 ~.
\end{align}
It is expected that if the amplification factors based only on a single solution are approximately equal, then the same would be true when considering the third factor, which uses both solutions.
But from a better look at \fref{fig4:plotZ53} we can see that this fact is not true, especially in higher values of $\ell$.
\begin{figure}[h]
	\centering
	\vspace{0.2cm}
	\begin{subfigure}[c]{0.6\textwidth}
        \includegraphics[width=\textwidth]{noPrecisionZ53}
	\end{subfigure}
	\caption{Log plot demonstrating error propagation for $\uu[\pm 1]{Z}{53}$ when computing the factor using both numerical solutions for the radial part of $\phi_0$ and $\phi_2$.}
	\label{fig4:plotZ53}
\end{figure}
Somehow it appears that we are not able to compute the ratio of $\mathscr{Y}_\mathrm{in}$ and $\mathscr{Z}_\mathrm{out}$ with enough accuracy, probably because the large values that $f_{\pm}(x_\infty)$ take do not hold the necessary precision to perform the division.

Picking carefully a superradiant frequency. For larger values of $\ell$ (with small $m$) the gain/loss is practically zero.
Still we have huge differences in the order of magnitude of the amplification factor when comparing results from using only one solution and using both.
Since two different equation are numerically solved, it will always be a discrepancy, $Z^{(2)} = Z^{(1)} (1+\eta)$, with $\eta$ very small.
The problem is that this error is propagated in absolute value, $Z^{(3)} \simeq Z^{(1)} - \eta$.
For example, when $(\mathscr{J}, \ell, m, \bar{\omega})=(0.9999,5,3,0.1)$ we have $\eta \simeq -0.003$ and $Z^{(1)}\sim 10^{-20}$, which implies that when using both solutions we have a discrepancy of a factor of $10^{17}$.
Therefore we cannot use the expression $Z^{(3)}$ to compute the amplification factor, when we have $\eta \gg Z^{(1)} , Z^{(2)}$.

We go a step further to increase the numerical precision in this problem by considering higher order terms in the asymptotic expansion of \eqref{eq3:asymptoticR}.
Separately, we will substitute both two asymptotic series in \eqref{eq3:radialTeukolskyAdimensional}, one for the ingoing part and another for the outgoing.
Together they have the form
\begin{align}
	\label{eq4:radialIOseries}
	\uu[-1]{R}{\ell m}(r) &= e^{-i \bar{\omega} x}  \, x^{ - 1 - i (2-\tau) \bar{\omega} } \sum_{n=0}^{N_\infty} I_n \, x^{-n} + \, e^{-i \bar{\omega} x} x^{ 1 + i (2-\tau) \bar{\omega} } \sum_{n=0}^{N_\infty} O_n \, x^{-n} ~,
\end{align}
where we identify $I_0 \,r_{+} = \mathscr{Z}_\mathrm{in}/ \mathscr{Z}_\mathrm{hole}$ and $O_0/r_{+} = \mathscr{Z}_\mathrm{out}/\mathscr{Z}_\mathrm{hole}$.
Although we have chosen the $s=-1$ solution, the same procedure can be done for $s=+1$, because when using a higher order expansion, both ingoing and outgoing coefficients are present.

Firstly, we directly substitute the series into \eqref{eq3:radialTeukolskyAdimensional}, neglecting terms above $\mathscr{O}(x^{N_\infty})$ and grouping the exponentials terms, in order to obtain $I_n\propto I_0$, $O_n\propto O_0$ ($1 \le n \le N_\infty$), exactly like the series used above to define boundary conditions at the horizon.
Secondly, substitution of the numerical ansatz \eref{eq4:numericalRansatz} in the LHS of the previous equation, together with it's derivative, we have a system two linear equations, which in the limit of large-$x$ limit allows to determine
\begin{align}
	\frac{1}{r_+} \,\frac{\mathscr{Z}_\mathrm{in}}{\mathscr{Z}_\mathrm{hole}} = I_0\Big( f_{-}(x_\infty), \,f_{-}{\!}'(x_\infty) \Big) ~,\qquad r_{+} \frac{\mathscr{Z}_\mathrm{out}}{\mathscr{Z}_\mathrm{hole}} = O_0\Big( f_{-}(x_\infty), \, f_{-}{\!}'(x_\infty) \Big) ~.
\end{align}
Lastly, we may use previous expression \eref{eq4:InOutHole} to compute $\uu[\pm1]{Z}{\ell m}$ only using one of the coefficient, instead of using \eqref{eq3:amplificationBAoutAin}. 
This new method solves some of the precision problems from the first method when using both $\phi_0$ and $\phi_2$, for a smaller $x_\infty = 80 \times 2 \pi / |\bar{\omega}|$ and $N_\infty = 10$, with the same $\epsilon = 10^{-12}$.
\begin{figure}[!t]
	\centering
	\vspace{0.2cm}
	\begin{subfigure}[c]{0.9\textwidth}
        \includegraphics[width=\textwidth]{precisionLogZ}
	\end{subfigure}
	\caption{MoreZPlots}
	\label{fig4:logZ}
\end{figure}

%----------------------------------------------------------------------------------------

\cleardoublepage 
%% !TEX root = ../main.tex

\chapter{Superradiant scattering of plane waves} % Main chapter title
\label{Chapter5}

The prediction of EM and gravitational radiation amplification was a surprising prediction of Einstein's theory of gravitation in Kerr geometry.
A method for direct or indirect observation of this process would provide a probe for rotating BHs and thus it would constitute an important test of GR in regions of extreme gravity.
We owe to find in which conditions the scattering an EM wave composed of multiple mode $(\omega,\ell, m)$ provides information to probe the occurrence of superradiance in a BH.
We know that each mode will be independently amplified/attenuated as shown above. The challenge is to predict if superradiant scattering occurred given the wave composed of these modified modes. 

Having shown that superradiance occurs for small frequencies we need to find astrophysical sources that emit EM waves.
Binary systems of rotating neutron stars and BH may exhibit the necessary conditions for superradiant scattering.
These objects, also known as pulsars, possess a strong magnetic field with magnetic dipole moment typically not aligned with the rotation axis.
Obviously the magnetic field configuration of a neutron star can be very complicated but its main properties are best described by the \emph{oblique-rotator} model, which considers only the leading order in the multipolar expansion, \emph{i.e.} a magnetic moment dipole
\begin{align}
    \mathbf{m}_P = \frac{m_P}{2} \left[ e^{-i \omega t} \sin\alpha_S ( \mathbf{\hat{x}} \pm i \mathbf{\hat{y}}) + \cos\alpha_S \,\mathbf{\hat{z}} \right] + \text{ c.c.} ~,
\end{align}
where $\omega$ is the frequency of rotation.
The upper (lower) sign corresponds to a neutron star co-rotating (counter-rotating) with the BH.
The moment $\mathbf{m}_P$ makes an angle $\alpha_S$ with respect with the rotation axis, resulting in the precession of the pulsars' magnetic axis, which produces a periodic focused beam of EM radiation.
This periodicity is so precise that makes pulsars ideal for measuring time differences in GR tests.
Neutron stars usually have a millisecond period producing radiation of a few kHz, which is in range of the superradiant frequencies of a typical stellar mass extremal BH.  

(IN NEED OF A FIGURE WITH BH AND PULSAR)

We will focus on scattering of incident plane waves, which means we will consider a source that is far away from the BH.
More specifically, we consider incident plane waves from a magnetic dipole source whose electric and magnetic radiation fields, which are found in standard textbook, are given by
\begin{align}
    \begin{split}
        \mathbf{E} &= \frac{\mu_0}{8\pi}
        \frac{e^{i\omega|\mathbf{r}-\mathbf{r}_S|}}{|\mathbf{r}-\mathbf{r}_S|} \left(
        \frac{\mathbf{r}-\mathbf{r}_S}{|\mathbf{r}-\mathbf{r}_S|} \times
        \frac{\dd^2 \mathbf{m}_P}{\dd t^2} \right) + \text{ c.c.} \\
        & \simeq  - \frac{\mu_0}{8\pi} \frac{e^{i \omega L}}{L}
        e^{i \mathbf{k} \cdot \mathbf{r}} \left( \mathbf{\hat{r}}_S \times
        \frac{\dd^2 \mathbf{m}_P}{\dd t^2} \right) + \text{ c.c.} ~,
    \end{split}
\end{align}
where $\mathbf{k}= -\omega \mathbf{\hat{r}}_S$ and $L=|\mathbf{r}_S|$ is the distance to the source to the BH.
This approximation is valid for when $r=|\mathbf{r}|$ is large compared with the radiation wavelength and the physical dimension of the dipole.
Additionally, in the last step we require that $r \ll L$.  With the similar procedure the magnetic field can be obtain using $\mathbf{B}\simeq - \mathbf{\hat{r}}_S \times \mathbf{E}$.
Thus, when sufficiently far away from the dipole the radiation can be seen as plane waves propagating in the direction of $(- \mathbf{\hat{r}}_S) = (\sin\theta_0 \cos\varphi_0, \sin\theta_0 \sin\varphi_0, \cos\theta_0)$.

%----------------------------------------------------------------------------------------

\section{Harmonics decomposition}

By projecting the complex representation of $\mathbf{E}$ using the perpendicular directions $\mathbf{e}_{\hat{\theta}_0}$ and $\mathbf{e}_{\hat{\varphi}_0}$, we can obtain the two EM field polarizations,
\begin{align}
    \epsilon_{\theta} = \frac{\mu_0 \,m_P \,\omega^2 \sin\alpha_S}{8 \pi} \frac{e^{i \omega L}}{L} e^{\pm i \varphi_0} \cos\varphi_0 ~,\qquad
    \epsilon_{\varphi} = \pm i\, \frac{\mu_0 \,m_P \,\omega^2 \sin\alpha_S}{8 \pi} \frac{e^{i \omega L}}{L} e^{\pm i \varphi_0} ~,
\end{align}
To use results from previous chapters it is convenient to write the EM degrees of freedom using the NP formalism.
There is no need for computing both NP scalars, since we know that the result will very similar.
Asymptotically we have $\mathfrak{m}\sim \partial_\theta + i \csc\theta \,\partial_\varphi$, thus we may show that $\phi_0 = (\mathbf{E} + i \mathbf{B}) \cdot (\mathbf{e}_{\hat{\theta}} + i \mathbf{e}_{\hat{\varphi}} )/\sqrt{2}$ and $2\phi_2 = (\mathbf{E} + i \mathbf{B}) \cdot (\mathbf{e}_{\hat{\theta}} - i \mathbf{e}_{\hat{\varphi}} )/\sqrt{2}$.
Together with the dipole field approximation, the validity of this expansion is when $r_{+} \ll r \ll L$.
Following the work done in \cref{Chapter4}, we will keep using $\phi_2$ as our primary scalar as it is the indicated for studying outgoing radiation.
Thus, we may write
\begin{align}
    \label{eq5:phi2plane}
    \phi_2{}^{(\mathrm{plane})} = - \frac{2 \pi i }{3} \left( \epsilon_R \,e^{-i \omega t + i \mathbf{k}\cdot\mathbf{r}} + \epsilon_L^* \,e^{i \omega t - i \mathbf{k}\cdot\mathbf{r}} \right) \sum_{m=-1}^{+1} \uu[-1]{Y}{1,m}(\theta_0,\varphi_0)^{*} \uu[-1]{Y}{1,m}(\theta,\varphi) ~,
\end{align}
where $\mathbf{\hat{k}}\equiv(\theta_0,\varphi_0)$ and $\mathbf{\hat{r}}\equiv(\theta,\varphi)$ are the directions of incidence and observation, respectively.
This result can be easily obtained by explicitly expanding the harmonics sum.
The left and right polarizations are defined as
\begin{align}
    \begin{split}
        \epsilon_R &= \frac{\epsilon_{\theta} - i \epsilon_{\varphi}}{\sqrt{2}}
        = \mp \frac{\mu_0 \,m_P \,\omega^2 \sin\alpha_S}{2\sqrt{6\pi}} 
        \frac{e^{i\omega L}}{L} \,\uu[-1]{Y}{1,\pm 1}(\theta_0,\varphi_0) ~, \\
        \epsilon_L^* &= \frac{\epsilon_{\theta}^* - i \epsilon_{\varphi}^*}{\sqrt{2}} =
        \pm \frac{\mu_0 \,m_P \,\omega^2 \sin\alpha_S}{2\sqrt{6\pi}}
        \frac{e^{-i\omega L}}{L} \,\uu[-1]{Y}{1,\mp 1}(\theta_0,\varphi_0) ~.
    \end{split}
\end{align}

It may seem that $\phi_2$ for a plane wave is approximately describe using only $\ell=1$ harmonics, but we must not forgot the angular dependence in
\begin{align}
    e^{i \mathbf{k} \cdot \mathbf{r}} = 4 \pi \sum_{\ell,m} i^\ell j_\ell(\omega r) \uu{Y}{\ell m}(\theta_0,\varphi_0)^{*} \uu{Y}{\ell m}(\theta,\varphi) ~,
\end{align}
whose decomposition in terms of $s=0$ spherical harmonics is well-known \cite{Jackson1998}, where $j_\ell(z)$ corresponds to the spherical Bessel function of the first kind.
Substitution into the result in a superposition of different spin-weight harmonics and after grouping $\mathbf{\hat{k}}$ and $\mathbf{\hat{r}}$ terms these can be expanded using Clebsh-Gordon coefficients.
\begin{align}
    \label{eq5:phi2planeExpanded}
    \begin{split}
        \phi_2{}^{(\mathrm{plane})} &= - 2 \pi \,\epsilon_R \,e^{-i \omega t} \,\sum_{\ell,m} \left(
        \sum_{n=\ell-1}^{\ell+1} i^{n + 1} \,j_n(\omega r) \,\frac{2n + 1}{2\ell + 1} 
        \left|\langle n,0 ; 1,1 | \ell,1 \rangle\right|^2 \right)
        \uu[-1]{Y}{\ell m}(\mathbf{\hat{k}})^{*} \uu[-1]{Y}{\ell m}(\mathbf{\hat{r}}) \\
        & \qquad + ( \,\epsilon_R \to \epsilon_L^*, \,\omega \to -\omega \,) \\[0.15cm]
        &\sim + 2 \pi \,\epsilon_R \,e^{-i \omega t} \,\sum_{\ell,m} \left(
        -\frac{1}{2 \omega} \frac{e^{i \omega r}}{r} 
        + (-1)^\ell \,\frac{\ell(\ell+1)}{8 \omega^3} \frac{e^{-i \omega r}}{r^3} \right)
        \uu[-1]{Y}{\ell m}(\mathbf{\hat{k}})^{*} \uu[-1]{Y}{\ell m}(\mathbf{\hat{r}}) \\
        & \qquad + (\,\epsilon_R \to \epsilon_L^*, \,\omega \to -\omega \,) ~.
    \end{split}
\end{align}
The expression for $\phi_0$ is very similar, changing the coefficients of $e^{\pm i \omega r}$ accordingly so they obey Eqs. \eref{eq3:separationB} and \eref{eq3:separationBdagger} when $r \gg r_{+}$, replacing $\uu[-1]{Y}{\ell m}(\mathbf{\hat{r}})\to\uu[+1]{Y}{\ell m}(\mathbf{\hat{r}})$.

We have shown that even a simple plane wave is composition of modes with positive and negative frequencies modulated by the left and right polarizations, respectively, which are proportional to $\uu[-1]{Y}{1, \pm 1}(\theta_0, \varphi_0)$.
According to condition \eref{eq3:superradiance}, modes with either $\omega>0$, $m>0$ or $\omega<0$, $m<0$ can be amplified.
The position of the source modulates the incident wave changing changing its mode composition. 
Therefore if the plane wave source co-rotates with the BH, when $\theta_0 \to 0$ the positive frequencies dominate because $\epsilon_L^*\to 0$, coinciding with the region were $m>0$ harmonics predominate. Analogously, when $\theta_0\to\pi$ negative frequencies dominate as $\epsilon_R \to 0$.
In the other hand, when considering counter-rotation and the incidence at one of the poles, harmonics with $m \omega >0$ have null coefficients so those modes are never amplified \cite{Rosa2016}.
More specifically, when we have exactly $\theta_0=0$ ($\theta_0=\pi$) the modes $m=1$ ($m=-1$) are the only non-zero contributions of the EM wave if and only if the source co-rotates with the BH, while other $m$ modes vanish.

(CAN BE USED A FIGURE TO SHOW LEFT/RIGHT DOMINANCE)

%----------------------------------------------------------------------------------------

\section{Scattering theory}

We understand that we(US HUMANS) have limited observational capabilities and only have access to given direction of observation for this hypothetical binary system.
If it were possible to map the entire scattered wave with enough detail we could in principle extract and compare each mode with the ones of the emitted wave. For this analysis we would only need to know the global gain/loss factor, given by $\uu[\pm1]Z{\ell m}$. Therefore we will resort to scattering theory of waves to study the angular effects of superradiance.

Intuitively, it is understood that only a small part of the incident wave will be scattered by the BH.
The scattered part together with the indent wave produce a characteristic interference pattern pattern.
In order to differentiate the scattered wave we need to remove the background incident plane wave.
Scattering theory assumes that we may write
\begin{align}
    \label{eq5:scattering}
    \phi_2 - \phi_2{}^{(\mathrm{plane})} = f(\theta,\varphi) \frac{e^{i \omega (r_{*}-t)}}{r} + (\omega\to-\omega, \,f \to g) ~,
\end{align}
where $\phi_2$ is written similarly with coefficients $\mathscr{Z}_\mathrm{out}$ and $\mathscr{Z}_\mathrm{in}$ obtained numerically in \cref{Chapter4}.

Up to this point we used the approximation of plane wave first introduced in \eref{eq5:phi2plane}, which can only be used in flat space. The fact is that this approximation does not take into account the long-range behaviour of Kerr's gravitational field, which goes is of $\mathscr{O}(\tfrac{1}{r})$ as obtained in \eref{eq3:asymptoticVeff}.
We know that from the asymptotic form of the radial function that this can be bypassed by a logarithmic phase-correction in the exponential, substituting $r\to r_{*}$.
We can match the ingoing parts of $\phi_2$ and $\phi_2{}^{(\mathrm{plane})}$ so the scattered part only have outgoing part, obtaining
\begin{align}
    \label{eq5:scatterFexpressionYY}
    f(\theta,\varphi) = - \frac{\pi \,\epsilon_R}{\omega} \sum_{\ell,m} \left[
    (-1)^{\ell+1} \frac{\ell(\ell+1)}{4 \omega^2 }
    \frac{\mathscr{Z}_\mathrm{out}}{\mathscr{Z}_\mathrm{in}} - 1 \right]
    \uu[-1]{Y}{\ell m}(\mathbf{\hat{k}})^{*} \uu[-1]{Y}{\ell m}(\mathbf{\hat{r}}) ~.
\end{align}
A similar expression is obtained for $g(\theta,\varphi)$ proportional to $\epsilon_L^*$.

The long-range effect of the background is independent of the BH rotation (also in Schwarzschild), \emph{i.e.} we must not mistake the spherical approximation with long-range effects of the effective gravitational potential.
Approximation of plane waves as described in \eref{eq5:phi2plane} also discards the effects of the BH rotation, which is the reason ats the plane wave is decomposed using spherical harmonics $\uu[-1]{Y}{\ell m}(\mathbf{\hat{r}})$ instead of SWSHs.
We can also recall that the mode factor in \eqref{eq5:scatterFexpressionYY} is very similar the expression \eref{eq3:amplificationBAoutAin}, derived in \cref{Chapter3}.
We see that for $a\omega \to 0$,
\begin{align}
    \label{eq5:phaseFactor}
    (-1)^{\ell+1} \frac{\mathscr{B}}{4 \omega^2} \frac{\mathscr{Z}_\mathrm{out}}{\mathscr{Z}_\mathrm{in}} \simeq (-1)^{\ell+1} \frac{\ell(\ell+1)}{4 \omega^2} \frac{\mathscr{Z}_\mathrm{out}}{\mathscr{Z}_\mathrm{in}} ~,
\end{align}
remembering that $\mathscr{B} = \left[ (\uu[\pm 1]{\mathscr{E}}{\ell m})^2 - 4 a^2 \omega^2 + 4 m a \omega \right]^{1/2}$.
An argument could be made state that the latter expression fo the coefficient is the correct instead of the one in \eqref{eq5:scatterFexpressionYY}, but this approximation is good enough when considering superradiant frequencies $|\omega| \simeq 0.4 \Omega_H$ for a typical stellar mass extremal BH (see FIGS SWSH/EVS).

%----------------------------------------------------------------------------------------

\section{Phase shifts}

If to assume co-rotation of the source with optimal incidence at $\theta_0=\varphi_0=0$ we will only need to compute $f(\theta,\varphi)$, since $\epsilon_L^*=0$.
This assumption eases the need to compute modes other then $m=1$.
Assuming that we truncate the expansion \eref{eq5:scatterFexpressionYY} at some $\ell=\ell_{\max}$, this reduces the number of necessary harmonics in $\ell_{\max}(\ell_{\max} + 1)$.
Proceeding with the sum in FIGURE.XXX it appears that the partial wave sum is slowly convergent and even divergent near $\theta \simeq 0$.

(FIGURE WITH SUM HERE, WITHOUT REGULARIZATION)

This problem is due to the long-range effect of gravitational potential of BHs. This problem is known problem in Coloumb scattering. Central potentials falling as $1/r$ do not have effect on the global amplitude of the have but the scattered wave has phase shifts in each of the modes coefficients, producing a divergence at $(\theta,\varphi)=(\theta_0,\varphi_0)$.
This result appears strange at first, but we must remember that, being a complete space of functions, the harmonics obey $\sum_{\ell m} \uu[-1]{Y}{\ell m}(\mathbf{\hat{k}})^{*} \uu[-1]{Y}{\ell m}(\mathbf{\hat{r}})) = \delta(\cos\theta - \cos\theta_0) \delta(\varphi - \varphi_0)$.

In order to regularize the sum for $f(\theta,\varphi)$, it is convenient to separate it in two terms,
\begin{align}
    \label{eq5:fNfD}
    f(\theta,\varphi) = f_\mathrm{N}(\theta,\varphi) + f_\mathrm{D}(\theta,\varphi) ~,
\end{align}
$f_\mathrm{N}(\theta,\varphi)$ carries all the scattering information about the Newtonian effects of the long-range $1/r$ (Coulomb) potential.
It can be written as
\begin{align}
    f_\mathrm{N}(\theta,\varphi) = - \frac{\pi \,\epsilon_R}{\omega}
    \sum_{\ell,m} \left( e^{2 i \delta_N } - 1 \right)
    \uu[-1]{Y}{\ell m}(\mathbf{\hat{k}})^{*} \uu[-1]{Y}{\ell m}(\mathbf{\hat{r}}) ~,
\end{align}
where the phase shifts are \cite{Futterman1988}
\begin{align}
    e^{2 i \delta_N } = \frac{\Gamma(\ell + 1 - 2 i M \omega)}{\Gamma(\ell+1 + 2 i M \omega)} ~.
\end{align}
Assuming an incidence of $\theta_0=0$, summing the series leads to a similar result as the Rutherford elastic scattering in a Coulomb potential, $|f_\mathrm{N}(\theta,0)| \propto 1/\sin^{4}(\theta/2) \sim 1/\theta^4$, which appears to explain the divergence at $\theta=0$. 

On the other hand, the $f_\mathrm{D}(\theta,\varphi)$ encloses all the information regarding the main scattering effects, including superradiance.
From \eqref{eq5:fNfD}, simple algebra states that
\begin{align}
    f_\mathrm{D}(\theta,\varphi) = - \frac{\pi \,\epsilon_R}{\omega}
    \sum_{\ell,m} \left( \frac{\ell(\ell+1)}{4 \omega^2} \left| \frac{\mathscr{Z}_\mathrm{out}}{\mathscr{Z}_\mathrm{in}} \right| e^{2 i \delta } - e^{2 i \delta_\mathrm{N} } \right)
    \uu[-1]{Y}{\ell m}(\mathbf{\hat{k}})^{*} \uu[-1]{Y}{\ell m}(\mathbf{\hat{r}}) ~,
\end{align}
where we define $\delta = \mathrm{arg} \left[ (-1)^{\ell+1} \,\mathscr{Z}_\mathrm{out}/\mathscr{Z}_\mathrm{in} \right]$.
In order of this sum to converge two things


\cleardoublepage 
%% !TEX root = ../main.tex

\chapter{Discussion and future work} % Main chapter title
\label{Chapter6}

In this work we soughed to understand the effect of superradiance scattering in EM waves on Kerr BHs.
The objective was to demonstrate if superradiance occurred in case of scattering of a EM wave radiated from a physically realistic source by a rotating BH.
General waves are a superpositions of modes ($\omega, \ell, m$), which we know that occurs superradiance when $\omega(\omega - m \Omega_H)<0$.
In this region the modes are either reflected or amplified, with the maximum amplification in the case of EM waves being of approximately 4.4\%.
This occurs when the BH is extremal, $a\to M$ and on the lowest multipole $\ell=1$, with the percentage dropping quickly to zero as $\ell$ increases.
Modes with large $|\omega|$ are quickly absorbed by the BH since they can ``cross'' the gravitational potential barrier, reaching the event horizon.
To compute these amplification factor we need: (i) to compute the angular eigenvalues in order to substitute into the radial equation; (ii) obtain the coefficients of $\mathscr{Z}_\mathrm{in}$ and $\mathscr{Z}_\mathrm{out}$, by solving the radial equation.
The second step we devised way of not rewriting the radial equation using the tortoise coordinate $r_*$, removing the singularities by considering a cleaver ansatz, without sacrificing precision or computational speed.
We showed that is possible to find the amplification factor of each mode either using $|\mathscr{Z}_\mathrm{in}|^2$ or $|\mathscr{Z}_\mathrm{out}|^2$ our both.
In this work we discuss why it is more advantageous to write the gain/loss factor only using only one of the previous coefficients.
Although the routine is defined for EM perturbations, it can be quickly updated to accommodate GW perturbations for future work studies.
The resultant complex coefficients also play an integral part in the computation of phase shifts, which are present in for each mode.
Particularly, for a plane wave with superradiant frequency we know that most of the multipole modes are deflected with no change in the amplitude.
This effect is characteristic of long-ranged potential that fall as $1/r$.
When observing the BH from a particular direction, the effect of these phase shifts will dominate and conceal the effects of superradiance.
In principle we could remove these interference effects, integrate all the solid angle by taking making observations from all directions, leaving only global amplification/absorption effects, but we do not know if that will be ever possible.
Therefore we need to find a way to isolate the lower superradiant modes. A possible idea would be use a source orbiting the BH, with possibility for variation of distance to the BH and incidence angle, due to the chaotic orbits of the Kerr geometry.  


%----------------------------------------------------------------------------------------
%	THESIS CONTENT - APPENDICES
%----------------------------------------------------------------------------------------

\addtocontents{toc}{\vspace{2em}} % Add a gap in the Contents, for aesthetics

\appendix % Cue to tell LaTeX that the following 'chapters' are Appendices

% !TEX root = ../main.tex

\label{AppendixA} % For referencing this appendix elsewhere, use \ref{AppendixA}
\chapter{Spin-weighted spherical harmonics} % Main appendix title

SWSHs play an important role BH physics and was first introduced by Teukolsky when considering non-scalar wave perturbations on a Kerr background, obtaining a separable master equation in four dimensions. After the usual change of coordinates, the polar differential equation goes as 
\begin{equation}
	\frac{1}{S} \frac{\dd}{\dd x} \left( (1-x^2) \, \frac{\dd S}{\dd x} \right) + (c x)^2 - 2 c s x  -\frac{(m + s x)^2}{1 - x^2} + s  = - \lambda
\label{eqA:diffSWSH}
\end{equation}
with $x=\cos\theta$, where $\lambda$ is the eigenvalue for a given SWSH solution. Periodic boundary conditions on the azimuthal wave function constrains $m$ to the integers.   

\section{Connection with spheroidal harmonics}

By setting $s=0$ (scalar) and $c=0$ (spherical), then it's clear that~\eqref{eqA:diffSWSH} appears as a generalization of the spherical harmonics equation. In this last case, the solution are given by the associated Legendre polynomials, $P^m_\ell (x)$, for which the eigenvalue is $\ell(\ell+1)$, restricted to the condition of $|m| \le \ell$. The closed form for spherical harmonics, after normalization, is
\begin{equation}
	\sts[0]{Y^m}{\ell}(x)= (-1)^{m} \sqrt{\frac{(2\ell +1)}{4\pi} \frac{(\ell -m)!}{(\ell +m)!}} \, P_\ell^m (x)
\label{eqA:SH}
\end{equation}
where $P_{\ell}^{m}$ are the associated Legendre polynomials which can be obtained using the famous Rodrigues' formula.

\section{Spin raising/lowering differential operators}

\section{Generalized addition of angular momentum formula}

\section{Some useful harmonics}
%% !TEX root = ../main.tex

\chapter{Additional Newman-Penrose definitions and computations} % Main chapter title
\label{AppendixNPFormalism}

In this appendix we will present important computations using NP formalism in Kerr background using the Kinnersley tetrad defined in~\eref{eq2:kinnerslytetrad}.

\section{Spin coefficients}
\label{AppendixNPSpinCoef}

The spin connection has $24$ components due to the antisymmetry of the two tetrad indices. These are designated using $12$ complex variables,
\begin{equation}
    \begin{alignedat}{4}
        \kappa  =& \gamma_{311} ~,\qquad \varrho =& \gamma_{314} ~,\qquad \varepsilon =& \tfrac{1}{2} (\gamma_{211} + \gamma_{341}) ~, \\
        \sigma  =& \gamma_{313} ~,\qquad \mu     =& \gamma_{243} ~,\qquad \gamma      =& \tfrac{1}{2} (\gamma_{212} + \gamma_{342}) ~, \\
        \lambda =& \gamma_{311} ~,\qquad \tau    =& \gamma_{312} ~,\qquad \alpha      =& \tfrac{1}{2} (\gamma_{214} + \gamma_{344}) ~, \\
        \nu     =& \gamma_{311} ~,\qquad \pi     =& \gamma_{241} ~,\qquad \beta       =& \tfrac{1}{2} (\gamma_{213} + \gamma_{343}) ~.
    \end{alignedat}
\end{equation}
%% !TEX root = ../main.tex

\chapter{Spin-weighted spherical harmonics}
\label{AppendixSWHs}

Spin-weight spherical harmonics \cite{Goldberg1967,Scanio1977,TorresdelCastillo2003} are a generalization of the standard spherical harmonics
found in many well know physical problems such as the hydrogen atom.
They define a set of eigenfunctions which solves the equation
\begin{equation}
	\label{eqA:diffSWSH}
	\frac{1}{\sin\theta} \frac{\dd}{\dd\theta} \left( \sin\theta 
	\frac{\dd \,\uu[s]{Y}{\ell m}}{\dd \theta} \right)
    + \left[ s - \frac{(m + s \cos\theta)^2}{\sin^2\theta} \right] \uu[s]{Y}{\ell m} = - \lambda \;\uu[s]{Y}{\ell m}~,
\end{equation}
with eigenvalues $\lambda= \ell(\ell +1) - s(s+1)$.

These harmonics are complex functions defined on the $S^2$. If take a point in a sphere $(\theta,\varphi)$, we can define a right-handed basis at each point, $\bm{e}_\theta = \partial_\theta$ and $\bm{e}_\varphi = 1/\sin\theta \,\partial_\varphi$, where $\bm{e}_\theta \cdot \bm{e}_\theta = \bm{e}_\varphi \cdot \bm{e}_\varphi = 1$ and $\bm{e}_\theta \cdot \bm{e}_\varphi = 0$.
A given function $f$ defined on $S^2$ is said to have spin-weight $s$ if under the rotation of an angle $\alpha$ of the tangent vectors to the sphere,
\begin{align}
	\label{eqC:spinTransformation}
	\bm{e}_\theta \to \cos\alpha \,\bm{e}_\theta - \sin\alpha \,\bm{e}_\varphi ~, \qquad \bm{e}_\theta \to \sin\alpha \,\bm{e}_\theta + \cos\alpha \,\bm{e}_\varphi ~,
\end{align}
implies that the function transforms as
\begin{align}
	f(\theta, \varphi) \to e^{i s \alpha} f(\theta, \varphi) ~.
\end{align}
In the case of spherical symmetry, $a=0$, we may write the Kinnersly angular vector as $\bm{\mathfrak{m}} = (\bm{e}_\theta + i \,\bm{e}_\varphi)/(\sqrt{2} r^2)$. Under the same transformation, we have $\bm{\mathfrak{m}}\to e^{i\alpha} \bm{\mathfrak{m}}$. From definition \eref{eq3:maxwellNPphi}, since we contract the Maxwell tensor with $\bm{\bar{\mathfrak{m}}}$ once to obtain $\phi_2$, we know that $\phi_2 \to e^{-i \alpha} \phi_2$, thus is has spin-weight $-1$. On the other hand, for gravitational waves the NP scalars $\psi_0$ and $\psi_4$ are double contractions $\bm{\mathfrak{m}}$ and $\bm{\bar{\mathfrak{m}}}$ on the Weyl tensor, respectively. Therefore they are $s=2$ and $s=-2$ quantities, respectively.

All spin-weight spherical harmonics can be obtained using raising and lowering operators on the scalar spherical harmonics. In particular we have that $\uu[0]{Y}{\ell m} = Y_{\ell m}$.
These operators are defined as
\begin{align}
	\begin{split}
		\eth f &= - (\sin{\theta})^s \left\{ \partial_\theta + \frac{i}{\sin{\theta}} \, \partial_\varphi \right\} \left[ (\sin{\theta})^{-s} f \right] 
		= - \left( \partial_\theta + \frac{i}{\sin{\theta}} \, \partial_\varphi - s \cot\theta \right) f ~, \\
		\bar{\eth} f &= - (\sin{\theta})^{-s} \left\{ \partial_\theta - \frac{i}{\sin{\theta}} \, \partial_\varphi \right\} \left[ (\sin{\theta})^{s} f \right]
		= - \left( \partial_\theta - \frac{i}{\sin{\theta}} \, \partial_\varphi + s \cot\theta \right) f ~.
	\end{split}
\end{align}
Is clear from the definition of the operators, that for a function $f$ is a function with spin-weight $s$, then $\eth f$ has spin-weight $s+1$ while $\bar\eth f$ has spin-weight $s-1$, due to an extra $e^{\pm i \alpha}$ factor under the transformation \eref{eqC:spinTransformation}.

Expanding $\eth \bar{\eth}$ we can found the property that for any function $f$ with definite spin-weight, we have
\begin{align}
	\frac{1}{2} (\bar\eth \eth - \eth \bar{\eth} ) f = s f .
\end{align}
This last equation can also be shown using the properties
\begin{align}
	\begin{split}
		\eth ~\uu[s]{Y}{\ell m} = +\sqrt{\ell(\ell+1)-s(s+1)} ~\,\uu[s+1]{Y}{\ell m} ~, \\
		\bar\eth ~\uu[s]{Y}{\ell m} = -\sqrt{\ell(\ell+1)-s(s+1)} ~\,\uu[s-1]{Y}{\ell m} ~.
	\end{split}
\end{align}
We can apply multiple raising and lowering operators to obtain any spherical harmonic, given that $\ell\ge \max\{ |m|, |s| \}$,
\begin{align}
	\begin{split}
	\uu[s]{Y}{\ell m}(\theta,\varphi) &= (-1)^m \sqrt{\frac{2\ell+1}{4 \pi} (\ell+m)! (\ell-m)! (\ell+s)! (\ell-s)!} \\
	&\qquad\qquad \times \sum_{k=0}^{\ell-s} \frac{(-1)^m \left(\sin\tfrac{\theta}{2} \right)^{m+s+2k} \left(\cos\tfrac{\theta}{2} \right)^{2\ell-m-s-2k}}{k!(\ell-m-k)!(\ell-s-k)!(m+s+k)!} e^{i m \varphi}
	\end{split}
\end{align}
For this work, will be useful to list the lowest dipole ($s=-1$, $\ell=1$) spherical harmonics
\begin{align}
	\begin{split}
	\uu[-1]{Y}{1,\pm 1}(\theta,\varphi) &= -\sqrt{\frac{3}{8\pi}} \sin\theta ~,\\
	\uu[-1]{Y}{10}(\theta,\varphi) &= -\sqrt{\frac{3}{16\pi}} (\cos\theta \pm 1) e^{\pm i \varphi} ~,
	\end{split}
\end{align}
while the $s=1$ harmonics can be obtained using properties
\begin{align}
	\begin{split}
		\uu[-s]{Y}{\ell m}(\theta,\varphi)^* &= (-1)^{-s+m} \,\uu[s]{Y}{\ell, -m}(\theta,\varphi) ~, \\
		\uu[-s]{Y}{\ell m}(\pi-\theta,\varphi+\pi)^* &= (-1)^{\ell} \,\uu[s]{Y}{\ell m}(\theta,\varphi) ~.
	\end{split}	
\end{align}
Another possible way of writing the spin-weight spherical harmonics is by using the hypergeometric function,
\begin{align}
	\begin{split}
	\uu[s]{Y}{\ell m}(\theta,\varphi) &= (-1)^m \sqrt{\frac{2\ell+1}{4 \pi} \frac{(\ell+m)! (\ell-m)!}{(\ell+s)! (\ell-s)!}} \left(\sin\tfrac{\theta}{2} \right)^{m+s} \left(\cos\tfrac{\theta}{2} \right)^{2\ell-m-s} \\
	&\qquad\qquad \times \uu[2]{F}{1}\left( m-\ell, s-\ell, m+s+1; - \tan^2\tfrac{\theta}{2} \right) e^{i m \varphi} ~.
	\end{split}
\end{align}

The product of two spin-weighted spherical harmonics with the same argument can be written as a linear combination of other harmonics, admitting a Clebsh-Gordon decomposition,
\begin{align}
	\label{eqC:clebschGordanDecomp}
	\uu[s']{Y}{j' m'} ~\uu[s]{Y}{j m} = \sum_{S,J,M} C_{SJM} ~\uu[S]{Y}{JM} ~,
\end{align}
where
\begin{align}
	\begin{split}
		\label{eqC:clebschGordanCoef}
		C_{SJM} &= (-1)^{j+j'-J} \sqrt{\frac{(2j+1)(2j'+1)}{4\pi(2J+1)}} \\
		&\qquad\qquad \times \langle j',m' ; j, m | J,M \rangle\langle j',s' ; j, s | J, S \rangle \,\delta_{M,m+m'} \,\delta_{S,s+s'} ~,
	\end{split}
\end{align}
with the restriction that the triangle inequality must hold, $|j - j'|\le J \le j+j'$. 

Since these harmonics are generalizations of the standard $s=0$ spherical harmonics, we expect that for each spin-weight $s$ they for an orthogonal and complete set of functions
\begin{align}
	\begin{split}
		\int \dd \Omega ~\,\uu[s]{Y}{\ell' m'}(\theta,\varphi)^* ~\uu[s]{Y}{\ell m}(\theta,\varphi) &= \delta_{\ell\ell'} \,\delta_{mm'} ~,\\
		\sum_{\ell=|s|}^\infty \sum_{m=-\ell}^\ell \uu[s]{Y}{\ell m}(\theta_0,\varphi_0)^* ~\uu[s]{Y}{\ell m}(\theta,\varphi) &= \delta(\cos\theta - \cos\theta_0) \delta(\varphi - \varphi_0) ~,
	\end{split}
\end{align}
so that any spin-weighted $s$ function $f(\theta,\varphi)$ can be written as
\begin{align}
	\begin{split}
		f(\theta,\varphi) = \sum_{\ell=|s|}^\infty \sum_{m=-\ell}^\ell c_{\ell m} ~\uu[s]{Y}{\ell m}(\theta,\varphi) ~,
	\end{split}
\end{align}
so each mode coefficient $c_{\ell m}$ is uniquely defined.

\addtocontents{toc}{\vspace{2em}} % Add a gap in the Contents, for aesthetics

\backmatter

%----------------------------------------------------------------------------------------
%	BIBLIOGRAPHY
%----------------------------------------------------------------------------------------

\label{Bibliography}

\fancyhead[LO]{\textsc{Bibliography}}

\nocite{*}

\bibliography{../Bibliography.bib} % The references (bibliography) information are stored in the file named "Bibliography.bib"

\end{document}