% !TEX root = ../Main.tex

\section{Introduction}
\label{Section1}

\subsection{Kerr black hole}

The general scattering theory of perturbations off rotating black holes (BH) has long been known in General Relativity (GR) \cite{Futterman1988}.
We will take particular focus on neutral BHs since any intrinsic charge would increase electromagnetic (EM) forces on any opposite-charged plasma surrounding the BH which would quickly neutralize it.
The Kerr BH is a vacuum solution for the Einstein's field equations, which generalized the well-known Schwarzschild spherical geometry.
In this work, we use the Boyer-Linquist (BL) coordinates \cite{Boyer1967}, where the metric takes the form (natural units $c=G=\hbar=1$, signatute $+---$)
\begin{align}
\begin{split}
    ds^2 =& \left(1 - \frac{2 M r}{\rho^2} \right) \dd t^2 - 2 a \sin^2\theta \frac{(r^2+a^2-\Delta)}{\rho^2} \dd t \dd \varphi \\
    &- \frac{(r^2+a^2)^2- \Delta a^2 \sin^2\theta }{\rho^2} \sin^2\theta \dd\varphi^2 - \frac{\rho^2}{\Delta} \dd r^2 - \rho^2 \dd \theta^2 ~,
\end{split}
\end{align}
where $\Delta=r^2-2 M r +a^2$, $\bar{\rho} = r + i a \cos\theta$ and $\rho^2 \equiv |\bar{\rho}|^2$.
This geometry was found by considering commuting stationary and axisymmetric the Killing vectors ($\partial_t$ and $\partial_\varphi$ in BL coordinates) and assuming \emph{a priori} that Kerr is type D in the Petrov classification (same as Schwarzschild) \cite{Chandrasekhar1998, Wald2010}.
This BH has two intrinsic degrees of freedom, the ADM mass $M$ and angular momentum $J = a M$, which are fixed through the Komar integrals of the corresponding Killing vectors \cite{Heusler1996}.
We have radial null hypersurfaces when the norm of normal vector $g^{r \mu} \partial_\mu$ vanishes (\emph{i.e.} when $\Delta=0$), thus this geometry has an event horizon at $r=r_{+} \equiv M + \sqrt{M^2-a^2}$ and a Chauchy horizon at $r=r_{-}\equiv M - \sqrt{M^2-a^2}$. We need only to consider the region $r>r_{+}$.
Another non-trivial property of the Kerr BH is the existence of a \emph{ergoregion} where energy can be extracted (as can be exemplified by the \emph{Penrose process}) that is defined when the stationary vector $\partial_t$ becomes spacelike ($g_{tt}<0$). This is the main property that allows the process of wave superradiance. \cite{Townsend1997}

\subsection{Teukolsky equation}

It was Newman and Penrose that develop the necessary formalism of spinor calculus for the study of linearized perturbations \cite{Newman1962}.
The NP formalism focus on choosing a non-local tetrad $(\bm{l},\bm{n},\bm{m},\bm{\bar{m}})$ of complex null vectors and projecting the relevant tensors in this basis.
For example, for electromagnetic waves we use the Faraday tensor to define the relevant NP scalars
\begin{align}
    \phi_0 = F_{\mu\nu} l^\mu m^\nu ~,\qquad \phi_2 = F_{\mu\nu} \bar{m}^\mu n^\nu ~,
\end{align}
while for gravitational perturbations we use the Weyl tensor
\begin{align}
    \psi_0 = - C_{\mu\nu\sigma\rho} l^\mu m^\nu l^\sigma m^\rho ~,\qquad
    \psi_4 = - C_{\mu\nu\sigma\rho} n^\mu \bar{m}^\nu n^\sigma \bar{m}^\rho ~.
\end{align}
Choosing a suitable tetrad \cite{Kinnersley1969}, Teukolsky showed that is possible to obtain a separable wave equation for all types of massless perturbations (scalar, electromagnetic, gravitational) that are characterized by a \emph{spin-weight} parameter $s$ \cite{Teukolsky1973a, TeukolskyPress1973b, Teukolsky1974}.
Due to the underlying symmetries of the geometry we can perform a mode decomposition of the form 
\begin{align}
    \Upsilon_s = \int \dd \omega \,\sum_{\ell, m} e^{ - i \omega t + i m \varphi } \uu[s]{S}{\ell m}(\theta) \uu[s]{R}{\ell m}(r) ~,
    \label{eq1:modeDecomp}
\end{align}
where $\Upsilon_0$ obeys the Klein-Gordon wave equation in curved spacetime, $g^{\mu\nu} \nabla_\mu \partial_\nu \Upsilon_0 = 0$.
The other bosonic perturbations are characterized by two polarizations mixed differently in the corresponding complex NP quantities $\Upsilon_{+1}=\phi_0$ and $\Upsilon_{-1}=2(\bar{\rho}^*)^2 \phi_2$ for electromagnetic waves and $\Upsilon_{+2}=\psi_0$ and $\Upsilon_{-2}=4(\bar{\rho}^*)^4 \psi_4$ for gravitational waves. If the perturbations were not massless we would have an extra polarization and therefor we would not be able to separate the radial and angular equations.
From simple arguments we can show that superradiance occurs when 
\begin{align}
    \omega < m \Omega_H ~,
\end{align}
where $\Omega_H = a / (2 M r_{+})$ is the BH event horizon ``angular velocity''.

Plugging the decomposition \eqref{eq1:modeDecomp} into Teukolsky equation \cite{Teukolsky1972} obtain a angular equation which is very similar to the general Legendre equation. However, we cannot write the angular function $\uu[s]{S}{\ell m}(\theta)$ using spherical harmonics because the presence of BH angular momentum explicitly breaks spherical symmetry. Additionally, even considering the spherical case, $a=0$, for different types of perturbations we must use a spin-weighted spherical harmonics decomposition (more on \cite{TorresdelCastillo2003}), \emph{i.e.} we may write
\begin{align}
    \uu[s]{S}{\ell m}(\theta, \varphi) \equiv e^{i m \varphi} \uu[s]{S}{\ell m}(\theta) = \uu[s]{Y}{\ell m}(\theta, \varphi) + \mathcal{O}(a \omega) ~.
\end{align}
As for the radial equation it can be written as
\begin{align}
    \left[ \frac{1}{\Delta^s} \frac{\dd}{\dd r} \left( \Delta^{s+1} \frac{\dd}{\dd r}  \right) + \left( \frac{K^2 - 2 i s(r-M) K}{\Delta} + 4 i s \omega r - \lambda \right) \right] \uu[s]{R}{\ell m}(r) = 0 ~,
\end{align}
where $K=(r^2+a^2)\omega - m a$ and $\lambda = a^2 \omega^2 - 2 m a \omega + \uu[s]{\mathscr{E}}{\ell m} - s(s+1) = \ell(\ell+1)-s(s+1)+\mathcal{O}(a \omega)$ is the separation constant between the radial and angular equations. The angular eigenvalue $\uu[s]{\mathscr{E}}{\ell m}$ depends strongly on $a \omega$, for which no analytical solution was found, thus its values must be computed numerically, \emph{i.e.} with the Leaver method \cite{Leaver1985,Leaver1986}.


\clearpage